%!TEX root = anderson-etal-blackswan-timeseries.tex

\section{Abstract}

Black swans are statistically improbable events that nonetheless occur---often with profound consequences. While extremes in the physical environment, such as monsoons and heat waves, are widely studied and increasing in magnitude and frequency, it remains unclear the extent to which ecological populations buffer or suffer from such extremes. Here, we develop a probability model to estimate the degree of heavy-tailedness (presence of black swans) in ecological process noise. We apply our model to \NPops~time series from around the world across \NOrders~taxonomic orders and seven classes. We find strong evidence of black swans, but they are rare occurring in \overallMinPerc--\overallMaxPerc\% of populations: most frequently for birds (\AvesRangePerc\%) followed by mammals (\MammaliaRangePerc\%), and insects (\InsectaRangePerc\%). Black swans were predominantly downward events (\percBSDown \%) and were not explained by any life-history covariates, but tended to be driven by external perturbations such as climate, severe winters, predator or parasite cycles, and the combined effects of multiple factors. Extreme events were more frequently detected for populations with longer time series and lower levels of process noise; for shorter and noisier time series, our simulations suggested black-swan dynamics are often not identifiable as such. The presence of black swans in population dynamics highlights the importance of developing robust conservation and management strategies---particularly as the frequency and magnitude of climate extremes increase over the next century.

\section{Introduction}

Black swans are unexpected extreme events with potentially dramatic consequences \citep{taleb2007,sornette2009}. One of the most striking black swans in ecology is the possible asteroid collision marking the end-Cretaceous mass extinction 65 million years ago \citep{alvarez1980,harnik2012}. Today, climate extremes, in concert with shifts in mean temperature, are expected to cause the greatest ecological and societal damage \citep{ipcc2012}. But, while extremes in the physical environment such as wave height, storm severity, and temperature are frequent experiences \citep{gaines1993,katz2005}, it remains unclear the extent to which ecological systems buffer or suffer from black swans \citep{nunez2012}.

There is compelling anecdotal evidence for ecological black swans, but systematic evidence across taxa has been elusive. A survey of ecologists indicated that surprising outcomes of field experiments are far more common than we assume \citep{doak2008}, and events such as the global invasion of  Argentine ants and the mutation of viruses to infect new hosts could be considered black swans \citep{nunez2012}. In fact, anecdotes of population catastrophes are numerous and catastrophes may be the most important element affecting population persistence \citep{mangel1994}. For marine mammal populations, we have compelling evidence of catastrophes \citep{gerber2001, ward2007}, and recently, heavy-tailed distributions have been discovered in time-series of marine microbe abundance \citep{segura2013} and time to extinction for experimental waterflea populations \citep{drake2014}. As far as we can tell there have been two systematic surveys for ecological black swans: one with North American breeding birds and the other with the Global Population Dynamics Database, which uncovered little clear evidence for them \citep{keitt1998,allen2001,halley2002}. Nevertheless there are methodological challenges to their detection \citep{allen2001,ward2007}.

There are two key reasons why we may find little evidence of ecological black swans. First, they might not exist in higher taxa. Indeed, the majority of model fitting and risk forecasting assumes that population dynamics are not heavy tailed \citep[e.g.][]{brook2006a,dennis2006,knape2012}. Alternatively, heavy-tailed dynamics might exist, but our ability to detect them requires further development of statistical tools. One such tool is the generalized extreme value distribution, which has been applied to environmental data \citep[e.g.][]{katz2005}. This distribution describes the most extreme event per time interval (e.g.~heaviest rainfall per year), but requires time series that are sufficiently long to be condensed into time intervals. Another statistical tool involves fitting a catastrophic mixture distribution in a state space model to quantify the probability that population events are extreme, although this is also data intensive \citep{ward2007}. A third tool is to compare the support for fits of thin- and heavy-tailed distributions \citep{halley2002}, but this analysis did not quantify the probability of black swans or allow for population dynamics.

Here we develop a framework for identifying ecological black swans in the process noise of population dynamics, i.e.\ the stochastic jumps in abundance from one time step to the next. Our framework quantifies both the magnitude and probability of heavy-tailed dynamics, allows for a range of population dynamic models, can incorporate observation uncertainty, and can be easily applied to abundance time series. We apply our model to \NPops\ populations and estimate the frequency and strength of black-swan events in population dynamics across taxonomy; verify some causes of black-swan events; and attempt to identify characteristics of time series or intrinsic life-history characteristics that are associated with heavy-tailed dynamics.

\section{Methods}

To obtain estimates of the probability and magnitude of black-swan events, we fit population dynamic models to abundance time series from around the world. For each population, we estimated the shape of the process noise tails by measuring the degrees of freedom ($\nu$) of a Student t-distribution.

\subsection{Time-series data}

We selected abundance time series from the Global Population Dynamics Database (GPDD; \citeauthor{gpdd2010} \citeyear{gpdd2010}), which contains nearly 5000 time series of abundance from $\sim$1000 species and $\sim$100 taxonomic orders. We filtered the data (Supporting Information) to remove populations from less reliable data sources, and those without sufficient data for our models, and then interpolated some missing values \citep[\textit{sensu}][]{brook2006a}. Our interpolation affected only $\sim$\interpPointsPerc \% of the final data points (Table~\ref{tab:stats}) and none of the data points that were later considered black-swan events. Our final dataset contained \NPops~populations across \NOrders~taxonomic orders and seven taxonomic classes, with a median of \medianTimeSteps~time steps (range of \minTimeSteps--\maxTimeSteps) (Table~\ref{tab:stats}, Fig.~\ref{fig:all-ts}).

\subsection{Population models}

Our main analysis focuses on the commonly applied Gompertz population dynamics model \citep[e.g.][]{knape2012,dennis2014,connors2014}. The Gompertz model represents population growth as a linear function in $\log$ space. If we let $x_t$ represent the $\log$ abundance ($N$) at time $t$, we can represent the Gompertz model as:
\begin{align*}
x_t &= \lambda + b x_{t-1} + \epsilon_t\\
\epsilon_t &\sim \mathrm{Student\mhyphen t}(\nu, 0, \sigma).
\end{align*}
The growth parameter $\lambda$ represents the expected growth rate if $N_t = 1$. The model is density independent if $b = 1$, maximally density dependent if $b = 0$, and inversely density dependent if $b < 0$. Usually, the process noise $\epsilon_t$ is modelled as normally distributed, but in our paper we assume it is drawn from a t distribution with scale parameter $\sigma$ and degrees of freedom $\nu$. In previous analyses of the GPDD, the Gompertz is most often identified as the most parsimonious population model that can be fit to these data \citep{brook2006}.

By allowing the process noise to be drawn from a Student-t distribution we can estimate the degree to which the process deviations have heavy tails and are thereof evidence of black-swan events (Fig.~\ref{fig:didactic}a, b). For example, at $\nu = 3$, the probability of drawing a value more than five standard deviations below the mean is $0.02$, whereas the probability of drawing such a value from a normal distribution is tiny ($2.9\cdot10^{-7}$). As the value of $\nu$ approaches infinity, the distribution approaches the normal distribution (Fig.~\ref{fig:didactic}a, b).

We fit all models in a Bayesian framework using Stan \citep{stan-manual2014} via the R computing environment \citep{r2014}. Stan samples from the posterior distribution with an adaptive version of Hamiltonian Markov chain Monte Carlo called the No-U-Turn Sampler and generally obtains less correlated samples than algorithms such as the Gibbs sampler \citep{hoffman2014}. We tested to ensure that the chains had sufficiently converged and that the sampler had obtained sufficient independent samples from the posterior ($\widehat{R} < 1.05$, $n_\mathrm{eff} > 200$; Supporting Information).

We chose weakly informative priors to incorporate our understanding of plausible population dynamics (\citeauthor{gelman2014} \citeyear{gelman2014}; Figs~\ref{fig:didactic}f--h, \ref{fig:priors}). For $\nu$, we chose an exponential prior with rate parameter of $0.01$ truncated at values above two---a slightly less informative prior than suggested by \citet{fernandez1998}. This prior gives only a \basePriorProbHeavy \% probability that $\nu < 10$ but constrains the sampling sufficiently to avoid wandering off towards infinity. In any case, for $\nu > 20$ the t distribution is almost indistinguishable from the normal distribution (Fig.~\ref{fig:didactic}). We investigated the sensitivity of our results to weaker and stronger priors (exponential rate parameter $= 0.005, 0.02$; Fig.~\ref{fig:priors}, Supporting Information). We used simulated data to test how easily we could detect $\nu$ given different sample sizes and to ensure we could recover unbiased parameter estimates from the Gompertz model (Supporting Information).

We fit alternative population models to test if four key phenomena systematically changed our conclusions. Autocorrelation has been suggested as a reason for increased observed variability of abundance time series through time, which could create apparent heavy tails \citep{inchausti2002}; therefore, we fit a model that included serial correlation in the residuals. Additionally, previous work has modelled abundance or growth rates without accounting for density dependence \citep{halley2002,segura2013}; therefore, we fit a simpler model in which we assumed density independence. Third, observation error could bias parameter estimates \citep{knape2012} or mask our ability to detect heavy tails \citep{ward2007}; therefore, we fit a model where we allowed for a fixed quantity of observation error ($0.2$ standard deviations on a log scale). Finally, the Gompertz model assumes that population growth rate declines linearly with log abundance. We also fit an alternative model, the Ricker-logistic model, which assumes that population growth rate declines linearly with abundance itself (Supporting Information).

\subsection{Covariates of population dynamic black swans}

We investigated possible covariates of heavy-tailed population dynamics visually and through multilevel modelling. We plotted characteristics of the time series ($\sigma$, $\lambda$, $b$, and time-series length) along with two life-history characteristics (body length and maximum lifespan obtained from \citet{brook2006a}) against our estimated probability of heavy tails, Pr$(\nu < 10) > 0.5$. We formally investigated these relationships by fitting beta regression multilevel models \citep{ferrari2004}. For covariates that were derived from Gompertz model parameter estimates, we incorporated standard deviations around the means. To account for broad patterns of phylogenetic relatedness, we allowed for hierarchical intercepts at the taxonomic class, order, and species level (Supporting Information). We fit our model in Stan with weakly informative priors on the coefficients \citep{gelman2008d} and variance parameters (\citeauthor{gelman2006c} \citeyear{gelman2006c} \citeauthor{gelman2014} \citeyear{gelman2014}; Supporting Information).

Finally, we investigated a sample of populations that our method categorized as having a high probability of heavy tails (Pr$(\nu < 10) > 0.5$). Where possible, we found the documented causes of ecological black swans in the primary data source cited in the GPDD or in other literature describing the population.

\section{Results}

We found strong, but rare, evidence for black-swan population dynamics. By defining black-swan dynamics as a greater than $0.5$ probability that $\nu < 10$, our main Gompertz model found evidence for heavy tails most frequently for birds (\birdPH\%) followed by mammals (\mammalsPH\%), insects (\insectsPH\%), and fishes (\fishPH\%) (Fig.~\ref{fig:nu-coefs}, Table~\ref{tab:causes-supp}). Black swans were taxonomically widespread, occurring in \POrdersHeavy\% of taxonomic orders. Accounting for time series length and partially pooling inference across taxonomic class and order with a multilevel model, there was stronger evidence for black swans in insect populations than is visually apparent in Fig.~\ref{fig:nu-coefs}---four of 10 orders with the highest median probability of heavy tails were insect orders---however, there was considerable uncertainty in these estimates (Fig.~\ref{fig:posteriors}a).

The majority of our heavy-tailed estimates were robust to alternative population models, observation error, and choice of priors. Our conclusions were not systematically altered when we included an autocorrelation structure in the residuals, modelled population growth rates without density dependence, or modelled the population dynamics as Ricker-logistic (Fig.~\ref{fig:alt}). However, setting observation error standard deviation to $0.2$ increased the median estimate of $\nu$ from $<10$ to $\ge 10$ in $\baseNuTenObsTenSwitch$ of $\baseNuTen$ populations, although the majority of $\nu$ estimates remained qualitatively similar (Fig.~\ref{fig:alt}). The strength of the prior on $\nu$ had little influence on estimates of black-swan dynamics (Fig.~\ref{fig:alt-priors}). Our simulation testing shows that, if anything, our models underpredict the true magnitude and probability of heavy tailed events---especially given the length of the time series in the GPDD (Figs~\ref{fig:sim-nu}, \ref{fig:sim-prob}).

Across populations, the probability of observing black-swan dynamics was positively related to time-series length and negatively related to magnitude of process noise ($\sigma$) but not clearly related to population growth rate ($\lambda$), density dependence ($b$), or maximum lifespan (Figs~\ref{fig:correlates}, \ref{fig:posteriors}b). Longer time-series length was the strongest covariate of observing black-swan dynamics. For instance, we were about \pIncHeavyNThirtyNSixty~times as likely to observe heavy tails in a population with 60 time steps compared to one with 30 time steps. However, the absolute change in probability with increased time series length was small ($\pHeavyNSixty$ vs.\ $\pHeavyNThirty$ in the previous example, Fig.~\ref{fig:correlates}).

We examined all time series with published explanations of why the black-swan events occurred (Tables~\ref{tab:sparks} and \ref{tab:causes-supp}). The majority (\percBSDown \%) of documented events were downward black swans and involved a series of unfortunate events. For example, a shortage of nest sites for the shag long-necked bird (\textit{Phalacrocorax aristotelis}) in the United Kingdom reduced population productivity and made the population more vulnerable when a red tide hit in 1968 \citep{potts1980}. The population experienced a black-swan downswing, but the new availability of nest sites resulted in a rapid population upswing \citep{potts1980}. As another example, an interaction between environmental- and predation-mediated population cycles are thought to have caused a downward black-swan event for a water vole (\textit{Arvicola terrestris}) population \citep{saucy1994}. Other black swans were the result of a sequence of extreme climate events on their own. For instance, severe winters in 1929, 1940--1942, and 1962--1963 were associated with black-swan downswings in grey heron (\textit{Ardea cinerea}) abundance in the United Kingdom \citep{stafford1971}. Our analysis finds that the last event was a combination of two black-swan events in a row and it took the population three times longer to recover than predicted \citep{stafford1971}. \textit{(TODO: substitute a parasite example?)}

%TODO IMPORTANT quantify across all identified black swan events how many were downward and how many upward. The assumption of your model is a symmetric distribution, but if in fact downward events are more common, perhaps a skewed distribution would be more realistic. Furthermore, if there is skew this would *increase* the probability of higher extinction risk. Something to definitely highlight in the results and abstract.

\section{Discussion}

We found strong evidence for black swans (heavy-tailed process noise) in 3--5\% of ecological time series. Black swans were usually downward events and were detected more frequently in longer time series and in populations with a smaller magnitude of process noise. Black swans were not associated with density dependence, population growth rate, or lifespan. In verified cases, black-swan events were often a result of the interaction between elements of extreme climate, predators and parasite cycles, and strong changes in human pressures. Our empirical results, sensitivity analyses, and simulation tests suggest that estimating the tail shape of process noise is a viable method of detecting black-swan population dynamics and if anything will underestimate the probability of black-swan events. The presence of black swans highlights the importance of developing management strategies that detect quickly, respond to, and are robust to extremes in population dynamics---particularly as the frequency and magnitude of climatic extremes increase over the next century \citep{easterling2000,ipcc2012}.

Our results clarify previous related analyses. An analysis with an older version of the GPDD assessed the distribution of abundance time series but focused on identifying if the log-normal distribution was the most frequently parsimonious model \citep{halley2002}. For heavy-tailed distributions, \citet{halley2002} fit the extremely heavy-tailed Cauchy distribution and the four-parameter Levy stable distribution and found little information criteria support for these distributions in longer time series. However, black-swan events are by definition rare, and the majority of time series in the GPDD are short for these purposes. Therefore, we would not expect to observe black swans in a large proportion of populations. By probabilistically quantifying process noise tails on a continuous spectrum, and allowing for non-stationary time series and density dependence, our analysis allows for a more nuanced description of the evidence of ecological black swans. In an earlier study, \citet{keitt1998} described heavy (power law) tails in breeding bird population abundance. However, this finding was challenged by \citet{allen2001}, who showed that mixing data across species could falsely generate heavy tails. This rebuttal supports our use of the t distribution to represent heavy tails, since a t distribution can be represented by a mixture of normal distributions \citep[with the same mean and inverse-gamma-distributed variances,][]{gelman2014}.

We might expect to observe black-swan dynamics in ecological time series because of unmodelled intrinsic properties of populations or extrinsic forces acting on populations. Since a t-distribution can be formed by a mixture of normal distributions \citep{gelman2014}, we could observe heavy tails if we miss some underlying mixture of intrinsic processes \citep{allen2001}. That process might be an aggregation of populations across space, or population diversity, or an intrinsic change in population variability through time. Extrinsic forces could also cause black-swan dynamics \citep[e.g.][]{nunez2012}. These forces could be extreme themselves. For example, extreme climate, predation from (or competition with) other species experiencing black swans, or sharp changes in human pressure such hunting, fishing, or habitat destruction might cause black swans. Alternatively, the interaction of multiple ``normal'' extrinsic forces could give rise to black-swan ecological dynamics. This could occur if the interaction is synergistic \citep[e.g.][]{kirby2009} or even if non-synergistic interactions experience a rare alignment \citep{denny2009}.

There are a number of caveats when considering the generality of our results. The GPDD data represent a taxonomically and geographically biased sample of populations---the longer time series we focus on are dominated by commercially and recreationally important species and a disproportionate number of populations are located in the United Kingdom. Although we would expect to find qualitatively similar evidence for black swans in many other large taxonomic or geographic samples of populations, the common forces driving those black swans would likely differ (e.g.~severe winters in Table~\ref{tab:sparks}). In addition to a possibly biased sample of populations, some black-swan detections could just be recording mistakes, or conversely, some extreme observations may have been discarded or altered if they were erroneously suspected of being recording mistakes. Indeed, three of the populations our method initially identified as heavy tailed turned out to be data-entry errors and were discarded (Supporting Information). A final caveat is that the temporal scales of observation and population dynamics vary considerably across populations in the GPDD and this likely influences the detection of heavy tails. As an example, if we make frequent observations relative to generation time (e.g.~for many large-bodied mammals) we will average across generations and perhaps miss black swans. Conversely, if we census populations infrequently relative to generation time (e.g.~many insects in the GPDD) the recorded data may average across extreme and less-extreme events and also dampen black-swan dynamics.

Recognizing the prevalence of heavy-tailed dynamics suggests a number of policy directions. Ecological resource management can draw from other disciplines that focus on heavy tails. For example, earthquake preparedness and response is focussed on black-swan events. Similarly to ecological black swans, we can rarely predict the specific location and timing of large earthquakes. But, earthquake preparedness involves spatial planning based on forecast probabilities to focus early detection efforts and develop disaster response plans. A similar focus might benefit resource management once our ability to predict the spatial probability and covariates of ecological black swans improves. The presence of black swans also suggests that we develop management policy that is robust to heavy tails. For instance, setting target population abundances that are appropriately set back from critical limits may buffer black-swan events \citep[e.g.][]{caddy1996}, and maintaining genetic, phenotypic, and behavioural diversity may allow some components of populations persist when others are affected by disease or extreme environmental forces \citep[e.g.][]{hilborn2003, schindler2010, anderson2014}. Finally, extreme and unexpected, surprising, or counterintuitive ecological dynamics offer a tremendous opportunity to learn about ecological systems, evaluate when our models break down, and adjust future management policy \citep{doak2008, pine-iii2009, lindenmayer2010}.

Our results suggest a number of research questions related to ecological black swans. Given that black swans do occur, can we forecast the probability of black swans in space and time? Furthermore, what management policies allow us to detect them quickly after they happen? Can we isolate the components of ecological dynamics that experience black swans by moving from phenomenological models such as the Gompertz to mechanistic models, that take into account, for example, recruitment? We expect that greater insight into the mechanisms and covariates of ecological black swans may be best obtained through specific geographic and taxonomic subsets of data where longer time series with low levels of observation error are available \citep[e.g.][]{segura2013}.

Most importantly, what is the impact of allowing for black swans in forecasts of ecological risk? In recent decades, ecology has moved toward focussing on aspects of variance in addition to mean responses \citep[e.g.][]{loreau2010a, thompson2013}. Our results suggest that an added focus on ecological extremes represents the next frontier, particularly in the face of increased climate extremes \citep{meehl2004,ipcc2012}. Financial analysts are concerned with the shape of financial-return-distribution tails because these impact estimates of risk---the probability of a specific magnitude of undesired event occurring \citep{rachev2008}. A comparable focus in ecology would increase our estimates of extinction risk, since these would be disproportionately impacted by downward black-swan events.

\section{Acknowledgements}

We thank Justin D. Yeakel, TODO, and other members of the Earth to Ocean research group for helpful discussions and comments. We are grateful to the contributors and maintainers of the Global Population Dynamics Database and to Compute Canada's WestGrid high-performance computing resources. Silhouette images were obtained from \texttt{phylopic.org} under Creative Commons licenses; sources are listed in the Supporting Information. Funding was provided by a Simon Fraser University Graduate Fellowship (SCA), the Natural Sciences and Engineering Research Council of Canada (NKD, ABC), the Canada Research Chairs Program (NKD).

\section{Supporting Information}

The following supporting information is available online for this article:\\
Tables S1 and S2\\
Figures S1--S6\\
Example Stan code for a heavy-tailed Gompertz model and multilevel beta
regression\\
The GPDD IDs used in our analysis\\
All code and data to recreate our analysis are available at:\\
\url{https://github.com/seananderson/heavy-tails}

\bibliographystyle{ecologyletters}
\bibliography{/Users/seananderson/Dropbox/tex/jshort,/Users/seananderson/Dropbox/tex/ref3}

\clearpage

\section{Tables}

\LTcapwidth=\textwidth
\bibpunct{}{}{;}{a}{}{;}

\begin{small}
\begin{longtable}{>{\RaggedRight}m{2.0cm}>{\RaggedRight}p{3.0cm}>{\RaggedRight}p{7.0cm}>{\RaggedRight}p{2.0cm}}

\caption{Example population dynamic black swans from the Global Population
  Dynamics Database and descriptions of their causes. Red and blue dots
  indicate downward and upward events that have a probability of occurring of
  0.001\% or less if the population dynamics were explained by a Gompertz model
  with normally distributed process noise. These populations are a non-random
  sample that we were able to verify in the primary literature.}\\

\toprule
Time series (log scale) & Population & Black swan description & Reference \\
\midrule

\includegraphics[width=2cm]{../analysis/sparks/6528} &
Shag,
\textit{Phalacrocorax aristotelis},
UK &
Shortage of nest sites reduced productivity; red-tide event in 1968 caused
extreme mortality; no longer a nest shortage; population rapidly increased &
\citep{potts1980}\\

\includegraphics[width=2cm]{../analysis/sparks/10007} &
Water vole,
\textit{Arvicola terrestris},
UK &
Short-term population cycles from predator interactions combined with longterm
environmental cycle caused sharp downswing  &
\citep{saucy1994}\\

\includegraphics[width=2cm]{../analysis/sparks/7115} &
Fur seal,
\textit{Arctocephalus pusillus},
South Africa &
Strong decreases in harvesting, loss of predators, and diamond mining
regulations reducing human traffic caused sharp upswings  &
\citep{shaughnessy1982}\\

\includegraphics[width=2cm]{../analysis/sparks/10113} &
Willow grouse,
\textit{Lagopus lagopus},
UK &
Parasite and predation effects interacted to cause low years  &
\citep{dobson1995}\\

\includegraphics[width=2cm]{../analysis/sparks/10162} &
Red grouse,
\textit{Lagopus lagopus scoticus},
UK &
Good environmental conditions produced high numbers and vulnerable populations;
bad conditions and overcrowding combined to create crashes  &
\citep{mackenzie1952}\\

\includegraphics[width=2cm]{../analysis/sparks/1235} &
Wren,
\textit{Troglodytes troglodytes},
UK &
Severe winters where food was submerged in snow caused population crash &
\citep{newton1998} \\

\includegraphics[width=2cm]{../analysis/sparks/20579} &
Grey heron,
\textit{Ardea cinerea},
UK &
Severe winters in 1929, 1940--1942, and 1962--1963; 1963 event so severe that
recovery took three times as long as expected &
\citep{stafford1971} \\

%\includegraphics[width=2cm]{../analysis/sparks/20580} &
%Chamois, \textit{Rupicapra rupicapra}, Switzerland &
% &
%\citep{brook2006a}\\

%\includegraphics[width=2cm]{../analysis/sparks/5019} &
%Barbary macaque,  &
% &
%REF\\

%\includegraphics[width=2cm]{../analysis/sparks/9675} &
%Carrot fly (\textit{Psila rosae}, Finland &
% &
%\citep{markkula1965}\\

%\includegraphics[width=2cm]{../analysis/sparks/10139} &
%Grey heron (\textit{Ardea cinerea}), UK &
% &
%\citep{stafford1971}\\

\bottomrule
\label{tab:sparks}
\end{longtable}
\end{small}

% reset citation style:
\bibpunct{(}{)}{;}{a}{}{;}


\clearpage

\begin{figure}[htbp]
\begin{center}
\includegraphics[width=\textwidth]{../analysis/t-nu-eg2.pdf}
\caption{
An illustration of fitting population dynamic models that allow for heavy tails, represented by the Student-t degrees of freedom parameter $\nu$. (a, b) The probability density for t distributions with a scale parameter of 1 and different values of $\nu$. Small values of $\nu$ create heavy tails. As $\nu$ approaches infinity the distribution approaches the normal distribution. For example, at $\nu = 2$, the probability of drawing a value more than five standard deviations below the mean is 1.8\%, whereas the probability of drawing such a value from a normal distribution is nearly zero ($2.9\cdot10^{-5}$\%). (c--e) Simulated population dynamics from a Gompertz model with process noise drawn from t distributions with three different values of $\nu$. Coloured dots in panels c and d represent jumps with less than a 1 in 1000 chance of occurring in a normal distribution. (f--h) Estimates of $\nu$ from models fit to the times series in panels c--e. Shown are the posterior samples (histograms), median and interquartile range of the posterior (IQR) (dots and line segments), and the exponential prior on $\nu$ (dashed lines). Colour shading behind panels f--h illustrates the region of heavy tails.
}
\label{fig:didactic}
\end{center}
\end{figure}

\clearpage

\begin{figure}[htbp]
\begin{center}
\includegraphics[width=0.44\textwidth]{../analysis/nu-coefs-2.pdf}
\caption{
Estimates of how heavy-tailed population dynamics are for \nuCoefPopN\ populations of birds, mammals, insects, and fishes. Small values of $\nu$ ($\lesssim 10$) suggest heavy-tailed black-swan dynamics; larger values of $\nu$ suggest approximately normal-tailed dynamics. Vertical points and line segments represent posterior medians and 50\% / 90\% credible intervals for individual populations. Inset plots show probability that $\nu < 10$ (probability of heavy tails) for populations arranged by taxonomic order and sorted by decreasing mean Pr($\nu < 10$). Taxonomic orders with three or fewer populations in panel a are omitted for space. Red to yellow points highlight populations with a high to moderately high probability of heavy-tailed black-swan dynamics.
}
\label{fig:nu-coefs}
\end{center}
\end{figure}
\clearpage

\begin{figure}[htbp]
\begin{center}
\includegraphics[width=0.9\textwidth]{../analysis/correlates-p10.pdf}
\caption{
Potential covariates of heavy-tailed population dynamics (indicated by a high probability that $\nu < 10$). Shown are (a--c) parameters from the Gompertz heavy-tailed population model ($\sigma$, $\lambda$, $b$), (b) number of time steps, (c) body length, and (d) lifespan. For the Gompertz parameters, $\sigma$ refers to the scale parameter of the Student-t process-noise distribution, $\lambda$ refers to the expected log abundance at the next time step at an abundance of one, $b$ refers to the density dependence parameter ($1$ is maximally density independent, $0$ is maximally density dependent, and $<0$ is inversely density dependent). Circles representing a few sharks, crustaceans, and gastropods are filled in white. Median and 90\% credible interval posterior predictions of a beta regression multilevel model are shown in panels a and d where there was a high probability the slope coefficient was different from zero (Fig.~\ref{fig:posteriors}b).
}
\label{fig:correlates}
\end{center}
\end{figure}

\begin{figure}[htbp]
\begin{center}
\includegraphics[width=0.45\textwidth]{../analysis/order-posteriors-covariates.pdf}
\caption{
Posterior probability distributions from beta regression multilevel models.
(a) Taxonomic-order-level posterior densities of Pr($\nu < 10$) (approximately the probability of heavy tails) after accounting for time-series length. Estimates are at the geometric mean of time series length across all the data (approximately 27 time steps). Colour shading refers to taxonomic class (yellow: fishes, green: insects, purple: birds, and red: mammals). Long dotted vertical line in panel a indicates the median expected Pr($\nu < 10$) from the prior distribution.
(b) Main effect posterior densities for potential covariates of Pr($\nu < 10$). The beta regression models were fit on a logit scale with hierarchical intercepts for taxonomic class, taxonomic order, and species. All covariates were standardized by subtracting their mean and dividing by twice their standard deviation.
In both panels, short vertical line segments indicate median posterior estimates.
}
\label{fig:posteriors}
\end{center}
\end{figure}
