\documentclass[9pt,twocolumn,twoside]{pnas-new}
% Use the lineno option to display guide line numbers if required.
% Note that the use of elements such as single-column equations
% may affect the guide line number alignment.

\templatetype{pnasresearcharticle} % Choose template
% {pnasresearcharticle} = Template for a two-column research article
% {pnasmathematics} = Template for a one-column mathematics article
% {pnasinvited} = Template for a PNAS invited submission

\newcommand{\tabstats}{{}{28}}
\newcommand{\tabcausessupp}{{2}}
\newcommand{\figpriors}{{1}}
\newcommand{\figsimnu}{{2}}
\newcommand{\figsimprob}{{3}}
\newcommand{\figalt}{{4}}
\newcommand{\figaltpriors}{{5}}
\newcommand{\figcorrelates}{{6}}
\newcommand{\figskewnu}{{7}}
\newcommand{\figskewprojections}{{8}}
\mathchardef\mhyphen="2D % math hyphen

\usepackage{bm}

\newcommand{\basePriorMean}{102}
\newcommand{\basePriorMedian}{71}
\newcommand{\basePriorProbHeavy}{7.7}
\newcommand{\medianTimeSteps}{26}
\newcommand{\meanTimeSteps}{30.2}
\newcommand{\minTimeSteps}{20}
\newcommand{\maxTimeSteps}{117}
\newcommand{\birdN}{191}
\newcommand{\insectsN}{182}
\newcommand{\mammalsN}{125}
\newcommand{\fishN}{108}
\newcommand{\birdNH}{14}
\newcommand{\insectsNH}{5}
\newcommand{\mammalsNH}{6}
\newcommand{\fishNH}{0}
\newcommand{\birdPH}{7}
\newcommand{\insectsPH}{3}
\newcommand{\mammalsPH}{5}
\newcommand{\fishPH}{0}
\newcommand{\NOrdersHeavy}{16}
\newcommand{\POrdersHeavy}{41}
\newcommand{\baseFiftyObsFiftySwitch}{8}
\newcommand{\baseSeventyFiveObsFiftySwitch}{2}
\newcommand{\totalHeavyFifty}{26}
\newcommand{\totalHeavySeventyFive}{17}
\newcommand{\baseFiftyObsFiftySwitchPerc}{31}
\newcommand{\baseSeventyFiveObsFiftySwitchPerc}{8}
\newcommand{\baseNuTenObsTenSwitch}{8}
\newcommand{\baseNuTen}{26}
\newcommand{\baseNuFiveObsTenSwitch}{2}
\newcommand{\pHeavyNThirty}{0.13}
\newcommand{\pHeavyNSixty}{0.20}
\newcommand{\pIncHeavyNThirtyNSixty}{1.6}
\newcommand{\obsErrorNuFivePerc}{75}
\newcommand{\modelsNoConvergeAROne}{1}
\newcommand{\modelsNoConvergeAROneHeavyBase}{0}
\newcommand{\percImputedPops}{17}
\newcommand{\percImputedPoints}{0.7}
\newcommand{\nuCoefPopN}{606}
\newcommand{\AvesRangePerc}{4--8}
\newcommand{\InsectaRangePerc}{2--3}
\newcommand{\MammaliaRangePerc}{4--6}
\newcommand{\OsteichthyesRangePerc}{0}
\newcommand{\overallMinPerc}{3}
\newcommand{\overallMaxPerc}{5}
\newcommand{\overallBasePerc}{4}
\newcommand{\NPops}{609}
\newcommand{\NOrders}{39}
\newcommand{\NClasses}{7}
\newcommand{\interpPointsPerc}{1}
\newcommand{\nBSUp}{8}
\newcommand{\nBSDown}{51}
\newcommand{\ratioBSDownToUp}{6.4}
\newcommand{\percBSDown}{86}
\newcommand{\crashUnderRange}{1.1--2}
\newcommand{\crashUnderMedian}{1.3}
\newcommand{\probDensSkewedForHeavyPops}{86}
\newcommand{\percNormPopsNotSkewed}{88}
 % R output

\title{Black-swan events in animal populations}

% Use letters for affiliations, numbers to show equal authorship (if applicable) and to indicate the corresponding author
\author[a,b]{Sean C. Anderson}
\author[b]{Trevor A. Branch}
\author[c]{Andrew B. Cooper}
\author[a]{Nicholas K. Dulvy}

\affil[a]{Earth to Ocean Research Group, Department of Biological
Sciences, Simon Fraser University, Burnaby BC, V5A 1S6, Canada}
\affil[b]{School of Aquatic and Fishery Sciences, Box 355020,
University of Washington, Seattle, WA 98195, USA}
\affil[c]{School of Resource and Environmental Management, Simon
Fraser University, Burnaby, BC, V5A 1S6, Canada}

% Please give the surname of the lead author for the running footer
\leadauthor{Anderson}

% Please add here a significance statement to explain the relevance of your work
\significancestatement{Black swans---statistically improbable events
with profound consequences---happen
more often than expected in financial, social, and natural systems. Our work
demonstrates the rare but systematic presence of black-swan events in animal populations
around the world (mostly birds, mammals, insects, and fishes). These events are
predominantly downward, implying that unexpected population crashes occur more
frequently than increases. Black-swan events are not driven by life history (e.g.\ lifespan),
but by external causes such as extreme winters and disease. Ignoring the presence of
downward black swans may severely underestimate extinction risk in animal populations,
particularly under a changing climate where such extreme events are expected to increase
in frequency and magnitude.}

% Please include corresponding author, author contribution and author declaration information
\authorcontributions{S.C.A. and T.A.B. conceived the project;
S.C.A., T.A.B., A.B.C., and N.K.D. designed the study; S.C.A. analyzed the
data and wrote the paper with input from all authors.}
\authordeclaration{Please declare any conflict of interest here.}
\correspondingauthor{\textsuperscript{2}To whom correspondence should be addressed. E-mail: sandrsn@uw.edu}

% Keywords are not mandatory, but authors are strongly encouraged to provide them. If provided, please include two to five keywords, separated by the pipe symbol, e.g:
\keywords{
Bayesian $|$
die-offs $|$
ecological surprises $|$
Gompertz $|$
heavy tails $|$
leptokurtic $|$
mass mortality $|$
population dynamics $|$
risk $|$
Student-$t$ distribution
}

\begin{abstract}
Black swans are improbable events that nonetheless
occur---often with profound consequences. 
% While physical extremes, such as
% monsoons and heat waves, are widely studied and increasing in magnitude and frequency, 
Such events drive important transitions in social systems (e.g.\ banking
collapses) and physical systems (e.g.\ earthquakes)
yet it remains unclear the extent to which ecological population numbers
buffer or suffer from such extremes. 
Here we estimate the prevalence and
direction of black-swan events (heavy-tailed process noise)
in \NPops\ animal populations
after accounting for population dynamics (productivity, 
density dependence, and typical stochasticity).
We find strong evidence for 
black-swan events in \(\sim\)\overallBasePerc \% of populations.
These events occur most frequently for birds (\birdPH \%), mammals
(\mammalsPH \%), and insects (\insectsPH \%),
and are not explained by any life-history covariates, but
tend to be driven by external perturbations such as climate, severe winters,
predators, parasites, or the combined effects of multiple factors.
Black-swan events manifest primarily as
population die-offs and crashes (\percBSDown \%) rather than unexpected increases,
and ignoring heavy-tailed process noise leads to 
% a \crashUnderMedian-fold (50\% interval \crashUnderRange) 
an underestimate in the magnitude of population crashes.
We suggest modelers consider heavy-tailed downward-skewed 
probability distributions, 
such as the skewed Student-$t$ used here, 
when making forecasts of population abundance.
Our results demonstrate the importance of both
modeling heavy-tailed downward events in populations, and developing
conservation strategies that are robust to ecological surprises.
\end{abstract}

\dates{This manuscript was compiled on \today}
\doi{\url{www.pnas.org/cgi/doi/10.1073/pnas.XXXXXXXXXX}}

\begin{document}

% Optional adjustment to line up main text (after abstract) of first page with line numbers, when using both lineno and twocolumn options.
% You should only change this length when you've finalised the article contents.
\verticaladjustment{-2pt}

\maketitle
\thispagestyle{firststyle}
\ifthenelse{\boolean{shortarticle}}{\ifthenelse{\boolean{singlecolumn}}{\abscontentformatted}{\abscontent}}{}

% If your first paragraph (i.e. with the \dropcap) contains a list environment (quote, quotation, theorem, definition, enumerate, itemize...), the line after the list may have some extra indentation. If this is the case, add \parshape=0 to the end of the list environment.
\dropcap{M}ajor surprises happen more often than expected in
financial, social, and natural systems \cite{taleb2007, sornette2009, may2008}.
Massive unpredictable market swings are responsible for the majority of
financial gains and losses \cite{taleb2007}, fatalities from the largest wars
dwarf those from all others  \cite{newman2005}, and the frequency of the most
damaging earthquakes has exceeded past expectations \cite{sornette2009}. In
ecological systems, background rates of global extinction are punctuated by
mass extinction \cite{harnik2012}, viruses can mutate suddenly to infect new
hosts, and billions of animals can die at once in mass mortality
events \cite{fey2015}.

Indeed, such die-offs may be the most important element
affecting population persistence \cite{mangel1994} and their importance is
likely to increase given projected increases in the frequency and magnitude of
climate-related extremes \cite{ipcc2012}.
% Despite this anecdotal evidence for
% ecological black swans, systematic searches of large numbers of timeseries have
% yet to reveal black-swan events \cite{keitt1998, allen2001, halley2002}.
However, the overwhelming majority of population model fitting and risk forecasting
assumes that deviations from model predictions can be represented by a normal
distribution (on a log scale; e.g.~\cite{brook2006a, knape2012}).
If black swans occur, though, a normal distribution would under-estimate the
probability of extreme events occurring \cite{taleb2007}.

Whereas there are many reports of black-swan events,
only a flexible comparative approach consistently applied
to a large number of time series can yield insights into the frequency, strength,
and correlates of black-swan events.
We are unaware of such a comparative analysis.
Previous comparative analyses fitting
heavy-tailed distributions
(distributions with higher probabilities of extreme events
than the normal distribution)
to time series
have not accounted for an underlying population dynamics model
\cite{keitt1998, allen2001, halley2002, segura2013}.
Alternatively, most examples of population die offs come from identifying
sudden changes in abundance that exceed a chosen threshold of decline
\cite{young1994, gerber2001, reed2003, fey2015}.
But this approach
does not distinguish events from
expected dynamics.
For instance, the range of natural variability
can differ dramatically across taxa.

\begin{figure*}[htb]
\centering
\includegraphics[width=0.65\textwidth]{../analysis/t-nu-eg2.pdf}
\caption{ Illustration of population dynamic models that allow for heavy
tails. (A, B) Probability density for the Student-$t$
distribution with scale parameter of $1$ and different values of \(\nu\). Small
values of \(\nu\) create heavy tails while as \(\nu\) approaches infinity the
distribution approaches the normal distribution. (C to E)
Simulated population dynamics from a Gompertz model with process noise drawn
from Student-$t$ distributions with three values of \(\nu\). Coloured dots in
panels C and D represent jumps with less than a 1 in 1000
chance of occurring in a normal distribution. (F to H)
Estimates of \(\nu\) from models fit to the times series in panels C to
E. Shown are posterior samples (histograms), median and interquartile
range of the posterior (IQR, dots and line segments), and the exponential prior
on \(\nu\) (dashed lines). Colour shading behind panels F to H
illustrates the approximate region of heavy tails.}
\label{fig:1}
\end{figure*}



Here, we develop a new approach to estimate the frequency and magnitude of
black-swan dynamics across time series of 609 populations from a wide array of
taxonomic groups---including many  birds, mammals, insects, and fishes
(Table~S1). We identify characteristics of time series or
life-history traits associated with the detection of black-swan events and
verify known causes. To accomplish this, we develop a framework for identifying
heavy-tailed (black-swan) process noise in population dynamics.
We test if the largest stochastic jumps in log abundance from one time step to the next
are more extreme than typically seen with a normal distribution. Our framework
allows for a range of population dynamic models, can incorporate observation
uncertainty and skewness in process noise, and can be readily applied to
abundance time series.

We fit population dynamics models in which the process noise is
drawn from a Student-$t$ distribution. By estimating the degrees of freedom
parameter, \(\nu\), we can estimate the degree to which the process deviations
have heavy tails and are therefore diagnostic of black-swan events (Fig.~1).
Lower values of \(\nu\) result in heavy-tailed distributions.
For example, an event four standard deviations below the mean
is expected to happen once every 35 years with a $t$ distribution at \(\nu = 2\)
compared to once every 32,000 years with a normal distribution.
As \(\nu\) approaches infinity, the t
distribution approaches the normal distribution (Fig.~1).

\section*{Results}

We found that black-swan dynamics are infrequent but highly
influential: in our dataset they were detected most
frequently for birds (7\%), mammals (5\%), and
insects (3\%) but almost never in fishes (Fig.~2). Black-swan
events were taxonomically widespread, occurring in 38\% of taxonomic orders.
Accounting for time series length and partially pooling inference across
taxonomic class and order with a hierarchical model, we found stronger evidence
for black swans in insect populations than these statistics suggest---four of
eight orders with the highest median probability of heavy tails were insect
orders (Fig.~3a).

\begin{figure}[htb]
\centering
\includegraphics[width=0.3\textwidth]{../analysis/nu-coefs-2.pdf}
\caption{Estimates of population dynamic heavy-tailedness for
populations of birds, mammals, insects, and fishes. (A to D)
Small values of \(\nu\) suggest heavy-tailed black-swan dynamics. Vertical
points and line segments represent posterior medians and 50\% / 90\% credible
intervals for individual populations. Inset plots show probability that \(\nu
< 10\) for populations arranged by taxonomic order and sorted by decreasing
mean Pr(\(\nu < 10\)). Taxonomic orders with three or fewer populations in
panel a are omitted for space. Red to yellow points highlight populations
with a high to moderately high probability of heavy-tailed dynamics.
}
\label{fig:2}
\end{figure}


The majority of our heavy-tailed estimates were robust to alternative
population models, observation error, and choice of Bayesian priors. Our
conclusions were not systematically altered when we included autocorrelation
in the residuals, modelled population growth rates with or without
density dependence, or modelled the population dynamics as Ricker-logistic
instead of Gompertz (Fig.~S\figalt). Similarly, introducing moderate
observation error (CV = 0.2) only slightly decreased the estimated prevalence
of black-swan events (Fig.~S\figalt), and the strength of the prior on
\(\nu\) had little influence on estimates of black-swan dynamics
(Fig.~S\figaltpriors). Finally, our simulation testing shows that, if
anything, our models underpredict the true magnitude and probability of
heavy-tailed events---especially given that the time series are relatively short,
with a median of only 26
years in our analysis (Figs~S\figsimnu,~S\figsimprob).

% Our results show that weaker and stronger priors have little influence on our
% conclusions about heavy-tailed dynamics (Fig.~\figaltpriors). When the
% data are informative about tail behaviour (i.e. when there is strong evidence
% of low \(\nu\) values, upper-right of Fig.~\figaltpriors), the prior has
% little impact on the estimate of \(\nu\). When the data are less informative
% about \(\nu\) (i.e.\ when there are no or few tail events and time series are
% short or noisy), the prior can pull the estimate of \(\nu\) towards larger or
% smaller values (Fig.~\figaltpriors). The vast majority of the
% populations with Pr\((\nu < 10)\) in the base prior were not altered
% qualitatively by this range of prior strength.

For model fits to the population data, the probability of detecting black-swan
dynamics was positively related to time-series length and negatively related to
magnitude of process noise but not clearly related to population growth rate,
density dependence, or maximum lifespan (Fig.~3b, Fig.~S\figcorrelates).
Longer time-series length was the strongest covariate of observing black-swan
dynamics: there is a 1.6 times greater probability of detecting a black-swan
event in populations with 60 time steps than in one with 30 time steps
(Fig.~S\figcorrelates).

\begin{figure}[htb]
\centering
\includegraphics[width=0.26\textwidth]{../analysis/order-posteriors-covariates.pdf}
\caption{Standardized probabilities and covariates of heavy-tailed
dynamics. (A) Taxonomic-order-level posterior densities of Pr(\(\nu
< 10\)) after accounting for time-series length in a hierarchical model. Dotted
vertical line in a indicates the median expected Pr(\(\nu < 10\)) from the
prior distribution. Colour shading refers to taxonomic class (yellow: fishes,
green: insects, purple: birds, and red: mammals). Estimates are standardized at
the geometric mean of time series length across all the data (approximately 27
time steps) and shown for orders with \(\ge 5\) populations. (B)
Posterior densities for potential covariates of Pr(\(\nu < 10\)). In both
panels, short vertical line segments within the density polygons indicate
medians.
}
\label{fig:3}
\end{figure}


The majority of black-swan events (\percBSDown \%) were downward (die-offs)
rather than upward (unexpectedly rapid abundance increases). Of the black-swan
events with published explanations (Table~S2), the majority involved
a combination of multiple factors. For example, a synchronization of
environmental- and predation-mediated population cycles is thought to have
caused a downward black-swan event for a water vole (\emph{Arvicola
terrestris}) population \cite{saucy1994}. Other black swans were the result of
a sequence of extreme climate events on their own. For instance, severe winters
in 1929, 1940--1942, and 1962--1963 were associated with black-swan downswings
in grey herons (\emph{Ardea cinerea}) in the United
Kingdom \cite{stafford1971} (Fig.~4c). Our analysis finds that the last event
was a combination of two heavy-tailed events in a row and the population took
three times longer to recover than predicted \cite{stafford1971}. Downward black
swans were sometimes followed by upward black swans. For example, during
a period of population crowding and nest shortages, a population of European
shag cormorants (\emph{Phalacrocorax aristotelis}) on the Farne Islands, United
Kingdom declined suddenly following a red tide event in 1968 \cite{potts1980}.
This freed up quality nest sites for first-time breeders, productivity
increased, and the population experienced a rapid upswing in
abundance \cite{potts1980}.

\begin{figure}[htb]
\centering
\includegraphics[width=0.45\textwidth]{../analysis/skew-fig.pdf}
\caption{Heavy-tailed process noise tends to be downward skewed and
ignoring this can underestimate risk. (A) Illustrations of Student-$t$
distributions with three levels of skewness (\(\gamma\)) and heavy-tails (\(\nu\)).
(B) Posterior density of the skewness parameters aggregated across
populations grouped into heavy-tailed (\(\hat{\nu} < 10\)), slightly
heavy-tailed (\(10 \leq \hat{\nu} \leq 70\)), and normal-tailed populations
(\(\hat{\nu} > 70\)). Approximate mid-values from B are illustrated in
panel A. (C) Example time series of heavy-tailed populations with
different levels of skewness. Red dots highlight likely heavy-tailed events.
Forecasts (grey regions) show median (solid lines) and lower 99\% credible
intervals (dotted lines) of abundance. Black and red lines indicate forecasts
from Gompertz models with lognormal and skew-$t$ process noise, respectively.
}
\label{fig:4}
\end{figure}

Given the prevalence of downwards events, we refit our heavy-tailed models to
measure the degree of skewness \(\gamma\) of the process noise using a skew-$t$
distribution (Fig.~4a), and used these models to make near-term risk forecasts.
Aggregated across populations with strong evidence of heavy tails (median \(\nu
< 10\)), 85\% of the \(\gamma\) probability density was below 1, indicating
strong evidence of downwardly skewed process noise (Fig.~4b,
Fig.~S\figskewnu). In contrast, populations that did not have heavy
tails (median \(\nu \geq 70\)) had little evidence of skewed process noise
(Fig.~4b; 90\% of 95\% credible intervals overlapped 1). Projecting these
heavy-tailed populations forward five years revealed that assuming standard
normal process noise underestimated risk (99\% lower credible interval of
abundance) by \crashUnderRange-fold (interquartile range; Fig.~4c,
Fig.~S\figskewprojections).

\section*{Discussion}

We systematically evaluated the presence of black-swan events in hundreds of
animal populations.
We find strong evidence for their occurrence in animal populations. Black-swan
population crashes are substantially more frequent
than black-swan population
increases and are usually driven by external events such as weather and
disease.
Thus our analysis provides strong evidence
for downward-skewed heavy-tailed events in abundance time series of higher
taxa, and ignoring these events will tend to underestimate the risk of
population declines.
Next we consider
(1) the possible mechanisms underlying black-swan events,
(2) caveats to our findings,
(3) the consequences of our findings for making risk forecasts, and
(4) for managing natural resources.

% than black-swan population
% increases and are usually driven by external events such as weather and
% disease. Therefore, ignoring the presence of black-swan events, and the
% predominance of crashes over increases, may underestimate extinction risk.
% TODO: add numbers, add skew add projections add length and Sigma

There are many possible causes of black-swan events, including
unmodelled intrinsic properties of populations or extrinsic forces acting on
populations. For example, we could observe black-swan dynamics if we miss an
underlying mixture of processes, since a mixture of normal distributions with
different variances can generate a $t$ distribution with heavy
tails \cite{gelman2014}. In ecology, these processes could arise through an
aggregation of populations across space, population
diversity \cite{schindler2010}, or an intrinsic change in population variability
through time \cite{carpenter2006}. Extrinsic ecological forces could also cause
black-swan dynamics \cite{nunez2012} including extreme climate
events \cite{meehl2004, katz2005, ipcc2012}, predation from (or competition
with) other species experiencing black-swan events, or sharp changes in human
pressures such as fishing or habitat destruction. Alternatively, the
synchrony of multiple extrinsic forces could give rise to black-swan dynamics
through synergistic interactions \cite{kirby2009} or a rare alignment of
non-synergistic forces \cite{denny2009}.

There are a number of caveats when considering the generality of our results.
The GPDD data represent a taxonomically and geographically biased sample of
populations---the longer time series we focused on are dominated by commercially
and recreationally important species and a disproportionate number of
populations are located in the United Kingdom. Although we would expect to find
qualitatively similar evidence for black swans in other large taxonomic or
geographic samples of populations, the common forces driving those black swans
would likely differ. Additionally, some apparent black-swan events could be
recording mistakes, although conversely, some extreme observations may have been
discarded or altered if they were erroneously suspected of being recording
mistakes. Indeed, we discarded three of the populations that our method
initially identified as heavy tailed because they turned out to be data-entry
errors. Finally, the temporal scales of observation and population dynamics
vary considerably across populations in the GPDD and these likely influence the
detection of heavy tails. For example, if we make frequent observations
relative to generation time (e.g.~for many large-bodied mammals) we will
average across generations and perhaps miss black-swan events. Conversely, if
we census populations infrequently relative to generation time (e.g.~many
insects in the GPDD) the recorded data may average across extreme and
less-extreme events and also dampen black-swan dynamics.

% The models we fit may have misspecified
% the dynamics of the populations,
% but this is precisely the point of our paper.
% Black swans represent extreme deviations from our expectations
% and our expectations are represented by the models we fit.
% However, rarely, if ever, are our models perfect representations
% of the underlying truth.
% We might misunderstand the breeding biology of a species,
% the weather can suddenly change,
% disease can suddenly spread,
% and species can unexpectedly interact.

% todo the strongest correlates of finding blocks one events was the timeseries length
% ...implications

Given our results, we suggest modelers should routinely consider representing population dynamic
process noise with a heavy-tailed, and possibly downward-skewed, distribution,
especially when making forecasts to evaluate risk.
But what heavy-tailed distribution should one pick
and how heavy-tailed should it be?
% nearly any reason for why a population modeler has incorrectly specified the exact and full ess %%% generating the abundance times could generate a heavy tailed process noise.
This is an open research question.
The exact shape of the tails has proven important
in the field of dispersal biology \cite{kot1996, clark1999}.
Answers for the field of population dynamics
might come from simulation analyses exploring
the implications of tail shapes and
from comparing the relative fit of heavy-tailed
population models to time series
with suspected mass mortality events
(e.g.~the $t$ distribution compared to a mixture
of two normal distributions \cite{ward2007}).
Nevertheless, our proposed $t$ (or the skew-$t$) distribution
is a simple and elegant solution that
can be easily substituted for the normal distribution.
Furthermore, if the degrees of freedom parameter ($\nu$)
is fixed,
it requires estimating no more parameters
than the normal distribution.
Barring a better estimate,
we suggest using $\nu = 5$; and
a skewness parameter ($\gamma$) of $\sim 0.7$ (Fig.~4a)
if a modeler wishes to allow for downward heavy-tailed events.
These are midrange values from the populations identified as heavy tailed here.
Above $\nu  =  10$ the $t$ distribution closely resembles
the normal distribution,
and $\nu  = 5$ allows for an event
three standard deviations below the mean to happen
% 1/(1-pt(3, 5))
once every 66 years instead of once every 740 years.
Adopting heavy-tailed process noise by default follows the intuition
that our models are usually reasonable but occasionally very wrong.

In light of our findings, we suggest that natural resource management can
learn from disciplines that focus on heavy tails. For example, earthquake
preparedness and response is focused on black-swan events. Similarly to
ecological black swans, we can rarely predict the specific timing of large
earthquakes. But, earthquake preparedness involves spatial planning based on
forecast probabilities to focus early detection efforts and develop disaster
response plans \cite{nrc2007}. The presence of ecological black swans also suggests that we
develop management policy that is robust to heavy tails and encourages
general resilience \cite{carpenter2012}. For instance, setting target
population abundances far back from critical limits will buffer against black-swan
events \cite{caddy1996}, and maintaining genetic, phenotypic, and behavioural
diversity may allow some components of populations to persist when others are
affected by disease or extreme environmental forces \cite{schindler2010}.
Finally, surprising, or counterintuitive ecological dynamics offer a tremendous
opportunity to learn about ecological systems, evaluate when models break down,
and adjust future management policy \cite{doak2008, lindenmayer2010}.

Rare catastrophes can have a profound influence on population
persistence \cite{mangel1994}. In recent decades, ecology has moved toward
focusing on aspects of variance in addition to mean
responses \cite{thompson2013}. Our results suggest that an added focus on
ecological extremes represents the next frontier, particularly in the face of
increased climate extremes \cite{meehl2004, ipcc2012, thompson2013}. Financial
analysts are concerned with the shape of downward tails in financial returns
because these directly impact estimates of risk---the probability of a
market crash occurring.
Similarly, ecologists should focus more
on estimating and predicting downward tails of population abundance, since
these increase true extinction risk.



\matmethods{


\subsection{Data}

We selected abundance time series from the Global Population Dynamics
Database \cite[GPDD,][]{gpdd2010}, which contains nearly 5,000 time series of
abundance from \(\sim\) 1,000 species and \(\sim\) 100 taxonomic orders. We
filtered the data (Supporting Information) to remove populations from less
reliable data sources, and those without sufficient data for our models, and
then interpolated remaining missing values (sensu \cite{brook2006a}).
Our interpolation affected only \(\sim\)
\interpPointsPerc \% of the final data points (Table~S1) and none of the data
points that were later considered black-swan events. Our final dataset
contained \NPops\ populations across \NOrders\ taxonomic orders and seven
taxonomic classes, with a median of \medianTimeSteps\ time steps (range
\minTimeSteps--\maxTimeSteps) (Table~S1).

% This could represent the effect of unmodeled spatial heterogeneity,
% nonstationary environmental processes, or
% varying responses to temperature among individuals, for example.


\subsection{Modeling framework}

We fit heavy-tailed Gompertz population dynamics models to data from the GPDD.
The Gompertz model represents population growth as a linear function in
log space (here and throughout we use log
to refer to the natural logarithm).
If \(x_{t}\) represents the log abundance (\(N\)) at time \emph{t} then
\begin{align}
x_t &= \lambda + bx_{t - 1} + \epsilon_t\\
  \epsilon_t &\sim \mathrm{Student}\mhyphen t(\nu, 0, \sigma).
\end{align}
The parameter \(\lambda\) represents the expected growth rate if \(N_{t}
= 1\). The model is density independent if \(b = 1\), maximally density
dependent if \(b = 0\), and inversely density dependent if \(b < 0\). Usually,
the process noise \(\epsilon_{t}\) is modelled as normally distributed, but we
allow it to be drawn from a Student-$t$ distribution with scale parameter
\(\sigma\) and degrees of freedom \(\nu\). We can then estimate the degree to
which the process deviations have heavy tails and are therefore evidence of
black-swan events (Fig.~1A, B).

We chose the $t$ distribution for a number of reasons.
Pragmatically, the $t$ distribution requires the estimation of only one additional
parameter over the normal distribution and reverts to the normal distribution
if strong evidence of heavy-tailed process deviations is absent.
Furthermore, the $t$ distribution arises from a mixture of normal distributions
with the same mean and different variances \cite{gelman2014},
which mimics the underlying basis for black-swan events.
There are a wide range of mechanistic processes
that could give rise to extreme events in population dynamics,
such as extreme climate events
\cite{meehl2004, katz2005, ipcc2012}
or sudden changes in predation from other species.
While evaluating these various hypotheses
would be important for diagnosing the underlying causes
of extreme events and to develop predictive models,
it would be impossible to tease out each of these
underlying causes from population trend data alone.
Instead we seek a simple flexible phenomenological model
that can be applied across the widest possible
range of population time series to provide a
comparative overview of the prevalence
of black-swan events.

One alternative approach would involve fitting
the generalized extreme value (GEV) distribution,
which represents the limit distribution of a series of maxima or minima.
Although the GEV is well-suited for physical ecological variables
(e.g.\ sedimentation rates or wind speeds \cite{gaines1993, denny2009}),
the GEV requires maxima per time block
or peaks beyond a chosen threshold.
Therefore, the GEV requires longer time series then typically available
for wild animal population abundance.
Furthermore, the GEV approach cannot be easily integrated into population dynamics models.

For the Gompertz model,
we chose weakly informative priors for 
all parameters (Fig.~S\figpriors, Supporting Information).
For the degrees of freedom parameter $\nu$, 
we chose an exponential prior 
with a rate parameter
of \(0.01\) truncated at values above two---a slightly less informative prior
than suggested by Ref.~\cite{fernandez1998}. We truncated the
distribution at \(2\), since at \(\nu < 2\) the variance of the $t$ distribution
is undefined. This prior gives only a \(7.7\)\% probability that \(\nu < 10\)
but constrains the sampling sufficiently to avoid \(\nu\) drifting towards infinity. In
any case, for \(\nu > 20\) the $t$ distribution is almost indistinguishable from
the normal distribution (Fig.~1). Based on the shape of the $t$ distribution, we
chose the probability that \(\nu < 10\), Pr(\(\nu < 10\)), to define the
probability of heavy-tailed (black-swan) dynamics. When categorizing a population as heavy
tailed, we applied a 0.5 probability threshold. In the scenario where the data
are uninformative about heavy tails (e.g.~Fig.~1E, H), the posterior for
\(\nu\) will approximately match the prior (prior median \(= 71\), mean \(=
102\)) and a metric of Pr\((\nu < 10) > 0.5\) will not flag the
population as heavy tailed.

We fit our models with Stan 2.6.0 \cite{stan-manual2015, hoffman2014}, and R
3.2.0 \cite{r2015}. We began with four chains and 2,000 iterations, discarding
the first 1,000 as warm up (i.e.~4,000 final samples). If \(\hat{R}\) (the
potential scale reduction factor---a measure of chain convergence) was greater
than \(1.05\) for any parameter, or if the minimum effective sample size,
\(n_\mathrm{eff}\) (a measure of the effective number of uncorrelated samples),
for any parameter was less than \(200\), we doubled both the total iterations
and warm up period and sampled from the model again. 
% These thresholds are in
% excess of the minimums recommended by Ref.~\cite{gelman2006a} of
% \(\hat{R} < 1.1\) and \(n_\mathrm{eff} > 100\) for reliable point estimates and
% confidence intervals.
In the majority of cases our minimum thresholds were
greatly exceeded. We continued this procedure up to 8,000 iterations (16,000
total samples) by which all chains had sufficiently converged.
The No-U-Turn Hamiltonian Markov chain Monte Carlo sampler in Stan generally
requires fewer iterations to obtain equivalent effective sample sizes than the
Gibbs or Metropolis--Hastings algorithms \cite{stan-manual2015}.

\subsection{Alternative models and simulation testing}
We fit alternative population models to test if four key phenomena
systematically changed our conclusions. Autocorrelation has been suggested as
a reason for increased observed variability of abundance time series through
time, which could create apparent heavy tails \citep{inchausti2002};
therefore, we fit a model that included serial correlation in the residuals.
Additionally, previous work has modelled abundance or growth rates without
accounting for density dependence \citep{halley2002,segura2013}; therefore, we
fit a simpler model in which we assumed density independence. Third,
observation error could bias parameter estimates \citep{knape2012} or mask our
ability to detect heavy tails \citep{ward2007}; therefore, we fit a model
where we allowed for a fixed quantity of observation error ($0.2$ standard
deviations on a log scale). Finally, the Gompertz model assumes that
population growth rate declines linearly with log abundance. Therefore, we
also fit an alternative model, the Ricker-logistic model, which assumes that
population growth rate declines linearly with abundance (Supporting Information).

We investigated the sensitivity
of our results to weaker and stronger priors (exponential rate parameter $=
0.005, 0.02$; Fig.~S\figpriors, Supporting Information). Furthermore, we used
simulated data to test how easily we could detect $\nu$ given different sample
sizes and to ensure we could recover unbiased parameter estimates from the
Gompertz model (Supporting Information).

\subsection{Covariates of heavy-tailed dynamics} We fit a hierarchical beta
regression model to the predicted probability of heavy tails, Pr(\(\nu < 10\)),
to investigate potential covariates of heavy-tailed dynamics. We obtained
maximum lifespan and body-size data from Ref.~\cite{brook2006a}.
Beta regression is useful when response data are on a continuous scale
between zero and one \cite{ferrari2004}. The model was as follows:
\begin{align*}
\mathrm{Pr}(\nu_i < 10) &\sim \mathrm{Beta}(A_i, B_i)\\
\mu_i &= \mathrm{logit}^{-1}(\alpha
  + \alpha^\mathrm{class}_{j[i]}
  + \alpha^\mathrm{order}_{k[i]}
  + \alpha^\mathrm{species}_{l[i]}
  + \bm{X_i} \bm{\beta}),
  \: \\
  &\quad \, \text{for } i = 1, \dots, 609\\
A_i &= \phi_\mathrm{disp} \mu_i\\
B_i &= \phi_\mathrm{disp} (1 - \mu_i)\\
\alpha^\mathrm{class}_j &\sim
  \mathrm{Normal}(0, \sigma^2_{\alpha \; \mathrm{class}}),
  \: \text{for } j = 1, \dots, 6\\
\alpha^\mathrm{order}_k &\sim
  \mathrm{Normal}(0, \sigma^2_{\alpha \; \mathrm{order}}),
  \: \text{for } k = 1, \dots, 38\\
\alpha^\mathrm{species}_l &\sim
  \mathrm{Normal}(0, \sigma^2_{\alpha \; \mathrm{species}}),
  \: \text{for } l = 1, \dots, 301,
\end{align*}
where \(A\) and \(B\) represent the beta distribution shape parameters;
\(\mu_i\) represents the predicted value for population \(i\), class \(j\),
order \(k\), and species \(l\); \(\phi_\mathrm{disp}\) represents the
dispersion parameter; and \(\bm{X_i}\) represents a vector of predictors (such as
lifespan) for population \(i\) with associated \(\bm{\beta}\) coefficients. The
intercepts are allowed to vary from the overall intercept \(\alpha\) by
taxonomic class (\(\alpha^\mathrm{class}_j\)), taxonomic order
(\(\alpha^\mathrm{order}_k\)), and species (\(\alpha^\mathrm{species}_l\)) with
standard deviations \(\sigma_{\alpha \; \mathrm{class}}\), \(\sigma_{\alpha \;
\mathrm{order}}\), and \(\sigma_{\alpha \; \mathrm{species}}\). Where possible,
we also allowed for error distributions around the predictors by incorporating
the standard deviation of the posterior samples for the Gompertz parameters
\(\lambda\), \(b\), and \(\log \sigma\) around the mean point value as normal
distributions (not shown in the above equation).

We log transformed \(\sigma\), time-series length, and lifespan to 
to make the
relationship approximately linear on the logit-transformed response scale;
this matches the way the predictors are visually presented in 
Fig.~S\figcorrelates. 
All input variables were standardized by subtracting their mean and dividing by two
standard deviations to make their coefficients comparable in
magnitude \cite{gelman2008c}. We excluded body length as a covariate because it
was highly correlated with lifespan, and lifespan exhibited more overlap across
taxonomy than body length. Lifespan is also more directly related to time and
potential mechanisms driving black-swan dynamics.

We incorporated weakly informative priors into our model: \(\mathrm{Cauchy}(0,
10)\) on the global intercept \(\alpha\), \(\mathrm{half\mhyphen Cauchy}(0,
2.5)\) on all standard deviation parameters, \(\mathrm{half\mhyphen Cauchy}(0,
10)\) on the dispersion parameter \(\phi_\mathrm{disp}\), and
\(\mathrm{Cauchy}(0, 2.5)\) on all other parameters \cite{gelman2006c,
gelman2008d}. 
% Compared to normal priors, the Cauchy priors concentrate more
% probability density around expected parameter values while allowing for a
% higher probability density far into the tails, thereby allowing the data to
% dominate the posterior more strongly if it disagrees with the prior. 
We fit our
models with 5,000 total iterations per chain, 2500 warm-up iterations, four
chains, and discarding every second sample. We checked for chain convergence
visually and with the same criteria as before (\(\widehat{R} < 1.05\) and
\(n_\mathrm{eff} >200\) for all parameters). Furthermore, we checked our model
against a similar model with the glmmADMB \textsf{R} package \cite{glmmadmb}
without uncertainty around the predictors.

To derive taxonomic-order-level estimates of the probability of heavy tails
accounting for time-series length (Fig.~3A), we fit a separate hierarchical
model with the same structure but with only \(\log\) time-series length as a
predictor---in this case, we did not want to control for intrinsic population
characteristics such as density dependence. Since our predictors were centered
by subtracting their mean value, we obtained order-level estimates of the
probability of heavy tails at mean log time-series length by adding the
posteriors for \(\alpha\), \(\alpha^\mathrm{class}_j\), and
\(\alpha^\mathrm{order}_k\).

\subsection{Skewed Student-$t$ forecasts} To evaluate the apparent skewness
of heavy-tailed process noise we fit Gompertz models with skewed Student-$t$
distributed process noise \cite{fernandez1998}. This distribution adds one
parameter to the Student-$t$ distribution, \(\gamma\), which controls the
skewness. The distribution is symmetrical if \(\gamma = 1\), left skewed if \(0
< \gamma < 1\), and right skewed if \(1 < \gamma < \infty\). We placed a weakly
informative prior of \(\mathrm{Cauchy}(0, 2.5)\) on \(\log \gamma\). We
aggregated 5,000 randomly selected posterior samples from the \(\gamma\)
parameter of each model at three levels of evidence for heavy-tails:
\emph{heavy tailed}, median \(\nu < 10\); \emph{slightly heavy tailed}, \(10 \le
\mathrm{median}\, \nu \le 70\); and \emph{normal tailed}, median \(\nu > 70\)
(Fig.~4).

To generate five-year forecasts of abundance, we combined the posterior
parameter samples from the skewed-t Gompertz models with stochastically
generated process noise. We compared these forecasts to those generated by a
standard Gompertz model with normally distributed process noise fit to the same
data. We calculated the ratio of the lower 99\% quantile credible interval
between the two model projections for all populations in which Pr(\(\nu < 10\))
\(> 0.5\). To ensure the results had stabilized in the tails of the forecast
posterior, we increased the number of posterior samples in Stan. We ran four
chains of 20,000 iterations and discarded the first 10,000 as warmup for a
total of 40,000 samples.



}

\showmatmethods % Display the Materials and Methods section

\acknow{We thank J.W. Moore, A.O. Mooers, L.R. Gerber, J.D. Yeakel, C. Minto and
members of the Earth to Ocean Group for helpful discussions and comments. We
are grateful to the contributors and maintainers of the Global Population
Dynamics Database and to Compute Canada's WestGrid high-performance computing
resources. Silhouette images were obtained from \url{phylopic.org} under Creative
Commons licenses; sources are listed in the Supporting Information. Funding was
provided by a Simon Fraser University Graduate Fellowship and David H. Smith
Conservation Research Fellowship (S.C.A.), the Natural Sciences and Engineering
Research Council of Canada (N.K.D., A.B.C.), the Canada Research Chairs Program
(N.K.D.).
}

\showacknow % Display the acknowledgments section

% \pnasbreak splits and balances the columns before the references.
% If you see unexpected formatting errors, try commenting out this line
% as it can run into problems with floats and footnotes on the final page.
\pnasbreak

% Bibliography
\bibliography{jshort,ms}

\end{document}
