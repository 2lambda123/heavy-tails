\textbf{Title: Black-swan events in animal populations}

\textbf{Authors}: Sean C. Anderson\textsuperscript{1,2*}, Trevor A.
Branch\textsuperscript{3}, Andrew B. Cooper\textsuperscript{2}, Nicholas K.
Dulvy\textsuperscript{1}

\textsuperscript{1}Earth to Ocean Research Group, Department of Biological
Sciences, Simon Fraser University, Burnaby BC, V5A 1S6, Canada

\textsuperscript{2}School of Resource and Environmental Management, Simon
Fraser University, Burnaby, BC, V5A 1S6, Canada

\textsuperscript{3}School of Aquatic and Fishery Sciences, Box 355020,
University of Washington, Seattle, WA 98195, USA

\textsuperscript{*}Corresponding author: Sean C. Anderson; E-mail:
sandrsn@uw.edu; Present address: School of Aquatic and Fishery Sciences, Box
355020, University of Washington, Seattle, WA 98195, USA

%125

\textbf{Abstract}: Black swans are improbable events that nonetheless
occur---often with profound consequences. While physical extremes, such as
monsoons and heat waves, are widely studied and increasing in magnitude and
frequency, it remains unclear the extent to which ecological population numbers
buffer or suffer from such extremes. Here we estimate the frequency and
direction of black-swan events in \NPops\ animal populations, finding strong
evidence for their occurrence in \(\sim\)\overallBasePerc \% of populations.
Black-swan events occur most frequently for birds (\birdPH \%), mammals
(\mammalsPH \%), and insects (\insectsPH \%), and manifest primarily as
population die-offs and crashes (\percBSDown \%) driven by extrinsic
perturbations from climate variability. Thus, ignoring black-swan events will
underestimate extinction risk. Our results demonstrate the importance of both
modeling heavy-tailed downward events in populations, and developing
conservation strategies that are robust to future increases in climate
extremes.

% \textbf{One Sentence Summary}: Black swans---statistically improbable events
% with profound consequences---occur in animal populations and understanding
% them will improve climate risk assessments.

%\clearpage

% \textbf{Main text}: Major surprises happen more often than expected in
financial, social, and natural systems\cite{taleb2007, sornette2009, may2008}.
Massive unpredictable market swings are responsible for the majority of
financial gains and losses\cite{taleb2007}, fatalities from the largest wars
dwarf those from all others \cite{newman2005}, and the frequency of the most
damaging earthquakes has exceeded past expectations\cite{sornette2009}. In
ecological systems, background rates of global extinction are punctuated by
mass extinction\cite{harnik2012}, viruses can mutate suddenly to infect new
hosts, and billions of animals can die at once in mass mortality
events\cite{fey2015}. Indeed, such die-offs may be the most important element
affecting population persistence\cite{mangel1994} and their importance is
likely to increase given projected increases in the frequency and magnitude of
climate-related extremes\cite{ipcc2012}. Despite this anecdotal evidence for
ecological black swans, systematic searches of large numbers of timeseries have
yet to reveal black-swan events\cite{keitt1998, allen2001, halley2002} However,
the overwhelming majority of model fitting and risk forecasting assumes that
deviations from model predictions can be represented by a normal
distribution\cite{brook2006a, knape2012}. If black swans occur, though,
a normal distribution would under-estimate the probability of extreme events
occurring.

Here, we develop a new approach to estimate the frequency and magnitude of
black-swan dynamics across time series of 609 populations from a wide array of
taxonomic groups---including many  birds, mammals, insects, and fishes
(table~S1) \cite{SOM}. We identify characteristics of time series or
life-history traits associated with the detection of black-swan events and
verify known causes. To accomplish this, we develop a framework for identifying
heavy-tailed (black-swan) process noise in population dynamics, i.e.~whether
the largest stochastic jumps in log abundance from one time step to the next
are more extreme than typically seen with a normal distribution. Our framework
allows for a range of population dynamic models, can incorporate observation
uncertainty and skewness in process noise, and can be readily applied to
abundance time series.

Specifically, we fit population dynamics models in which the process noise is
drawn from a Student-t distribution. By estimating the degrees of freedom
parameter, \(\nu\), we can estimate the degree to which the process deviations
have heavy tails and are therefore diagnostic of black-swan events (Fig.~1,
fig.~S\ref{fig:priors}). Lower values of \(\nu\) result in heavy-tailed
distributions. For example, at \(\nu = 2\), the probability of drawing a value
more than five standard deviations below the mean is almost 70,000 times higher
than the probability of drawing such a value from a normal distribution
(\(0.02\) vs.\ \(2.9\cdot10^{-7}\)). As \(\nu\) approaches infinity, the
distribution approaches the normal distribution (Fig.~1).

We found evidence that black-swan dynamics are infrequent but highly
influential: they occur most frequently for birds (7\%), mammals (5\%), and
insects (3\%) and almost never in fishes for this dataset (Fig.~2). Black-swan
events were taxonomically widespread, occurring in 38\% of taxonomic orders.
Accounting for time series length and partially pooling inference across
taxonomic class and order with a hierarchical model, we found stronger evidence
for black swans in insect populations than these statistics suggest---four of
eight orders with the highest median probability of heavy tails were insect
orders (Fig.~3a).

The majority of our heavy-tailed estimates were robust to alternative
population models, observation error, and choice of Bayesian priors. Our
conclusions were not systematically altered when we included an autocorrelation
structure in the residuals, modelled population growth rates with or without
density dependence, or modelled the population dynamics as Ricker-logistic
instead of Gompertz (fig.~S\ref{fig:alt}). Similarly, introducing moderate
observation error (CV = 0.2) only slightly decreased the estimated prevalence
of black-swan events (fig.~S\ref{fig:alt}), and the strength of the prior on
\(\nu\) had little influence on estimates of black-swan dynamics
(fig.~S\ref{fig:alt-priors}). Finally, our simulation testing shows that, if
anything, our models underpredict the true magnitude and probability of
heavy-tailed events---especially given the time series have a median of only 26
years in our analysis (figs~S\ref{fig:sim-nu},~S\ref{fig:sim-prob}).

For model fits to the population data, the probability of detecting black-swan
dynamics was positively related to time-series length and negatively related to
magnitude of process noise but not clearly related to population growth rate,
density dependence, or maximum lifespan (Fig.~3b, fig.~S\ref{fig:correlates}).
Longer time-series length was the strongest covariate of observing black-swan
dynamics: there is a 1.6 times greater probability of detecting a black-swan
event in populations with 60 time steps than in one with 30 time steps
(fig.~S\ref{fig:correlates}).

The majority of black-swan events (\percBSDown \%) were downward (die-offs)
rather than upward (unexpectedly rapid abundance increases). Of the black-swan
events with published explanations (table~S2), the majority involved
a combination of multiple factors. For example, a synchronization of
environmental- and predation-mediated population cycles is thought to have
caused a downward black-swan event for a water vole (\emph{Arvicola
terrestris}) population\cite{saucy1994}. Other black swans were the result of
a sequence of extreme climate events on their own. For instance, severe winters
in 1929, 1940--1942, and 1962--1963 were associated with black-swan downswings
in grey heron (\emph{Ardea cinerea}) abundance in the United
Kingdom\cite{stafford1971} (Fig.~4c). Our analysis finds that the last event
was a combination of two heavy-tailed events in a row and the population took
three times longer to recover than predicted\cite{stafford1971}. Downward black
swans were sometimes followed by upward black swans. For example, during
a period of population crowding and nest shortages, a population of European
shag cormorants (\emph{Phalacrocorax aristotelis}) on the Farne Islands, United
Kingdom declined suddenly following a red tide event in 1968\cite{potts1980}.
This freed up quality nest sites for first-time breeders, productivity
increased, and the population experienced a rapid upswing in
abundance\cite{potts1980}.

Given the prevalence of downwards events, we refit our heavy-tailed models to
measure the degree of skewness \(\gamma\) of the process noise using a skew-t
distribution (Fig.~4a), and used these models to make near-term risk forecasts.
Aggregated across populations with strong evidence of heavy tails (median \(\nu
< 10\)), 85\% of the \(\gamma\) probability density was below 1, indicating
strong evidence of downwardly skewed process noise (Fig.~4b,
fig.~S\ref{fig:skew-nu}). In contrast, populations that did not have heavy
tails (median \(\nu \geq 70\)) had little evidence of skewed process noise
(Fig.~4b; 90\% of 95\% credible intervals overlapped 1). Projecting these
heavy-tailed populations forward five years revealed that assuming standard
normal process noise underestimated risk (99\% lower credible interval of
abundance) by \crashUnderRange-fold (interquartile range; Fig.~4c,
fig.~S\ref{fig:skew-projections}). Thus our analysis provides strong evidence
for downward-skewed heavy-tailed events in abundance time series of higher
taxa, and ignoring these events will tend to underestimate the risk of
population declines.

There are a number of possible causes of black-swan events, including
unmodelled intrinsic properties of populations or extrinsic forces acting on
populations. For example, we could observe black-swan dynamics if we miss an
underlying mixture of processes, since a mixture of normal distributions with
different variances can generate a t distribution with heavy
tails\cite{allen2001}. In ecology, these processes could arise through an
aggregation of populations across space, population
diversity\cite{schindler2010}, or an intrinsic change in population variability
through time\cite{carpenter2006}. Extrinsic ecological forces could also cause
black-swan dynamics\cite{nunez2012} including extreme climate
events\cite{meehl2004, katz2005, ipcc2012}, predation from (or competition
with) other species experiencing black swans, or sharp changes in human
pressures such hunting, fishing, or habitat destruction. Alternatively, the
synchrony of multiple extrinsic forces could give rise to black-swan dynamics
through synergistic interactions\cite{kirby2009} or a rare alignment of
non-synergistic forces\cite{denny2009}.

In light of our findings, we suggest that ecological resource management can
learn from disciplines that focus on heavy tails. For example, earthquake
preparedness and response is focused on black-swan events. Similarly to
ecological black swans, we can rarely predict the specific timing of large
earthquakes. But, earthquake preparedness involves spatial planning based on
forecast probabilities to focus early detection efforts and develop disaster
response plans\cite{nrc2007}. The presence of black swans also suggests that we
develop management policy that is robust to heavy tails and encourages
\textit{general} resilience\cite{carpenter2012}. For instance, setting target
population abundances far from critical limits will buffer against black-swan
events\cite{caddy1996}, and maintaining genetic, phenotypic, and behavioural
diversity may allow some components of populations to persist when others are
affected by disease or extreme environmental forces\cite{schindler2010}.
Finally, surprising, or counterintuitive ecological dynamics offer a tremendous
opportunity to learn about ecological systems, evaluate when models break down,
and adjust future management policy\cite{doak2008, lindenmayer2010}.

Rare catastrophes can have a profound influence on population
persistence\cite{mangel1994}. In recent decades, ecology has moved toward
focusing on aspects of variance in addition to mean
responses\cite{thompson2013}. Our results suggest that an added focus on
ecological extremes represents the next frontier, particularly in the face of
increased climate extremes\cite{meehl2004, ipcc2012, thompson2013}. Financial
analysts are concerned with the shape of financial return downward tails
because these directly impact estimates of risk---the probability of a specific
magnitude of undesired event occurring. Similarly, ecologists should focus more
on estimating and predicting downward tails of population abundance, since
these increase true extinction risk.

\textbf{References}

\renewcommand{\section}[2]{}%

\bibliographystyle{Science}\bibliography{jshort,ms}

\textbf{Acknowledgements}: We thank J.W. Moore, A.O. Mooers, L.R. Gerber, J.D.
Yeakel, C. Minto and members of the Earth to Ocean Group for helpful
discussions and comments. We are grateful to the contributors and maintainers
of the Global Population Dynamics Database and to Compute Canada's WestGrid
high-performance computing resources. Silhouette images were obtained from
phylopic.org under Creative Commons licenses; sources are listed in the
Supplementary Materials. Funding was provided by a Simon Fraser University
Graduate Fellowship (S.C.A.), the Natural Sciences and Engineering Research
Council of Canada (N.K.D., A.B.C.), the Canada Research Chairs Program
(N.K.D.).

\textbf{Supplementary Materials}\\
Materials and Methods\\
Supplementary References\\
Tables S1 to S2\\
Figs.~S1 to S8\\

%\textbf{Author Contributions}: S.C.A. and T.A.B. conceived the project;
%S.C.A., T.A.B., A.B.C., and N.K.D. designed the study; S.C.A. analyzed the
%data and wrote the paper with input from all authors.

\clearpage

\textbf{Figures}

\begin{center}
\includegraphics[width=\textwidth]{../analysis/t-nu-eg2.pdf}
\end{center}

\textbf{Fig.~1: Illustration of population dynamic models that allow for heavy
tails.} (\textbf{A} to \textbf{B}) Probability density for the Student-t
distribution with scale parameter of 1 and different values of \(\nu\). Small
values of \(\nu\) create heavy tails while as \(\nu\) approaches infinity the
distribution approaches the normal distribution. (\textbf{C} to \textbf{E})
Simulated population dynamics from a Gompertz model with process noise drawn
from Student-t distributions with three values of \(\nu\). Coloured dots in
panels \textbf{C} and \textbf{D} represent jumps with less than a 1 in 1000
chance of occurring in a normal distribution. (\textbf{F} to \textbf{H})
Estimates of \(\nu\) from models fit to the times series in \textbf{C} to
\textbf{E}. Shown are posterior samples (histograms), median and interquartile
range of the posterior (IQR, dots and line segments), and the exponential prior
on \(\nu\) (dashed lines). Colour shading behind \textbf{F} to \textbf{H}
illustrates the approximate region of heavy tails.

\clearpage

\begin{center}
\includegraphics[width=0.36\textwidth]{../analysis/nu-coefs-2.pdf}
\end{center}

\textbf{Fig.~2: Estimates of population dynamic heavy-tailedness for
populations of birds, mammals, insects, and fishes.} (\textbf{A} to \textbf{D})
Small values of \(\nu\) suggest heavy-tailed black-swan dynamics. Vertical
points and line segments represent posterior medians and 50\% / 90\% credible
intervals for individual populations. Inset plots show probability that \(\nu
< 10\) for populations arranged by taxonomic order and sorted by decreasing
mean Pr(\(\nu < 10\)). Taxonomic orders with three or fewer populations in
\textbf{A} are omitted for space. Red to yellow points highlight populations
with a high to moderately high probability of heavy-tailed dynamics.

\clearpage

\begin{center}
\includegraphics[width=0.35\textwidth]{../analysis/order-posteriors-covariates.pdf}
\end{center}

\textbf{Fig.~3: Standardized probabilities and covariates of heavy-tailed
dynamics.} (\textbf{A}) Taxonomic-order-level posterior densities of Pr(\(\nu
< 10\)) after accounting for time-series length in a hierarchical model. Dotted
vertical line in a indicates the median expected Pr(\(\nu < 10\)) from the
prior distribution. Colour shading refers to taxonomic class (yellow: fishes,
green: insects, purple: birds, and red: mammals). Estimates are standardized at
the geometric mean of time series length across all the data (approximately 27
time steps) and shown for orders with \(\ge 5\) populations. (\textbf{B})
Posterior densities for potential covariates of Pr(\(\nu < 10\)). In both
panels, short vertical line segments within the density polygons indicate
medians.

\clearpage

\begin{center}
\includegraphics[width=0.65\textwidth]{../analysis/skew-fig.pdf}
\end{center}

\textbf{Fig.~4: Heavy-tailed process noise tends to be downward skewed and
ignoring this can underestimate risk.} (\textbf{A}) Illustrations of Student-t
distributions with 3 levels of skewness (\(\gamma\)) and heavy-tails (\(\nu\)).
(\textbf{B}) Posterior density of the skewness parameters aggregated across
populations grouped into heavy-tailed (\(\hat{\nu} < 10\)), slightly
heavy-tailed (\(10 \leq \hat{\nu} \leq 70\)), and normal-tailed populations
(\(\hat{\nu} > 70\)). Approximate mid-values from \textbf{B} are illustrated in
\textbf{A}. (\textbf{C}) Example time series of heavy-tailed populations with
different levels of skewness. Red dots highlight likely heavy-tailed events.
Forecasts (grey regions) show median (solid lines) and lower 99\% credible
intervals (dotted lines) of abundance. Black and red lines indicate forecasts
from Gompertz models with lognormal and skew-t process noise, respectively.
