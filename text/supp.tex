%!TEX root = anderson-etal-blackswan-timeseries.tex

\begin{centering}
\LARGE
Supporting Information\\[1.0em]
\end{centering}

\section{Data selection}

We applied the following data selection and quality-control rules to the Global Population Dynamics Database (GPDD):

\begin{enumerate}

\item To remove populations with unreliable population indices that could be strongly confounded with economics and sampling effort, we removed all populations with a sampling protocol listed as \texttt{harvest} as well populations with the words \texttt{harvest} or \texttt{fur} in the cited reference title.

\item We removed all populations with uneven sampling intervals, i.e.\ we removed populations that didn't have a constant difference between the ``decimal year begin'' and ``decimal year end'' columns.

\item We removed all populations rated as $< 2$ in the GPDD quality assessment (on a scale of $1$ to $5$, with $1$ being the lowest quality data) \citep[following][]{sibly2005, ziebarth2010}.

\item Populations with negative abundance values were removed. Of the populations that remained at the end of our other filtering rules, the remaining populations with negative abundances listed were all from time series that had been standardized by subtracting the mean and dividing by the standard deviation. We verified this by locating the original papers the datasets were extracted from: \citet{colebrook1978} for zooplankton and \citet{lindstrom1995} for grouse. Since the papers did not include the original mean time-series values we could not back transform these data points.

\item We filled in all missing time steps with \texttt{NA} values and imputed single missing values with the geometric mean of the previous and following values. We chose a geometric mean to be linear on the log scale that the Gompertz and Ricker-logistic models were fit on.

\item We filled in single recorded values of zero with the lowest non-zero value in the time series \citep[following][]{brook2006a}. This assumes that single values of zero result from abundance being low enough that censusing overlooked individuals that were actually present. We turned multiple zero values in a row into \texttt{NA} values. This implies that multiple zero values were either censusing errors or caused by emigration. Regardless, our population models were fit on a multiplicative (log) scale and so could not account for zero abundance. To avoid distorting the original data too strongly, we removed populations in which we filled in more than four zeros.

\item We removed all populations without at least four unique values \citep[following][]{brook2006a}.

\item We removed all populations with four or more identical values in a row since these suggest either recording error or extrapolation between two observations.

\item We then wrote an algorithm to find the longest unbroken window of abundance (no \texttt{NA}s) with at least $20$ time steps in each population time series. If there were any populations with multiple windows of identical length, we took the most recent window. This is a longer window than used in some previous analyses \citep[e.g.][]{brook2006a}, but since our model attempts to capture the shape of the distribution tails, our model requires more data.

\item We removed GPDD Main IDs \texttt{20531} and \texttt{10139}, which we noticed were duplicates of \texttt{20579} (a heron population). \texttt{20579} contained additional years of data not present in \texttt{10139}. We removed a limited number of populations from class Angiospermopsida and Bacillariophyceae to focus the taxonomy in our analysis. We also removed any populations with an \texttt{Unknown} taxonomic class.

\item Finally, we removed populations with the following GPDD Main IDs, which we discovered data entry errors when verifying the populations with suspected black swans: \texttt{1207} because 1957 was entered as 2 but should have been 27 \citep{kendeigh1982}, \texttt{6531} because 1978 was entered as 7 should have been 47 \citep{minot1986}, and \texttt{6566} because the data did not match the graph \citep{heessen1996}.

\end{enumerate}

\noindent
We provide a supplemental figure of all the time series included in our analysis and indicate which values were interpolated (\percImputedPops\% of populations had at least one point interpolated also only \percImputedPoints\% of the total observations were interpolated) (Fig.~\ref{fig:all-ts}). Note that interpolation is highly unlikely to lead to black-swan detections, since black swans involve extreme increases or decreases. Table~\ref{tab:stats} shows the final taxonomic breakdown and the number of populations with interpolated values.

\section{Details on the heavy-tailed Gompertz probability model}

For the Gompertz model, our weakly-informative priors (Fig.~\ref{fig:priors}) were:
\begin{align*}
b &\sim \mathrm{Uniform}(-1, 2)\\ \lambda &\sim \mathrm{Normal}(0, 10^2)\\
\sigma &\sim \mathrm{Half\mhyphen Cauchy} (0, 2.5)\\ \nu &\sim
\mathrm{Truncated\mhyphen Exponential}(0.01, \mathrm{min.} = 2).
\end{align*}
Our prior on $b$ was uninformative between values of $-1$ and $2$. We would not expect values of $b$ with levels of density dependence as low as $-1$ (very strong inverse density dependence), nor would we generally expect values above $1$. We allowed values of $b$ above $1$ to allow for non-stationary time series of growth rates. The estimates of $b$ were well within these bounds. Our prior on $\lambda$ was very weakly informative within the range of expected values for population growth and is similar to the default priors suggested by \citet{gelman2008d} for intercepts of regression models. Our Half-Cauchy prior on $\sigma$ follows \citet{gelman2006c} and \citet{gelman2008d} and the scale parameter of $2.5$ is based on our expected range of the value in nature from previous studies \citep[e.g.][]{connors2014}. In our testing of a subsample of populations, our parameter estimates were not qualitatively changed by switching to an uninformative uniform prior on $\sigma$, but the weakly informative Half-Cauchy prior substantially sped up chain convergence.

Our prior on $\nu$ was based on \citet{fernandez1998}. They chose an exponential rate parameter of $0.1$. We chose a less informative rate parameter of $0.01$ and truncated the distribution at $2$, since at $\nu < 2$ the variance of the t distribution is undefined. This prior gives only a $7.7$\% probability that $\nu < 10$ but constrains the sampling sufficiently to avoid wandering off towards infinity --- above approximately $\nu = 20$ the t distribution is so similar to the normal distribution (Fig.~\ref{fig:didactic}) that time series of the length considered here are unlikely to be informative about the precise value of $\nu$. In the scenario where the data are uninformative about heavy tails (e.g.~Fig.~\ref{fig:didactic}e,~h), the posterior will approximately match the prior (prior median $= 71$, mean $= 102$) and the metrics used in our paper (e.g.~Pr$(\nu < 10) > 0.5$) are unlikely to flag the population as heavy tailed.

We fit our models with Stan 2.4.0 \citep{stan-manual2014}, and R 3.1.1 \citep{r2014}. We began with four chains and $2000$ iterations, discarding the first $1000$ as warm up (i.e.~4000 total samples). If $\hat{R}$ (the potential scale reduction factor --- a measure of chain convergence) was greater than $1.05$ for any parameter or the minimum effective sample size, $n_\mathrm{eff}$, (a measure of the effective number of uncorrelated samples) for any parameter was less than $200$, we doubled both the total iterations and warm up period and sampled from the model again. These thresholds are in excess of the minimums recommended by \citet{gelman2006a} of $\hat{R} < 1.1$ and effective sample size $> 100$ for reliable point estimates and confidence intervals. In the majority of cases our minimum thresholds were greatly exceeded. We continued this procedure up to $8000$ iterations ($16000$ total samples) by which all chains were deemed to have sufficiently converged. These chain lengths may seem low to those familiar with software such as WinBUGS or JAGS, but the No-U-Turn Hamiltonian Markov chain Monte Carlo Sampler in Stan generally requires far fewer iterations to obtain equivalent effective sample sizes \citep{stan-manual2014}.

\section{Alternative priors}

To test if the prior on $\nu$ influenced our estimate of black-swan dynamics, we refit our models with weaker and stronger priors. Our base model used a prior on $\nu$ of Truncated-Exponential(0.02, min.\ = 2). For a weaker prior we used Truncated-Exponential(0.005, min.\ = 2) and for a stronger prior we used Truncated-Exponential(0.02, min.\ = 2) (Fig.~\ref{fig:priors}). Note that the base and weaker priors are relatively flat within the region of $\nu < 20$---the region we are mostly concerned about when categorizing populations as heavy- or thin-tailed.

Our results show that these weaker and stronger priors would have little influence on our conclusions about heavy-tailed dynamics (Fig.~\ref{fig:alt-priors}). When the data are informative about tail behaviour (i.e.\ when there is strong evidence of low $\nu$ values, upper-right of Fig.~\ref{fig:alt-priors}), the prior has little impact on the estimate of $\nu$. When the data are less informative about $\nu$ (i.e.\ when there are no or few tail events and time series are short or noisy), the prior can pull the estimate of $\nu$ towards larger or smaller values (Fig.~\ref{fig:alt-priors}). The vast majority of the populations with Pr$(\nu < 10)$ in the base prior are not altered qualitatively by this range of prior strength.

\section{Alternative population models}

We fit four alternative population models to the time-series data to check how they would influence our conclusions. Our alternative models allowed for autocorrelation in the residuals, assumed no density dependence, allowed for observation error, or assumed a Ricker-logistic functional form. The range of percentages of black swans by taxonomic class cited in the abstract and results are based on lower and upper limits across our main Gompertz model and these four alternative models.

\subsection{Autocorrelated residuals}

We considered a version of the Gompertz model in which an autoregressive
parameter was fit to the process noise residuals:
\begin{align*}
x_t &= \lambda + b x_{t-1} + \epsilon_t\\
\epsilon_t &\sim \mathrm{Student\mhyphen t}(\nu, \phi \epsilon_{t-1}, \sigma).
\end{align*}
In addition to the parameters in the original Gompertz model, we estimate an additional parameter $\phi$, which represents the dependence of subsequent residuals. Based on the results of previous analyses with the GPDD \citep[e.g.][]{connors2014} and the chosen priors in previous analyses \citep[e.g.][]{thorson2014a} and to greatly speed up chain convergence when running our model across all populations, we placed a weakly informative prior on $\phi$ that assumed the greatest probability density near zero with the reduced possibility of $\phi$ being near $-1$ or $1$. Specifically, we chose $\phi \sim \mathrm{Truncated\mhyphen Normal}(0, 1, \mathrm{min.} = -1, \mathrm{max.} = 1)$. The MCMC chains did not converge for \modelsNoConvergeAROne\ populations according to our criteria ($\widehat{R} < 1.05, n_\mathrm{eff} > 200$) after 8000 iterations of four chains. This included only \modelsNoConvergeAROneHeavyBase\ populations in which Pr($\nu < 10$) $> 0.5$ categorized them as heavy in the main Gompertz model. We did not include these models in Fig.~\ref{fig:alt}.

\subsection{Assumed density independence}\label{assumed-density-independence}

We fit a simplified version of the Gompertz model in which the density dependence parameter $b$ was fixed at $1$ (density independent). This is equivalent to fitting a random walk model (with drift) to the $\log$ abundances or assuming the growth rates are drawn from a stationary distribution. The model was as follows:
\begin{align*}
x_t &= \lambda + x_{t-1} + \epsilon_t\\
\epsilon &\sim \mathrm{Student\mhyphen t}(\nu, 0, \sigma).
\end{align*}
We fit this model for three reasons: (1) it is computationally simpler and so provides a check that our more complicated full Gompertz model was obtaining reasonable estimates of $\nu$, (2) it provides a test of whether density dependence was systematically affecting our perception of heavy tails, (3) it matches how some previous authors have modelled heavy tails without accounting for density dependence \citep{segura2013}.

\subsection{Assumed observation error}

Observation error can bias parameter estimates \citep[e.g.][]{knape2012} and is known to affect the ability to detect extreme events \citep{ward2007}. In our main analysis, we fit a model that ignored observation error. One way to account for observation error would be to fit a full state-space model that simultaneously estimates the magnitude of process noise and observation error. However, simultaneously estimating observation and process noise is a challenging problem (e.g.\ because the observation and process noise parameters tend to negatively covary in model fitting) and is known to sometimes result in identifiability issues with the Gompertz population model \citep{knape2008}. Furthermore, our model was applied to hundreds of time series, often of short length (as few as 20 time steps) and our model estimates an additional parameter --- the shape of the process deviation tails --- potentially making identifiability and computational issues even greater. Therefore, we considered a version of the base Gompertz model that allowed for a fixed level of observation error:
\begin{align*}
U_t &= \lambda + b U_{t-1} + \epsilon_t\\
x_t &\sim \mathrm{Normal}(U_t, \sigma_\mathrm{obs}^2)\\
\epsilon_t &\sim \mathrm{Student\mhyphen t}(\nu, 0, \sigma_\mathrm{proc}),
\end{align*}
where $U$ represents the unobserved state vector, $\sigma_\mathrm{obs}$ represents the standard deviation of observation error (on a log scale), and $\sigma_\mathrm{proc}$ represent the process noise scale parameter. We set $\sigma_\mathrm{obs}$ to $0.2$, which represents the upper limit of values often used in simulation analyses \citep[e.g.][]{valpine2002, thorson2014b}.

\subsection{Ricker-logistic}

We also fit a Ricker-logistic model:
\begin{align*}
x_t &= x_{t-1} + r_{\mathrm{max}}\left(1 - \frac{N_{t-1}}{K}\right) + \epsilon_t\\
\epsilon_t &\sim \mathrm{Student\mhyphen t}(\nu, 0, \sigma),
\end{align*}
where $r_\mathrm{max}$ represents the theoretical maximum population growth rate that is obtained when $N_t$ (abundance at time $t$) $= 0$. The parameter $K$ represents the carrying capacity and, as before, $x_t$ represents the $\log$ transformed abundance at time $t$. The Ricker-logistic model assumes a linear decrease in population growth rate with increases in abundance. In contrast, the Gompertz model assumes a linear decrease in population growth rate with increases in \textit{log} abundance ($x_t$) \citep[e.g.][]{thibaut2012}.

To fit the Ricker-logistic models, we chose a prior on $K$ uniform between zero and twice the maximum observed abundance (\citet{clark2010} chose uniform between zero and maximum observed, which is more informative). We set the prior on $r_\mathrm{max}$ as uniform between 0 and 20 as in \citet{clark2010}. We used the same priors on $\nu$ and $\sigma$ as in the Gompertz model.

\section{Simulation testing the model}

We performed two types of simulation testing. First, we tested how easily the Student-t distribution $\nu$ parameter could be recovered given different true values of $\nu$ and different sample sizes. Second, we tested the ability of the heavy-tailed Gompertz model to obtain unbiased parameter estimates of $\nu$ given that a set of process deviations was provided in which the effective $\nu$ value was close to the true $\nu$ value.

We separated our simulation into these two components to avoid confounding two issues. (1) With smaller sample sizes, there may not be a stochastic draw from the tails of a distribution. In that case, no model, no matter how perfect, will be able to detect the shape of the tails. (2) Complex models may return biased parameter estimates if there are conceptual, computational, or coding errors. Our first simulation tested the first issue and our second simulation tested the latter. In general, our simulations show that, if anything, our model under predicts the magnitude and probability of heavy tailed events --- especially given the length of the time series in the GPDD.

\subsection{Estimating $\nu$ from a stationary t distribution}

First, we tested the ability to estimate $\nu$ given different true values of $\nu$ and different sample sizes. We took stochastic draws from t distributions with different $\nu$ values ($\nu = 3, 5, 10,$ and $10^6$ [$\approx$ normal]), with central tendency parameters of $0$, and scale parameters of $1$. We started with $1600$ stochastic draws and then fit the models again at the first $800, 400, 200, 100, 50,$ and $25$ draws. Each time we recorded the posterior samples of $\nu$.

We found that we could consistently and precisely recover median posterior estimates of $\nu$ near the true value of $\nu$ with large samples ($\ge 200$) (Fig.~\ref{fig:sim-nu} upper panels). At smaller samples we could still usually qualitatively distinguish heavy from not-heavy tails, but the model tended to underestimate how heavy the tails were. At the same time, at smaller sample sizes, the model tended to overestimate how large the scale parameter was (Fig.~\ref{fig:sim-nu} lower panels).

\subsection{Heavy-tailed Gompertz model simulations}

In the second part of our simulation testing, we tested the ability of the heavy-tailed Gompertz model to obtain unbiased parameter estimates when the process noise was chosen so that appropriate tail events were present. To generate these process deviations for the $\nu = 3$ and $\nu = 5$ scenarios, we repeatedly drew proposed candidate process deviations and estimated the central tendency, scale, and $\nu$ values each time. We recorded when $\hat{\nu}$ (median of the posterior) was within $0.2$ CVs (coefficient of variations) of the true $\nu$ value and used this set of random seed values in our Gompertz simulation. The following simplified R code illustrates this procedure (the actual code is available at \url{https://github.com/seananderson/heavy-tails}):

\begin{footnotesize}
\begin{verbatim}
get_effective_nu_seeds <- function(nu_true = 5, cv = 0.2, N = 50, seed_N = 20) {
  # nu_true: The true nu value
  # cv:      The permitted effective nu coefficient of variation
  # N:       The length of time series
  # seed_N:  The number of seed values to generate
  seeds <- numeric(length = seed_N)
  seed_value <- 0
  for (i in seq_len(seed_N)) {
    nu_close <- FALSE
    while (!nu_close) {
      seed_value <- seed_value + 1
      set.seed(i)
      y <- rt(N, df = nu_true)
      sm <- rstan::stan(... # fit the Stan model here
      med_nu_hat <- median(rstan::extract(sm, pars = "nu")[[1]])
      if (med_nu_hat > (nu_true - cv) & med_nu_hat < (nu_true + cv)) {
        nu_close <- TRUE
        seeds[i] <- seed_value
      }
    }
  }
  seeds
}
nu_3_seeds_N50 <- get_effective_nu_seeds(nu_true = 3)
nu_5_seeds_N50 <- get_effective_nu_seeds(nu_true = 5)
\end{verbatim}
\end{footnotesize}

We then fit our Gompertz models to the simulated datasets with all parameters (except $\nu$) set near the median values estimated in the GPDD. We repeated this with $50$ and $100$ samples without observation error, $50$ samples with observation error ($\sigma_\mathrm{obs} = 0.2$), and $50$ samples with the same observation error and a Gompertz model that allowed for correctly specified observation error magnitude.

Our results indicate that the Gompertz model can recapture the true value of $\nu$ when the process noise was chosen so that appropriate tail events were present (TODO FIG).
The addition of observation error caused the model to tend to underestimate the degree of heavy-tailedness. Fitting a model with correctly specified observation error did not make substantial improvements to model bias.

%(Figs~\ref{fig:sim-gompertz} and \ref{fig:sim-gompertz-boxplots}, red and green symbols in the top rows). Likewise, the other Gompertz parameters were estimated without any systematic bias (Figs~\ref{fig:sim-gompertz} and \ref{fig:sim-gompertz-boxplots}, red and green symbols). 
%, overestimate the magnitude of process noise, somewhat overestimate $\lambda$, and overestimate density dependence (blue symbols in Figs~\ref{fig:sim-gompertz} and \ref{fig:sim-gompertz-boxplots}). The overestimation of density dependence with observation error is a known phenomenon \citep{knape2012}. Fitting a model with correctly specified observation error made marginal improvements to model bias (purple symbols in Figs~\ref{fig:sim-gompertz} and \ref{fig:sim-gompertz-boxplots}).

When converting the posterior distributions of $\nu$ into Pr($\nu < 10$), the models distinguished heavy and not-heavy tails reasonably well. Without observation error, and using a probability of $0.5$ as a threshold, the model correctly classified all simulated systems with normally distributed process noise as not heavy tailed. The model would have miscategorized only one of $40$ simulations at $\nu = 5$ across a sample size of $50$ or $100$ data points (Fig.~\ref{fig:sim-prob}, upper panels). The model would have correctly categorized all cases where the process noise was not heavy tailed (indicated as ``$\nu =$ infinity'' in Fig.~\ref{fig:sim-prob}) and all cases where $\nu = 3$. With $0.2$ standard deviations of observation error, the model still categorized \obsErrorNuFivePerc\% of cases as heavy tailed when $\nu = 5$ and all cases where $\nu = 3$. Allowing for observation error made little improvement to the detection of heavy tails (Fig.~\ref{fig:sim-prob}, lower-left vs.\ lower-right panel). Therefore, we chose to focus on the simpler model without observation error in the main text, particularly given that the true magnitude of observation error was unknown in the empirical data.

\section{Modelling covariates of heavy-tailed dynamics}

We fit a multilevel beta regression model to the predicted probability of heavy tails, Pr($\nu < 10$), to investigate potential covariates of heavy-tailed dynamics. The beta distribution is useful when response data range on a continuous scale between zero and one. We used a logit link function as is typically used in logistic regression. The model was as follows:
\begin{align*}
\mathrm{Pr}(\nu_i < 0.5) &\sim \mathrm{Beta}(A_i, B_i)\\
\mu_i &= \mathrm{logit}^{-1}(\alpha
  + \alpha^\mathrm{class}_{j[i]}
  + \alpha^\mathrm{order}_{k[i]}
  + \alpha^\mathrm{species}_{l[i]}
  + X_i \beta),
  \: \text{for } i = 1, \dots, 617\\
A_i &= \phi_\mathrm{disp} \mu_i\\
B_i &= \phi_\mathrm{disp} (1 - \mu_i)\\
\alpha^\mathrm{class}_j &\sim
  \mathrm{Normal}(0, \sigma^2_{\alpha \; \mathrm{class}}),
  \: \text{for } j = 1, \dots, 6\\
\alpha^\mathrm{order}_k &\sim
  \mathrm{Normal}(0, \sigma^2_{\alpha \; \mathrm{order}}),
  \: \text{for } k = 1, \dots, 38\\
\alpha^\mathrm{species}_l &\sim
  \mathrm{Normal}(0, \sigma^2_{\alpha \; \mathrm{species}}),
  \: \text{for } l = 1, \dots, 301,
\end{align*}
where $A$ and $B$ represent the beta distribution shape parameters; $\mu_i$ represents the predicted value for population $i$, class $j$, order $k$, and species $l$; $\phi_\mathrm{disp}$ represents the dispersion parameter; and $X_i$ represents a vector of predictors (such as lifespan) for population $i$ with associated $\beta$ coefficients. The intercepts are allowed to vary from the overall intercept $\alpha$ by taxonomic class ($\alpha^\mathrm{class}_j$), taxonomic order ($\alpha^\mathrm{order}_k$), and species ($\alpha^\mathrm{species}_l$) with standard deviations $\sigma_{\alpha \; \mathrm{class}}$, $\sigma_{\alpha \; \mathrm{order}}$, and $\sigma_{\alpha \; \mathrm{species}}$. Where possible, we also allowed for error distributions around the predictors by incorporating the standard deviation of the posterior samples for the Gompertz parameters $\lambda$, $b$, and $\log \sigma$ around the mean point value as normal distributions (not shown in the above equation).

We log transformed $\sigma$, time-series length, and lifespan to match the way they are visually represented in Fig.~\ref{fig:correlates} and to make the relationship approximately linear on the logit-transformed response scale. All input variables were standardized by subtracting their mean and dividing by two standard deviations to make their coefficients comparable in magnitude \citep{gelman2008c}. We excluded body length as a covariate because it was highly correlated with lifespan, and lifespan exhibited more overlap across taxonomy than body length. Lifespan is also more directly related to time and potential mechanisms driving black-swan dynamics.

We incorporated weakly informative priors into our model: $\mathrm{Cauchy}(0, 10)$ on the global intercept $\alpha$, $\mathrm{Half\mhyphen Cauchy}(0, 2.5)$ on all standard deviation parameters, $\mathrm{Half\mhyphen Cauchy}(0, 10)$ on the dispersion parameter $\phi_\mathrm{disp}$, and $\mathrm{Cauchy}(0, 2.5)$ on all other parameters \citep{gelman2006c, gelman2008d}. Compared to normal priors, the Cauchy priors concentrate more probability density around expected parameter values while allowing for a higher probability density far into the tails, thereby allowing the data to dominate the posterior more strongly if it disagrees with the prior. Our conclusions were not qualitatively changed by using uniform priors. We fit our models with 5000 total iterations per chain, 2500 warm-up iterations, four chains, and discarding every second sample to save memory. We checked for chain convergence visually and with the same criteria as before ($\widehat{R} < 1.05$ and $n_\mathrm{eff} >200$ for all parameters).

To derive taxonomic-order-level estimates of the probability of heavy tails accounting for time-series length (Fig.~\ref{fig:posteriors}b), we fit a separate multilevel model with the same structure but with only $\log$ time-series length as a predictor. (In this case, we did not want to control for intrinsic population characteristics such as density dependence.) Since our predictors were centered by subtracting their mean value, we obtained order-level estimates of the probability of heavy tails at mean log time-series length by adding the posteriors for $\alpha$, $\alpha^\mathrm{class}_j$, and $\alpha^\mathrm{order}_k$.

\bibliographystyle{ecologyletters}
\bibliography{/Users/seananderson/Dropbox/tex/jshort,/Users/seananderson/Dropbox/tex/ref3}

% ------------------------------
% Supplemental Tables
% ------------------------------

\clearpage
\renewcommand{\thetable}{S\arabic{table}}
\setcounter{table}{0}


\begin{table}
\begin{footnotesize}
\caption{Summary statistics for the filtered Global Population Dynamics Database time series arranged by taxonomic class. Columns are: number of populations, number of taxonomic orders, numbers of species, median time series length, total number of interpolated time steps, total number of substituted zeros, and total number of time steps.}
\smallskip
\begin{tabular}{lrrrrrrrr}
\toprule
% latex table generated in R 3.3.1 by xtable 1.8-2 package
Taxonomic class & Populations & Orders & Species & Median length & Interpolated pts & Zeros pts & Total pts \\ 
  \midrule
Aves & 191 &  15 & 112 & 27.00 &  68 &  32 & 6160 \\ 
  Insecta & 182 &   7 &  91 & 25.00 &  26 &  55 & 4812 \\ 
  Mammalia & 125 &   8 &  51 & 28.00 &  18 &  21 & 4027 \\ 
  Osteichthyes & 108 &   6 &  35 & 26.50 &  13 &   3 & 3310 \\ 
  Chondrichthyes &   1 &   1 &   1 & 20.00 &   1 &   0 &  20 \\ 
  Crustacea &   1 &   1 &   1 & 33.00 &   0 &   0 &  33 \\ 
  Gastropoda &   1 &   1 &   1 & 21.00 &   0 &   0 &  21 \\ 
   \bottomrule

\label{tab:stats}
\end{tabular}
\end{footnotesize}
\end{table}

\clearpage

\LTcapwidth=\textwidth
\bibpunct{}{}{;}{a}{}{;}

\singlespacing
\begin{footnotesize}
\begin{longtable}{>{\RaggedRight}m{1.5cm}>{\RaggedRight}p{4.3cm}>{\RaggedRight}p{0.8cm}>{\RaggedRight}p{1.7cm}>{\RaggedRight}p{1.0cm}>{\RaggedRight}p{3.0cm}>{\RaggedRight}p{1.7cm}>{\RaggedRight}p{1.3cm}}
\caption{All populations with Pr$(\nu < 10) > 0.5$ in the base heavy-tailed Gompertz population dynamics model. Shown are the log abundance time series, population descriptions, Global Population Dynamics Database Main IDs, citation for the data source or separate verification literature, a description of the cause of the black swan events (if known), the probability of heavy tails as calculated by our model, and median estimate of $\nu$ from our model with 90\% quantile credible intervals indicated in parentheses. Red dots on the time series indicate downward black-swan events and blue values indicate upward black-swan events. (TODO add what probability these are.)}\\
\toprule
% latex table generated in R 3.3.2 by xtable 1.8-2 package
Time series & Population & ID & Ref & Description & Pr($\nu < 10$) \\ 
  \midrule
\includegraphics[width=1.7cm]{../analysis/sparks/6528.pdf} & Shag, \textit{Phalacrocorax aristotelis}, Farne Islands, Northumberland & 6528 & \cite{potts1980} & Red tide event combined with low productivity due to overcrowding & 1.00 \\
  \includegraphics[width=1.7cm]{../analysis/sparks/7115.pdf} & South African fur seal, \textit{Arctocephalus pusillus}, South Africa & 7115 & \cite{shaughnessy1982} & Harvesting and predation changes & 1.00 \\
  \includegraphics[width=1.7cm]{../analysis/sparks/10127.pdf} & Red grouse, \textit{Lagopus lagopus scoticus}, Scotland - un-named area & 10127 & \cite{potts1984} & Environment- and parisite-caused cycles & 0.99 \\
  \includegraphics[width=1.7cm]{../analysis/sparks/9382.pdf} & Pine looper or Bordered white, \textit{Bupalus piniaria}, Kessock & 9382 & \cite{broekhuizen1993} & Unknown, but sampling intensity was decreasing & 1.00 \\
  \includegraphics[width=1.7cm]{../analysis/sparks/10128.pdf} & Red grouse, \textit{Lagopus lagopus scoticus}, Scotland - un-named area & 10128 & \cite{potts1984} & Environment- and parisite-caused cycles & 1.00 \\
  \includegraphics[width=1.7cm]{../analysis/sparks/10007.pdf} & Water vole, \textit{Arvicola terrestris}, Le Pont & 10007 & \cite{saucy1994} & Predator-environment cycle interactions & 1.00 \\
  \includegraphics[width=1.7cm]{../analysis/sparks/20579.pdf} & Grey heron, \textit{Ardea cinerea}, Southern Britain & 20579 & \cite{stafford1971} & Severe winter & 0.98 \\
  \includegraphics[width=1.7cm]{../analysis/sparks/9655.pdf} & Flea beetle, \textit{Chaetocnoma concinna}, Finland & 9655 & \cite{markkula1965} & Cannot locate original source & 0.99 \\
  \includegraphics[width=1.7cm]{../analysis/sparks/1235.pdf} & Wren, \textit{Troglodytes troglodytes}, Eastern Wood, Bookham Common & 1235 & \cite{newton1998} & Severe winter & 0.98 \\
  \includegraphics[width=1.7cm]{../analysis/sparks/9667.pdf} & Gooseberry sawfly, \textit{Nemastus ribesii}, Finland & 9667 & \cite{markkula1965} & Cannot locate original source & 0.97 \\
  \includegraphics[width=1.7cm]{../analysis/sparks/10113.pdf} & Willow grouse, \textit{Lagopus lagopus}, Northern England & 10113 & \cite{dobson1995} & Parasites and predators & 0.98 \\
  \includegraphics[width=1.7cm]{../analysis/sparks/9679.pdf} & Unknown, \textit{Trioza apicalis}, Finland & 9679 & \cite{markkula1965} & Cannot locate original source & 0.86 \\
  \includegraphics[width=1.7cm]{../analysis/sparks/20527.pdf} & Wandering albatross, \textit{Diomedea exulans}, Taiaroa & 20527 & \cite{robertson1998} & Unknown & 0.99 \\
  \includegraphics[width=1.7cm]{../analysis/sparks/10039.pdf} & Red grouse, \textit{Lagopus lagopus scoticus}, Northern Scotland & 10039 & \cite{dobson1995} & Parasites and predators & 0.93 \\
  \includegraphics[width=1.7cm]{../analysis/sparks/10162.pdf} & Red grouse, \textit{Lagopus lagopus scoticus}, Atholl Estate & 10162 & \cite{mackenzie1952} & Bad environmental conditions and overcrowding combined to create crashes & 0.92 \\
  \includegraphics[width=1.7cm]{../analysis/sparks/7099.pdf} & European rabbit, \textit{Oryctolagus cuniculus}, Estate 2, East Anglia & 7099 & \cite{barnes1986} & Disease outbreak followed by years of good weather & 0.78 \\
  \includegraphics[width=1.7cm]{../analysis/sparks/9503.pdf} & Fisher or  Pekan, \textit{Martes pennanti}, Manitoba & 9503 & \cite{keith1963} & Unknown & 0.76 \\
  \includegraphics[width=1.7cm]{../analysis/sparks/2778.pdf} & Wheatear, \textit{Oenanthe oenanthe}, Skokholm Island & 2778 & \cite{lack1969} & Unknown, but decline noted specifically, cold winters caused some crashes & 0.66 \\
  \includegraphics[width=1.7cm]{../analysis/sparks/9659.pdf} & Cabbage root fly or maggot, \textit{Delia radicum}, Finland & 9659 & \cite{markkula1965} & Cannot locate original source & 0.60 \\
   \bottomrule

\label{tab:causes-supp}
\end{longtable}
\end{footnotesize}
\onehalfspacing

% ------------------------------
% Supplemental Figures
% ------------------------------

\renewcommand{\thefigure}{S\arabic{figure}}
\setcounter{figure}{0}

\begin{centering}
\clearpage
\includegraphics[width=\textwidth]{../analysis/all-clean-ts-mammals.pdf}\\
Figure~\ref{fig:all-ts} (mammals) continued on next page \ldots

\clearpage
\includegraphics[width=\textwidth]{../analysis/all-clean-ts-birds.pdf}\\
Figure~\ref{fig:all-ts} (birds) continued on next page \ldots

\clearpage
\includegraphics[width=\textwidth]{../analysis/all-clean-ts-insects.pdf}\\
Figure~\ref{fig:all-ts} (insects) continued on next page \ldots

\end{centering}

\begin{figure}[htbp]
\begin{center}
\includegraphics[width=\textwidth]{../analysis/all-clean-ts-fishes-others.pdf}
\caption{
  (fishes, crustaceans, gastropods, sharks). All filtered time series used in our analysis. The abundances are shown on a log10 vertical axis. Throughout this figure, red dots indicate values that were interpolated and red circles indicate values that were recorded as zero but were set to the next lowest observed abundance. Numbers before each species name are the GPDD main identification numbers.
}
\label{fig:all-ts}
\end{center}
\end{figure}

\clearpage

\begin{figure}[htbp]
\begin{center}
\includegraphics[width=0.8\textwidth]{../analysis/priors-gomp-base.pdf}
\caption{
Probability density of the Bayesian priors for the Gompertz models. From left to right and then top to bottom: (1) per capita growth rate at $\log$(abundance) = $0$: $\lambda \sim \mathrm{Normal}(0, 10^2)$; (2) scale parameter of t-distribution process noise: $\sigma \sim \mathrm{Half\mhyphen Cauchy} (0, 2.5)$; (3) t-distribution degrees of freedom parameter: $\nu \sim \mathrm{Truncated\mhyphen Exponential}(0.01, \mathrm{min.} = 2)$; (4) AR1 correlation coefficient of residuals: $\phi \sim \mathrm{Truncated \mhyphen Normal}(0, 1, \mathrm{min.} = -1, \mathrm{max.} = 1)$. Not shown is $b$, the density dependence parameter: $b \sim \mathrm{Uniform}(-1, 2)$. The $\nu$ panel also shows two alternative priors: a weaker prior $\nu \sim \mathrm{Truncated\mhyphen Exponential}(0.005, \mathrm{min.} = 2)$, and a stronger prior $\nu \sim \mathrm{Truncated\mhyphen Exponential}(0.02, \mathrm{min.} = 2)$. The inset panel shows the same data but with a log-transformed x axis. Note that the base and weaker priors are relatively flat within the region of $\nu < 20$ that we are concerned with.
}
\label{fig:priors}
\end{center}
\end{figure}

\clearpage

\begin{figure}[htbp]
\begin{center}
\includegraphics[width=\textwidth]{../analysis/gomp-comparison.pdf}
\caption{
  Estimates of $\nu$ from alternative models plotted against the base Gompertz
  model estimates of $\nu$. Shown are medians of the posterior (dots) and 50\%
  credible intervals (segments). The diagonal line indicates a one-to-one
  relationship. Different colours indicate various taxonomic classes. The
  grey-shaded regions indicate regions of disagreement if $\nu = 10$ is taken
  as a threshold of heavy-tailed dynamics. The Gompertz observation error model
  assumes a fixed standard deviation of observation error of $0.2$ on a log
  scale.
}
\label{fig:alt}
\end{center}
\end{figure}

\clearpage

\begin{figure}[htbp]
\begin{center}
\includegraphics[width=\textwidth]{../analysis/gomp-prior-comparison.pdf}
\caption{
Estimates of $\nu$ from Gompertz models with alternative priors on $\nu$. Shown are medians of the posterior (dots) and 50\% credible intervals (segments). The diagonal line indicates a one-to-one relationship. Different colours indicate various taxonomic classes. The grey-shaded regions indicate regions of disagreement if $\nu = 10$ is taken as a threshold of heavy-tailed dynamics. The base, weaker, and stronger priors on $\nu$ are illustrated in Fig.~\ref{fig:priors}. In general, the estimates are nearly identical in cases where the data are informative about low values of $\nu$. When the data are less informative about low values of $\nu$, the prior can slightly pull the estimates of $\nu$ towards higher or lower values.
}
\label{fig:alt-priors}
\end{center}
\end{figure}

\clearpage

\begin{figure}[htbp]
\begin{center}
\includegraphics[width=0.8\textwidth]{../analysis/t-dist-sampling-sim-prior-exp0point01.pdf}
\includegraphics[width=0.8\textwidth]{../analysis/t-dist-sampling-sim-sigma-prior-exp0point01.pdf}
\caption{
  Testing the ability to estimate $\nu$ (top panels) and the scale parameter of
  the process deviations (bottom panels) for a given number of samples (columns)
  drawn from a distribution with a given true $\nu$ value (rows). The red lines
  indicate the true population value. When a small number of samples are drawn
  there may not be samples sufficiently far into the tails to recapture the
  true $\nu$ value; however, heavy tails are still distinguished from normal
  tails in most cases, even with only 25 or 50 samples.
}
\label{fig:sim-nu}
\end{center}
\end{figure}

\clearpage

\begin{figure}[htbp]
\begin{center}
\includegraphics[width=\textwidth]{../analysis/check-sim-box.pdf}
\caption{
Simulation testing the Gompertz estimation model when the process deviation
draws were chosen so that $\nu$ could be estimated close to the true value
outside the full population model (``effective $\nu$'' within a CV of 0.2 of
specified $\nu$).
Plots show the probability that $\nu < 10$ (i.e.\ the approximate probability
of heavy tails) across three population $\nu$ values
(3, 5, and normal) and different scenarios panels: (1) 100 time steps and no
observation error, (2) 50 time steps and no observation error, (3) 50 time
steps and observation error drawn from $\mathrm{Normal} (0, 0.2^2)$ but
ignored, and (4) 50 time steps with observation error in which the quantity
of observation error was assumed known.
Within
each scenario the dots represent stochastic draws from the true population
distributions combined with model fits.
}
\label{fig:sim-prob}
\end{center}
\end{figure}

\clearpage

\noindent
Example Stan code for a heavy-tailed Gompertz model with AR1 correlated
residuals and a specified level of observation error. The specific code for used for the various models in our analysis is available at \url{https://github.com/seananderson/heavy-tails}.

\begin{spacing}{1.15}
\begin{footnotesize}
\begin{verbatim}
data {
  int<lower=3> N;              // number of observations
  vector[N] y;                 // vector to hold ln abundance observations
  real<lower=0> nu_rate;       // rate parameter for nu exponential prior
}
parameters {
  real lambda;                 // Gompertz growth rate parameter
  real<lower=-1, upper=2> b;   // Gompertz density dependence parameter
  real<lower=0> sigma_proc;    // process noise scale parameter
  real<lower=2> nu;            // t-distribution degrees of freedom
  real<lower=-1, upper=1> phi; // AR1 parameter
  vector[N] U;                 // unobserved states
  real<lower=0> sigma_obs;     // specified observation error SD
}
transformed parameters {
  vector[N] epsilon;           // error terms
  epsilon[1] <- 0;
  for (i in 2:N) {
    epsilon[i] <- U[i] - (lambda + b * U[i - 1])
                       - (phi * epsilon[i - 1]);
  }
}
model {
  // priors:
  nu ~ exponential(nu_rate);
  lambda ~ normal(0, 10);
  sigma_proc ~ cauchy(0, 2.5);
  phi ~ normal(0, 1);
  // data model:
  for (i in 2:N) {
    U[i] ~ student_t(nu,
                     lambda + b * U[i - 1]
                     + phi * epsilon[i - 1],
                     sigma_proc);
  }
  y ~ normal(U, sigma_obs);
}
\end{verbatim}
\end{footnotesize}

\clearpage
\noindent
Stan code for the multilevel beta regression:
\begin{footnotesize}
\verbatiminput{../analysis/betareg4.stan}
\end{footnotesize}

\clearpage

\noindent
The GPDD IDs used in our analysis.

\begin{footnotesize}
\noindent
{\tt
1 3 4 5 6 7 8 9 10 11 12 13 14 15 16 17 18 44 45 46 47 58 61 64 1149 1150 1153 1157 1159 1160 1162 1163 1165 1166 1168 1169 1170 1173 1174 1177 1179 1184 1185 1188 1189 1190 1195 1196 1197 1199 1200 1201 1202 1203 1204 1205 1206 1217 1227 1228 1229 1233 1234 1235 1237 1238 1239 1240 1243 1244 1247 1342 1377 1522 1523 1524 1525 1534 1602 1613 1618 1633 1660 1663 1664 1667 1669 1670 1671 1674 1682 1683 1792 1826 1829 1830 1831 1865 1866 1868 1869 1870 1875 1876 1880 1881 1883 1885 1886 1887 1888 1893 1894 1927 1964 1965 1966 1968 1970 1971 1973 1974 1976 1981 1982 1983 1986 1987 1991 1992 1993 1994 1998 1999 2003 2004 2005 2006 2007 2012 2013 2015 2016 2017 2018 2019 2020 2024 2025 2026 2027 2028 2031 2032 2033 2034 2066 2721 2722 2726 2732 2735 2736 2757 2758 2759 2770 2771 2772 2774 2775 2777 2778 2781 2829 2844 2857 2867 2869 2887 2903 2915 2974 2976 2991 3001 3003 3017 3051 3056 3059 3068 3214 3216 3218 3233 3249 3251 3253 3260 3265 3283 3356 3358 3360 3378 3442 3466 3468 3470 3477 3482 3508 3521 3625 3627 3639 3664 3673 3676 3678 3680 3706 3708 3716 3774 3776 3784 3795 3799 3811 3827 3829 3838 3840 3853 3866 3882 5019 5020 5032 5034 5035 5039 6057 6144 6527 6528 6529 6530 6532 6533 6534 6535 6536 6537 6539 6541 6542 6547 6548 6549 6550 6553 6554 6555 6556 6558 6560 6561 6562 6564 6565 6567 6568 6569 6570 6581 6582 6583 6633 6673 6674 6675 6676 6677 6678 6681 6683 6684 6685 6686 6687 6688 6770 6865 6867 6868 6869 6870 6876 6882 6885 6889 6890 6902 6904 6917 6920 6921 6922 6939 6940 6973 7048 7052 7053 7054 7060 7061 7067 7088 7089 7091 7092 7093 7094 7098 7099 7101 7102 7115 7116 9191 9192 9194 9195 9196 9200 9211 9215 9216 9217 9218 9219 9220 9221 9222 9223 9224 9225 9232 9308 9309 9330 9331 9381 9382 9393 9436 9437 9438 9439 9440 9441 9442 9443 9444 9445 9446 9468 9469 9470 9472 9477 9486 9488 9489 9490 9491 9492 9500 9501 9502 9503 9506 9515 9517 9518 9519 9586 9587 9606 9611 9612 9639 9641 9642 9644 9646 9647 9648 9650 9652 9654 9655 9656 9657 9658 9659 9661 9662 9663 9665 9667 9668 9669 9672 9673 9674 9675 9676 9677 9678 9679 9680 9681 9682 9688 9689 9690 9691 9793 9794 9795 9796 9797 9835 9836 9893 9894 9895 9896 9897 9898 9899 9900 9901 9902 9903 9904 9905 9907 9919 9921 9932 9933 9934 9936 9938 9948 9949 9950 9951 9953 9990 9991 9993 9994 9995 9997 9998 9999 10000 10001 10002 10005 10006 10007 10008 10009 10010 10011 10012 10013 10029 10030 10031 10036 10039 10040 10041 10042 10044 10045 10046 10047 10048 10049 10050 10051 10053 10054 10055 10060 10061 10063 10065 10070 10071 10085 10088 10089 10090 10092 10093 10094 10096 10097 10098 10099 10100 10101 10110 10111 10112 10113 10114 10117 10118 10120 10121 10122 10123 10124 10125 10127 10128 10131 10134 10136 10137 10140 10141 10142 10143 10144 10145 10149 10153 10156 10158 10159 10160 10161 10162 10163 10164 10165 20527 20530 20532 20534 20535 20536 20537 20539 20540 20541 20542 20543 20544 20546 20547 20548 20549 20550 20551 20552 20553 20555 20577 20578 20579 20580 20581 20582 20583 20587 20626 20628 20634 20635 20636 20639 20649 20650 20651 20652 20653 20654 20655 20656 20657 20658 20659 20660 20662 20663
}
\end{footnotesize}
\end{spacing}
