\documentclass[11pt]{article}
\usepackage{geometry}
\geometry{verbose,letterpaper,tmargin=2.54cm,bmargin=2.54cm,lmargin=2.54cm,rmargin=2.54cm}
\geometry{letterpaper}
\usepackage{graphicx}
\usepackage{amssymb}
\usepackage{amsmath}
\usepackage{amsfonts}
%\usepackage{pdfpages}
%\usepackage{lscape}
%\usepackage{pdflscape}
\usepackage{setspace}
\usepackage{booktabs}
\usepackage{lineno}

\usepackage{longtable}
\usepackage{multirow}
\usepackage{array}
\usepackage{ragged2e}
\usepackage{tabu} % for spacing between rows in longtable
\setlength{\tabulinesep}{9pt}

% Palatino
\usepackage[full]{textcomp}
     \usepackage[osf]{newpxtext} % osf for text, not math
     \usepackage{cabin} % sans serif
     \usepackage[varqu,varl]{inconsolata} % sans serif typewriter
     \usepackage[bigdelims,vvarbb]{newpxmath} % bb from STIX
     \usepackage[cal=boondoxo]{mathalfa} % mathcal

 % Linux Libertine:
%  \usepackage{textcomp}
%     \usepackage[sb]{libertine}
%     \usepackage[varqu,varl]{inconsolata}% sans serif typewriter
%     \usepackage[libertine,bigdelims,vvarbb]{newtxmath} % bb from STIX
%     \usepackage[cal=boondoxo]{mathalfa} % mathcal
%     \useosf % osf for text, not math
%     \usepackage[supstfm=libertinesups,%
%       supscaled=1.2,%
%       raised=-.13em]{superiors}


\mathchardef\mhyphen="2D

\textheight 22.0cm

\usepackage[round]{natbib}
\bibliographystyle{apalike}
\bibpunct{(}{)}{;}{a}{}{;}

\title{Black swans in ecological time series\\Black swans in population dynamics\\Heavy-tailed ecological time series}
\author{
Sean C. Anderson$^{1\ast}$ \and
Trevor A. Branch$^2$ \and
Andrew B. Cooper$^3$ \and
Nicholas K. Dulvy$^1$
}
\date{}

% remove numbers in front of sections:
\makeatletter
\renewcommand\@seccntformat[1]{}
\makeatother

\begin{document}
\newcommand{\basePriorMean}{102}
\newcommand{\basePriorMedian}{71}
\newcommand{\basePriorProbHeavy}{7.7}
\newcommand{\medianTimeSteps}{26}
\newcommand{\meanTimeSteps}{30.2}
\newcommand{\minTimeSteps}{20}
\newcommand{\maxTimeSteps}{117}
\newcommand{\birdN}{191}
\newcommand{\insectsN}{182}
\newcommand{\mammalsN}{125}
\newcommand{\fishN}{108}
\newcommand{\birdNH}{14}
\newcommand{\insectsNH}{5}
\newcommand{\mammalsNH}{6}
\newcommand{\fishNH}{0}
\newcommand{\birdPH}{7}
\newcommand{\insectsPH}{3}
\newcommand{\mammalsPH}{5}
\newcommand{\fishPH}{0}
\newcommand{\NOrdersHeavy}{16}
\newcommand{\POrdersHeavy}{41}
\newcommand{\baseFiftyObsFiftySwitch}{8}
\newcommand{\baseSeventyFiveObsFiftySwitch}{2}
\newcommand{\totalHeavyFifty}{26}
\newcommand{\totalHeavySeventyFive}{17}
\newcommand{\baseFiftyObsFiftySwitchPerc}{31}
\newcommand{\baseSeventyFiveObsFiftySwitchPerc}{8}
\newcommand{\baseNuTenObsTenSwitch}{8}
\newcommand{\baseNuTen}{26}
\newcommand{\baseNuFiveObsTenSwitch}{2}
\newcommand{\pHeavyNThirty}{0.13}
\newcommand{\pHeavyNSixty}{0.20}
\newcommand{\pIncHeavyNThirtyNSixty}{1.6}
\newcommand{\obsErrorNuFivePerc}{75}
\newcommand{\modelsNoConvergeAROne}{1}
\newcommand{\modelsNoConvergeAROneHeavyBase}{0}
\newcommand{\percImputedPops}{17}
\newcommand{\percImputedPoints}{0.7}
\newcommand{\nuCoefPopN}{606}
\newcommand{\AvesRangePerc}{4--8}
\newcommand{\InsectaRangePerc}{2--3}
\newcommand{\MammaliaRangePerc}{4--6}
\newcommand{\OsteichthyesRangePerc}{0}
\newcommand{\overallMinPerc}{3}
\newcommand{\overallMaxPerc}{5}
\newcommand{\overallBasePerc}{4}
\newcommand{\NPops}{609}
\newcommand{\NOrders}{39}
\newcommand{\NClasses}{7}
\newcommand{\interpPointsPerc}{1}
\newcommand{\nBSUp}{8}
\newcommand{\nBSDown}{51}
\newcommand{\ratioBSDownToUp}{6.4}
\newcommand{\percBSDown}{86}
\newcommand{\crashUnderRange}{1.1--2}
\newcommand{\crashUnderMedian}{1.3}
\newcommand{\probDensSkewedForHeavyPops}{86}
\newcommand{\percNormPopsNotSkewed}{88}
 % R output

%\begin{spacing}{1.4}
%\doublespacing
\onehalfspacing

\maketitle
\thispagestyle{empty}

\noindent
\textit{Authors are listed alphabetically (ABCD!) and open to reordering.}

\noindent
\textsuperscript{1}Earth to Ocean Research Group, Department of Biological Sciences, Simon Fraser University, Burnaby BC, V5A 1S6, Canada

\noindent
\textsuperscript{2}School of Aquatic and Fishery Sciences, University of Washington, Box 355020, Seattle, WA 98195, USA

\noindent
\textsuperscript{3}School of Resource and Environmental Management, Simon Fraser University, Burnaby, BC, V5A 1S6, Canada

\noindent
\textsuperscript{*}Corresponding author: Sean C. Anderson; Earth to Ocean Research Group, Department of Biological Sciences, Simon Fraser University, Burnaby BC, V5A 1S6; Phone: 1-778-782-3989; E-mail: sean\_anderson@sfu.ca

\clearpage

\setcounter{page}{1}

\section{Abstract}
- introduce black swans and heavy tails, perhaps use the climate and increasing climate variability hook

- we find evidence of heavy tails/black swans in ecological time series, but they are rare

- taxonomically widespread although most frequent in birds, followed by mammals, insects

- we don't find compelling intrinsic biological (e.g.\ lifehistory) correlates

- common causes were extrinsic extreme events or multiple extreme events in a row: climate and severe winters, cycles of natural enemies (parasites and predators) and interactions between these

- also mention the advance of our method: probabilistic in quantifying heavy tailedness, incorporates population dynamics, can allow for autocorrelation and observation error, and incorporate prior information

- conclude with ecological, modelling, policy, and management implications

\section{Introduction}

Increasing realization about the importance of extreme events in the environment (REFs)
and of the importance of ecological surprises and black swans (REFs)

Papers to work in:

\citep{inchausti2002,halley2002,inchausti2001}

\citep{jentsch2007}

\citep{ward2007}

\citep{garcia-carreras2011}
\citep{sornette2009}

\citep{nunez2012}

\citep{thompson2013}
\citep{beaugrand2012}
\citep{pine-iii2009}

\citep{doak2008}

\citep{smale2013}

\citep{easterling2000}
\citep{scheffer2003}
\citep{katz2005}

\citep{taleb2007}

\citep{vasseur2014}

\citep{vert-pre2013}
\citep{lindenmayer2010}

\citep{valpine2002} 
\citep{gregory2010}
\citet{garcia-carreras2011}
\citet{brook2006}
\citep{herrandoprez2014}

\citep{sibly2005, ziebarth2010}

\ldots

\section{Methods}

The data: briefly introduce the GPDD and how we filtered it. Expand in the supplement. Mention the breakdown of taxonomy and interpolation (Table~S1).

The Student-t Gompertz model is:
\begin{align*}
x_t &= \ln N_t\\
x_t &= \lambda + b x_{t-1} + \epsilon_t\\
\epsilon_t &\sim \mathrm{Student\mhyphen t}_\nu(0, \sigma),
\end{align*}

\noindent where $x_t$ is the $\ln$ abundance at time $t$. The model is density independent if $b = 1$, maximally density dependent if $b = 0$, and inversely density dependence if $b < 0$. The parameter $\lambda$ represents the expected population growth rate at $x_t = 0$. The process noise, $\epsilon_t$, is modelled as a Student-t distribution that is centered at $0$, has a scale parameter of $\sigma_\mathrm{proc}$, and has a degrees of freedom parameter of $\nu$. If $\nu$ is small ($\lesssim 10$) the distribution has much heavier tails than a normal distribution.  For example, at $\nu = 2$, the probability of drawing a value less than -5 is 1.8\%, whereas the probability of drawing such a value from a normal distribution is nearly zero ($2.9\cdot10^{-5}$\%). As $\nu$ approaches infinity the distribution approaches the normal distribution (Fig~\ref{fig:didactic}). By estimating the value of $\nu$ we can quantity the degree of heavy-tailedness in the process noise jumps from $\ln N_t$ to $\ln N_{t+1}$.

We fit alternative models that allowed for autocorrelation of the residuals, allowed for observation error, and allowed the functional form of the population dynamics to be represented as Ricker-logistic (Supporting Material).

Mention that we simulation tested the model.

Mention the priors (Fig.~\ref{fig:priors}) and expand on their justification in the supplement.

$\nu$ prior: \citep{fernandez1998}
$\sigma$ prior: \citep{gelman2006c}
consider alternative inverse prior on $\nu$ as in \citep{gelman2014}

mention data used for life-history hypothesis comparisons from \citet{brook2006a}

justify why we only present these as hypotheses but do not formally model them 

\section{Results}

- quantify how frequent and strong heavy tails were (Fig.~\ref{fig:nu-coefs})

- plot showing all p($\nu < 10$) $> 0.5$ (Fig.~\ref{fig:heavy-ts})

- across how many different species

- break down by taxonomy  (Fig.~\ref{fig:nu-coefs} inset panels)

- no obvious intrinsic biological drivers (Fig.~\ref{fig:correlates})

- mention the common causes (Table~\ref{tab:sparks})

- touch on the supplemental figures --- simulation results (Fig.~\ref{fig:sim-nu}, Fig.~\ref{fig:sim-gompertz}, Fig.~\ref{fig:sim-prob})

- alternative models: (Fig.~\ref{fig:ar1}, Fig.~\ref{fig:gomp-vs-ricker})

\section{Discussion}

- why we think we see them in some taxa but not others (observation scale dynamics mismatch; observation errors...)

Ways forward:

- hierarchical modelling of nu and other parameters

- better datasets for specific taxa (e.g.\ fish recruitment)

- joint prior on scale and nu parameters

- formally investigating what heavy tails are associated with

- do heavy tails have conservation importance?

- can we forecast their probability and develop correlates?

- how can we detect them relatively quickly after they happen?

- how often are heavy tails real? observation error, recording errors...

- what does this mean for how we model time series (should we consider using the t as a default even if we fix nu?)

- what does this mean for conservation management (crashes might be more frequent and greater in magnitude than our typical models tell us) --- importance of establishing polices that are robust to this

\section{Acknowledgements}

Silhouettes: \texttt{phylopic.org}: rabbit by Sarah Werning, grey heron by Ardea cinerea, hoverfly by Gareth Monger. All under Creative Commons Attribution 3.0 Unported license.

GPDD

Funding

\bibliography{/Users/seananderson/Dropbox/tex/jshort,/Users/seananderson/Dropbox/tex/ref3}

\clearpage

\section{Tables}

\LTcapwidth=\textwidth
\bibpunct{}{}{;}{a}{}{;}

\begin{small}
\begin{longtable}{>{\RaggedRight}m{2.0cm}>{\RaggedRight}p{3.0cm}>{\RaggedRight}p{7.0cm}>{\RaggedRight}p{2.0cm}}

\caption{Example population dynamic black swans from the Global Population Dynamics Database and descriptions of their causes. Red and blue dots indicate downward and upward events that have a probability of occurring of 0.001\% or less if the population dynamics were explained by a Gompertz model with normally distributed process noise. These populations are a non-random sample that we were able to verify in the primary literature.}\\

\toprule
Time series (log scale) & Population & Black swan description & Reference \\
\midrule

\includegraphics[width=2cm]{sparks/6528} &
Shag,
\textit{Phalacrocorax aristotelis},
UK &
Shortage of nest sites reduced productivity; red-tide event in 1968 caused extreme mortality; no longer a nest shortage; population rapidly increased &
\citep{potts1980}\\

\includegraphics[width=2cm]{sparks/10007} &
Water vole,
\textit{Arvicola terrestris},
UK &
Short-term population cycles from predator interactions combined with longterm environmental cycle caused sharp downswing  &
\citep{saucy1994}\\

\includegraphics[width=2cm]{sparks/7115} &
Fur seal,
\textit{Arctocephalus pusillus},
South Africa &
Strong decreases in harvesting, loss of predators, and diamond mining regulations reducing human traffic caused sharp upswings  &
\citep{shaughnessy1982}\\


\includegraphics[width=2cm]{sparks/10113} &
Willow grouse,
\textit{Lagopus lagopus},
UK &
Parasite and predation effects interacted to cause low years  &
\citep{dobson1995}\\

\includegraphics[width=2cm]{sparks/10162} &
Red grouse,
\textit{Lagopus lagopus scoticus},
UK &
Good environmental conditions produced high numbers and vulnerable populations; bad conditions and overcrowding combined to create crashes  &
\citep{mackenzie1952}\\

\includegraphics[width=2cm]{sparks/1235} &
Wren,
\textit{Troglodytes troglodytes},
UK &
Severe winters where food was submerged in snow caused population crash &
\citep{newton1998} \\

\includegraphics[width=2cm]{sparks/20579} &
Grey heron,
\textit{Ardea cinerea},
UK &
Severe winters in 1929, 1940--1942, and 1962--1963; 1963 event so severe that recovery took three times as long as expected &
\citep{stafford1971} \\

%\includegraphics[width=2cm]{sparks/20580} &
%Chamois, \textit{Rupicapra rupicapra}, Switzerland &
% &
%\citep{brook2006a}\\

%\includegraphics[width=2cm]{sparks/5019} &
%Barbary macaque,  &
% &
%REF\\

%\includegraphics[width=2cm]{sparks/9675} &
%Carrot fly (\textit{Psila rosae}, Finland &
% &
%\citep{markkula1965}\\

%\includegraphics[width=2cm]{sparks/10139} &
%Grey heron (\textit{Ardea cinerea}), UK &
% &
%\citep{stafford1971}\\

\bottomrule
\label{tab:sparks}
\end{longtable}
\end{small}

% reset citation style:
\bibpunct{(}{)}{;}{a}{}{;}

\clearpage

\section{Figures}

\begin{figure}[htbp]
\begin{center}
\includegraphics[width=\textwidth]{t-nu-eg.pdf}
\caption{An illustration of fitting population dynamics models that allow for heavy tails, represented by the Student-t degrees of freedom parameter $\nu$.
  (a, b) The probability density for t distributions with a scale parameter of 1 and different values of $\nu$.
  Small values of $\nu$ create heavy tails.
  As $\nu$ approaches infinity the distribution approaches the normal distribution.
  For example, at $\nu = 2$, the probability of drawing a value less than -5 is 1.8\%, whereas the probability of drawing such a value from a normal distribution is nearly zero ($2.9\cdot10^{-5}$\%).
  (c--e) Simulated population dynamics from a Gompertz model with process noise drawn from t distributions with different values of $\nu$.  
  Coloured dots in panels c and d represent jumpts with less than a 1 in 1000 chance of occurring in a normal distribution.
  (f--h) Estimates of $\nu$ from models fit to the times series in panels c--e.
  Shown are the posterior samples (histograms), median and interquartile range of the posterior (IQR) (dots and line segments), and the exponential prior on $\nu$ (dashed lines).
  Colour shading behind panels f--h illustrates the region of heavy tails.}
\label{fig:didactic}
\end{center}
\end{figure}

\clearpage

\begin{figure}[htbp]
\begin{center}
\includegraphics[width=0.46\textwidth]{nu-coefs-2.pdf}
\caption{
Estimates of how heavy-tailed population dynamics are for \nuCoefPopN\ populations of birds, mammals, insects, and fishes. Small values of $\nu$ ($\lesssim 10$) suggest heavy-tailed dynamics; larger values of $\nu$ suggest approximately normal-tailed dynamics. Vertical points and line segments represent posterior medians and 50\% / 90\% credible intervals for individual populations. Inset plots show probability that $\nu < 10$ (probability of heavy tails) for populations arranged by taxonomic order. Taxonomic orders are sorted by decreasing mean p($\nu < 10$).
Taxonomic orders with three or fewer populations in panel a are omitted for space.}\label{fig:nu-coefs}
\end{center}
\end{figure}

\clearpage

%\begin{figure}[htbp]
%\begin{center}
%\includegraphics[width=\textwidth]{gomp-base-p10-dot-order.pdf}
%\caption{
%  Probability $\nu<20$, i.e.\ one way of asking what the probability is of heavy tails.
%  Dots represent populations.
%  Something like this could be made more pretty and sorted in some appropriate way\ldots maybe with little silhouettes for the animals.
%  Looks pretty similar for $\nu < 10$, with the probabilities a bit lower.
%  There is some vertical jitter added. Could do this and have another subpanel that breaks down the mammals by family, since there's the most interesting stuff happening there and there are lots of them... or just colour the dots by family within each panel.}
%\label{fig:dots}
%\end{center}
%\end{figure}

\clearpage

\begin{figure}[htbp]
\begin{center}
\includegraphics[width=0.9\textwidth]{correlates.pdf}
\caption{
Potential correlates of heavy-tailed behaviour (indicated by low $\nu$ values). Shown are (a--c) parameters from the Gompertz heavy-tailed population model, (b) number of time steps, (c) body length, and (d) lifespan.
For the Gompertz parameters, $\sigma$ refers to the scale parameter of the Student-t process-noise distribution, $\lambda$ refers to the population growth rate at an abundance of one, $b$ refers to the density dependence parameter ($1$ is maximally density independent, $0$ is maximally density dependent, and $<0$ is inversely density dependent).
For parameter values, points and lines represent medians and 50\% credible intervals of the posterior.
Circles representing four populations are filled with white and represent a shark, crustaceans, and one gastropod.
}
\label{fig:correlates}
\end{center}
\end{figure}

\clearpage

\begin{centering}
\LARGE
Supporting material\\[1.5em]
\end{centering}

\section{Data selection}

To TODO \ldots

\begin{enumerate}

\item To remove populations with unreliable population indices that could be strongly confounded with economics and sampling effort, we removed all populations with a sampling protocol listed as \texttt{"harvest"} as well populations with the words \texttt{"harvest"} or \texttt{"fur"} in the cited reference title.
\item We removed all populations with uneven sampling intervals. I.e\ we removed populations that didn't have a constant difference between the ``decimal year begin'' and ``decimal year end'' columns.
\item We removed all populations rated as $< 2$ in the GPDD quality assessment (on a scale of $1$ to $5$, with $1$ being the lowest quality data) \citep[following][]{sibly2005, ziebarth2010}
\item Populations with negative abundance values were assumed to be log values and transformed by taking $10$ to the power of the recorded abundance. In many cases this was noted for the population, but not in all cases. We inspected each of these  \totalAssumedLog\ time series to make sure our assumption made sense (Fig.~\ref{fig:log10-assumed}).
\item We filled in all missing time steps with \texttt{NA} values and imputed single missing values with the geometric mean of the previous and following values. We chose a geometric mean to be linear on the log scale that the Gompertz and Ricker-logistic models were fit on.
\item We filled in single recorded values of zero with the lowest non-zero value in the time series \citep[following][]{brook2006a}. This assumes that single values of zero result from abundance being low enough that censusing missed present individuals. We turned multiple zero values in a row into \texttt{NA} values. This implies that multiple zero values were either censusing errors or caused by emigration. Regardless, our population  models were fit on a multiplicative (log) scale and so could not account for zero abundance.
\item We removed all populations with four or more identical values in a row since these suggest either recording error or extrapolation between two observations.
\item We removed all populations without at least four unique values \citep[following][]{brook2006a}.
\item We then wrote an algorithm to find the longest unbroken window of abundance (no \texttt{NA}s) with at least 20 time steps in each population time series.  If there were any populations with multiple windows of identical length, we took the most recent window. This is a longer window than used in some previous analyses \citep[e.g.][]{brook2006a}, but since our model attempts to capture the shape of the distribution tails, our model requires more data.
\item Finally, we removed GPDD Main ID \texttt{20531}, which we noticed was a duplicate of \texttt{10139} (a heron population). 
\end{enumerate}

We provide a supplemental figure of all the time series included in our analysis and indicate which values were interpolated (\percImputedPops\% of populations had at least one point interpolated and only \percImputedPoints\% of the total observations were interpolated) (Fig.\ SX). Table~S1 shows the final taxonomic breakdown and the number of populations with interpolated values.

\section{Details on the heavy-tailed Gompertz probability model}

For the Gompertz, the mostly weakly-informative priors I'm using are:
\begin{align*}
b &\sim \mathrm{Uniform}(-1, 2)\\
\lambda &\sim \mathrm{Normal}(0, 10^2)\\
\nu &\sim \mathrm{Exponential}(0.01)\\
\sigma &\sim \mathrm{Half\mhyphen Cauchy} (5).
\end{align*}

See Figure~\ref{fig:priors}.

Priors justification: (TODO this is old)

\begin{itemize}
\item $b$ is bounded just past stationary so we can detect if they are non-stationary will keeping the sampler from wandering off too far

\item $\lambda$ prior (variance = 100) is basically uninformative within the range of expected values for population growth... it allows even a X probability at a value of X

\item $\nu$ is based on \citet{fernandez1998}; they chose a more informative 0.1 value, we chose a less informative 0.01 and justify it based on performance in supplemental figure X (gives X probability of value less than 10) but constrains the sampling somewhat since above 40 or 50 the shape of the t distribution is almost identical and data are not usually informative at these data quantities... sampler would head off to infinity otherwise

\item $\phi$ given a standard deviation of 1 given prior analyses which suggest autocorrelation in these datasets is minimal; prior is weak enough to allow high or low values, but given little information will stay near our expectation; similar approach in \citep{thorson2014a}

\item $\sigma_\mathrm{proc}$ can be justified based on \citet{gelman2006c} and the expected range of this variable in nature from previous studies \citep[e.g.][]{connors2014}
\end{itemize}

Using Stan 2.4.0 \citep{stan-manual2014}, and R 3.1.1 \citep{r2014} I'm starting with 4 chains and 1000 iterations with the first 500 as warmup (i.e.\ 2000 total samples). If rhat is greater than 1.05 for any parameter or the minimum effective sample size is less than 200 for any parameter then I double both the total iterations and warmup and run again. These conditions were met by 8000 iterations (16000 total samples) in all cases and in almost all cases (X\%) within the initial 1000 iteration case (i.e.\ a total of 2000 samples).


\section{Simulation testing the model}

Throughout all of this --- show that the model, if anything, under-predicts heavy tails but with sufficient data is unbiased.

2 parts: how many samples from the true population t distribution do you need to detect low nu? And, given that you have a set of deviations in which nu is detectable (effective nu is within 0.5 CV of true nu), can the more complex Gompertz still capture this?

First part: We drew from t distributions with different nu values and mean of 0, scale of 1. We started with 1600 samples and then fitted again at the first 800, 400, 200,  100, 50, 25. Each time we recorded the nu posterior.

Second part: To generate a series of process deviations with an effective nu approximately equal to the true nu, we generated process deviation sets repeatedly and estimated the mean, scale, and nu values each time. We recorded when the estimated nu was within 0.5 CVs of the true nu and used this set of random seed values in our Gompertz simulation. We then fit AR1 Gompertz models to the simulated datasets with all parameters (except nu) set near the median values estimated in the GPDD.

\section{Alternative population models}

\subsection{Ricker-logistic}

We also fitted a Ricker-logistic model:
\begin{align*}
x_t &= x_{t-1} + r_{\mathrm{max}}\left(1 - \frac{N_{t-1}}{K}\right) + b x_{t-1} + \epsilon_t\\
\epsilon_t &\sim \mathrm{Student\mhyphen t}_\nu(0, \sigma),
\end{align*}
\noindent
where  $r_\mathrm{max}$ represents the maximum population growth rate that is obtained when $N$ (abundance) $= 0$. The parameter $K$ represents the carrying capacity and, as before, $x_t$ represents the $\ln$ transformed abundance at time $t$. The Ricker-logistic model assumes a linear decrease in population growth rate ($x_t / x_{t-1}$) with increases in abundance ($N_t$). In contrast, the Gompertz model assumes a linear decrease in population growth rate with increases in $\ln$ abundance ($x_t$) (REF).

To fit the Ricker-logistic models ,we chose a prior on $K$ uniform between 0 and twice the maximum observed abundance (\citet{clark2010} chose 0 and maximum observed, which is less conservative). We set the prior on $r_\mathrm{max}$ as uniform between 0 and 20 as in \citet{clark2010}. We used the same priors on $\nu$ and $\sigma$ as in the Gompertz model.

\subsection{Autocorrelated residuals}

We considered a version of the Gompertz model in which an autoregressive parameter was fit to the process noise residuals:
\begin{align*}
x_t &= \lambda + b x_{t-1} + \epsilon_t\\
\epsilon_t &\sim \mathrm{Student\mhyphen t}_\nu(\phi \epsilon_{t-1}, \sigma).
\end{align*}
\noindent
In addition to the parameters in the original Gompertz model, an additional parameter, $\phi$, is estimated that represents the relationship between of subsequent process noise residuals. Based on the results of previous analyses with the GPDD \citep[e.g.][]{connors2014} and the chosen priors in previous analyses \citep[e.g.][]{thorson2014a} and to greatly speed up chain convergence when running our model across all populations, we placed a weakly informative prior on $\phi$ that assumed the greatest probability density near zero with the reduced possibility of $\phi$ being near $-1$ or $1$. Specifically, we chose $\phi \sim \mathrm{Truncated\mhyphen Normal}(0, 1, \mathrm{min.} = -0.99, \mathrm{max.} = 0.99)$.

\subsection{Assumed observation error}

We considered a version of the base Gompertz model that allowed for a specified level of observation error:
\begin{align*}
U_t &= \lambda + b U_{t-1} + \epsilon_t\\
x_t &\sim \mathrm{Normal}(U_t, \sigma_\mathrm{obs})\\
\epsilon_t &\sim \mathrm{Student\mhyphen t}_\nu(0, \sigma_\mathrm{proc}),
\end{align*}

\noindent
where $U$ represents the unobserved state vector, and $\sigma_\mathrm{obs}$ represents the standard deviation of observation error (on a log scale), which was set at $0.3$. TODO justify $0.3$.

%\end{spacing}

\renewcommand{\thetable}{S\arabic{table}}
\setcounter{table}{0}

% latex table generated in R 3.1.1 by xtable 1.7-3 package
% Thu Sep 25 13:22:15 2014
\begin{table}[ht]
\centering
\caption{Summary statistics for the filtered Global Population Dynamics Database time series arranged by taxonomic class. Columns are: number of populations, number of taxonomic orders, numbers of species, median time series length, total number of interpolated time steps, and total number of substituted zeros.} 
\begin{tabular}{lrrrrrr}
  \toprule
Taxonomic class & Populations & Orders & Species & Median length & Interpolated pts & Zeros pts \\ 
  \midrule
Aves & 227 &  15 & 113 &  26 &  68 &  32 \\ 
  Insecta & 182 &   7 &  91 &  25 &  26 &  55 \\ 
  Mammalia & 125 &   8 &  51 &  28 &  18 &  21 \\ 
  Osteichthyes & 109 &   6 &  36 &  26 &  14 &   4 \\ 
  Crustacea &  13 &   2 &   2 &  28 &   1 &   0 \\ 
  Chondrichtyhes &   1 &   1 &   1 &  20 &   1 &   0 \\ 
  Gastropoda &   1 &   1 &   1 &  21 &   0 &   0 \\ 
   \bottomrule
\end{tabular}
\end{table}


\renewcommand{\thefigure}{S\arabic{figure}}
\setcounter{figure}{0}

% \begin{figure}[htbp]
% \begin{center}
% \includegraphics[width=\textwidth]{nu-covariates-p-base.pdf}
% \caption{Same as previous figure but the y axis is p(nu $<$ 20), which I think might be better to use\ldots, in reality this kind of coarse dichotomy I think is more realistic. If p(nu $<$ 20) is high (say above 0.5) then there's a good chance the tails are heavy. Otherwise, we just don't know.}
% \label{default}
% \end{center}
% \end{figure}

%\begin{figure}[htbp]
%\begin{center}
%\includegraphics[width=0.7\textwidth]{p10-mammals-cross.pdf}
%\caption{Probability that nu $<$ 10 vs.\ log10(body mass) and log10(years at sexual maturity).}
%\label{default}
%\end{center}
%\end{figure}

\begin{figure}[htbp]
\begin{center}
\includegraphics[width=\textwidth]{ts-gpdd-heavy-eg-log10-no-fur.pdf}
\caption{Time series for populations with p($\nu < 10$) $>$ 0.5 using the base Gompertz population model. Panels are ordered by increasing p($\nu < 10$). Vertical axes are on a log10 scale. Colours indicate taxonomic classes. TODO add dots for upswings and downswings as in Table 1. The labels on each panel indicate p($\nu < 10$) $>$ 0.5, the common name for the species, and the GPDD ID number.}
\label{fig:heavy-ts}
\end{center}
\end{figure}

\begin{figure}[htbp]
\begin{center}
\includegraphics[width=0.9\textwidth]{t-dist-sampling-sim-prior-exp0point01.pdf}
\includegraphics[width=0.9\textwidth]{t-dist-sampling-sim-sigma-prior-exp0point01.pdf}
\caption{
  Testing the ability to estimate $\nu$ (top panels) and the scale parameter of the process error (bottom panels) for a given number of samples (columns) drawn from a distribution with a given true $\nu$ value (rows). The red lines indicate the true population value. When a small number of samples are drawn there may not be samples sufficiently far into the tails to recapture the true $\nu$ value; however, heavy tails are still distinguished from normal tails in most cases, even with only 25 or 50 samples. TODO switch to nu = black swan, 3, 5, and normal.
}
\label{fig:sim-nu}
\end{center}
\end{figure}

\clearpage

\begin{figure}[htbp]
\begin{center}
\includegraphics[width=1.1\textwidth]{sim-gompertz.pdf}
\caption{Simulation testing the Gompertz estimation model when the process error is chosen so that $\nu$ can be estimated close to the true value outside the full population model (``effective $\nu$'' within a CV of 0.5 of specified $\nu$).
  The simulation was run across population $\nu$ values (columns) and different scenarios (colours): (1) 100 time steps and no observation error, (2) 50 time steps and no observation error, (3) 50 time steps and observation error drawn from $\mathrm{Normal} (0, 0.3^2)$ but ignored, and (4) 50 time steps with observation error in which the quantity of observation error was assumed known.
  Dashed horizontal lines show the true population values;
  these true values were chosen to represent approximately the median values as estimated from the GPDD.
  Individual dots and lines represent a stochastic draws from the true population distribution and a model fitting.
  The panels show true values of $\nu$ =  3, 5, normal (very heavy, heavy, and not heavy tailed).
  The panels from top to bottom show $1/\nu$, process noise scale parameter $\sigma_\mathrm{proc}$, growth rate parameter $\lambda$, and the density dependence parameter $b$.
  Dots and line segments are medians and 80\% credible intervals of the posterior.}
\label{fig:sim-gompertz}
\end{center}
\end{figure}

\begin{figure}[htbp]
\begin{center}
\includegraphics[width=\textwidth]{check-sim-box.pdf}
\caption{The same simulations results as the previous figure but here represented as probability that $\nu < 10$ (i.e.\ the probability of heavy tails). Within each scenario the dots represent stochastic draws from the true population distributions.}
\label{fig:sim-prob}
\end{center}
\end{figure}

\clearpage

\begin{figure}[htbp]
\begin{center}
\includegraphics[width=0.9\textwidth]{gomp-vs-logistic-nu.pdf}
\caption{Gompertz vs.\ Ricker logistic $\nu$ (nu) estimates. Line segments show 50\% credible intervals; dots show median estimates. The vast majority of heavy-tailed estimates are similar for the two population models. In a few instances one model estimates heavy tails where the other does not. This happens more often for the Ricker-logistic than the Gompertz model.}
\label{fig:gomp-vs-ricker}
\end{center}
\end{figure}

\begin{figure}[htbp]
\begin{center}
\includegraphics[width=0.8\textwidth]{effect-of-ar1-on-nu.pdf}
\caption{The effect of assuming no autocorrelation of residuals (vertical axis) vs.\ estimating autocorrelation of residuals (horizontal axis). Shown are medians and 50\% credible intervals of the posterior. The colours indicate $\phi$ --- the parameter estimating the lag-1 correlation of the residuals. The diagonal line illustrates a one-to-one relationship. There are a few populations in which the heavy-tail estimate changes when autocorrelation is incorporated, but a systematic difference.}
\label{fig:ar1}
\end{center}
\end{figure}

\clearpage

\begin{figure}[htbp]
\begin{center}
\includegraphics[width=\textwidth]{priors-gomp-base.pdf} \caption{Probability
  density of the Bayesian priors for the base Gompertz model. Per capita growth
  rate at log(abundance) = 0: $\lambda \sim \mathrm{Normal}(0, 10^2)$; scale
  parameter of t-distribution process error: $\sigma_\mathrm{proc} \sim \mathrm{Half\mhyphen
    Cauchy} (5)$; t-distribution degrees of freedom parameter: $\nu \sim
  \mathrm{Exponential}(0.01)$. Not shown is $b$, the density dependence
  parameter: $b \sim \mathrm{Uniform}(-1, 2)$.}
  \label{fig:priors}
\end{center}
\end{figure}

\clearpage


\begin{figure}[htbp]
\begin{center}
\includegraphics[width=\textwidth]{log10-assumed.pdf} \caption{Populations in which the abundance was specified as or assumed to be recorded as log10 values. The numbers at the front of each panel label are the GPDD IDs.}
    \label{fig:log10-assumed}
\end{center}
\end{figure}



\begin{figure}[htbp]
\begin{center}
\includegraphics[width=\textwidth]{all-clean-ts-3.pdf} \caption{All filtered times series used in our analysis. The abundances are shown on a log10 vertical axis. Colours indicate taxonomic classes. Open black circles indicate abundance values that were interpolated. Closed black circles indicate single abundance values that were recorded as zero but were set to the next lowest observed abundance.}
    \label{fig:all-ts}
\end{center}
\end{figure}

\clearpage

\noindent
Example Stan code for heavy-tailed Gompertz model with AR1 correlated residuals without observation error.

\begin{verbatim}
data {
  int<lower=0> N;        // number of data rows
  vector[N] y;           // vector to hold ln abundance observations
  real<lower=0> nu_rate; // rate parameter for nu exponential prior
}
parameters {
  real lambda;                 // Gompertz growth rate parameter
  real<lower=-1, upper=2> b;   // Gompertz density dependence parameter
  real<lower=0> sigma_proc;    // process noise standard deviation
  real<lower=2> nu;            // t-distribution degrees of freedom
  real<lower=-1, upper=1> phi; // AR1 parameter
}
transformed parameters {
  vector[N] epsilon;     // error terms
  epsilon[1] <- 0;
  for (i in 2:N) {
    epsilon[i] <- y[i] - (lambda + b * y[i - 1])
                       - (phi * epsilon[i - 1]);
  }
}
model {
  // priors:
  nu ~ exponential(nu_rate);
  lambda ~ normal(0, 10);
  sigma_proc ~ cauchy(0, 2.5);
  phi ~ normal(0, 1);
  // data model:
  for (i in 2:N) {
    y[i] ~ student_t(nu,
                     lambda + b * y[i - 1]
                     + phi * epsilon[i - 1],
                     sigma_proc);
  }
}
\end{verbatim}



\end{document}

1 - there is evidence of heavy tails/black swans... in ecological time series... but they are rare

- taxonomically widespread
- we don't find compelling... intrinsic obvious biological reason
- commonly extrinsic extreme events and interactions - climate and severe winters, cycles of natural enemies (parasites + predators),
- as climate gets more variable, extremes more frequent and greater magnitude (and skewed), might expect more of these
- and as our time series get longer and censusing gets more accurate... more ability to detect these
- calls for policy that is be robust to the possibility of black swans
- lead with ecological application

grouse are hunted - shags have erupted

2 - method
- probabilistic
- you can get away with 'these kinds of mistakes if you allow for heavy-tailed process errors'
-  first with a population dynamics model
- para on why novel

hook - to what degree to extreme environmental events..
population dynamics can often be connected to
will be increase in...
and then, however it remains unclear the degree to which animal populations also exhibit (too neutral, something like suffer) from extreme events...

ecology letters

- argue for why we do what we're doing

some we were able to track down

- digg into Barbary macaque

decreasing median - order plot

4 rows one column

add little years on spark lines at beginning and end at bottom?

switch to nu = 20 (or 10?) in first figure
l
combine fig 2 and 3 into new fig 2

new fig3 is cross plots
b, lambda, sigma_proc, N,
log10length, log10Lifespan or min age at maturity,

add 'black swan' simulation in which single value chosen at deviation = 8 or something like that

new supp fig that shows cross plot of accounting or not for observation error

show boxes of misclassification in cross plot figure of various model estimates

line up black swans, say with birds in UK - NAO

%%%%%%%%%%%%%%%%%%%%%%%%%%%%%%%%%%%%%%%%

citation with good justification of gompertz and other references: \citep{herrandoprez2014}

The Gompertz was by far the most favoured model in \citet{brook2006} with the GPDD.

I could see an argument being made that we're just seeing them because of autocorrelation in the residuals. Autocorrelation plays a big role in \citet{inchausti2002} justifying why they think they're seeing increasing CV of time series in the GPDD with time and an increasing CV could create heavy tails. They just model the spectral properties and CV of the abundance time series --- not a population dynamics model. I tried estimating a coefficient for the first-order autocorrelation of the residuals. There is some autocorrelation it picks up sometimes, but interestingly this doesn't change things that much with respect to heavy tail results. Nothing systematic at least (Another important reference on spectral analysis and the GPDD: \citet{garcia-carreras2011}.)

little evidence for benefit of including annual climate or climate extremes with these data:
\citep{gregory2010}
and little evidence for allee effect with them

the natural logorithm version of the logistic is often convenient for parameter estimation
\citep{valpine2002} (and they cite a couple people too)

cite Limited evidence for the demographic Allee effect from numerous species across taxa
Stephen D. Gregory1,4, Corey J. A. Bradshaw2,3, Barry W. Brook2, and Franck Courchamp1
Read More: http://www.esajournals.org.proxy.lib.sfu.ca/doi/full/10.1890/09-1128.1

###

\begin{enumerate}

\item illustration of how the exponential prior slightly constrains the sampling of $\nu$\ldots illustrating what the prior probability of heavy tails is given uninformative data (about 5\% probability $\nu < 10$.

\item plots of the priors overlayed with estimates across populations for all parameters

\item cross plot of $\nu$ estimates from the models: no density dependence, add density dependence, add AR1, add assumed observation error

\item time series plots of all heavy-tailed populations

\item simulation testing of detecting nu given reduced sampling from true distribution; show repeated estimates of nu with different sample sizes, and show a few example samples with different observation window lengths and highlight the deviations that are beyond the 0.001 and 0.999 probabilities for a normal distribution.

\item simulation testing of the Gompertz model with process error deviations fixed to have an effective $\nu$ estimate at the true level --- this asks how well is the Gompertz model able to partition the various parameters in a precise and unbiased way given that we know the heavy tails are there (also include a scenario with a massive black swan to show how it behaves there)

\end{enumerate}

%Also, we may not be crazy seeing the heavier tails for heavier longer lived animals. There's some precedence for that in a paper that found more red noise (longer term fluctuations dominating the variance and therefore making the CV grow with time) in large-bodied animals. They attributed it to a mismatch between dynamics and the observation scale. They used a different life-history dataset too. But, there are other systematic differences with the body size-nu relationship that make this sketchy for us. E.g.\ taxonomic orders and interval frequency of data collection vary systematically and in clumps with body size (not shown in the below figure). And basically the only stuff with very \textit{low} probability of heavy tails are the the small ones with generation times less than a year and small collection intervals. It's basically a step function. We may not be able to tease that all apart... at least not in this paper.

black swans:

Pederson, N., Dyer, J.M., McEwan, R.W., Hessl, A.E., Mock, C.J., Orwig, D.A. et al. (in press). The legacy of episodic climatic events in shaping temperate, broadleaf forests. Ecol. Monogr., 10.1890/13– 1025.1.
Cohn, J.P. (2000). Saving the Salton Sea: researchers work to understand its problems and provide possible solutions. Bioscience, 50, 295–301.

Gelman 2006: "Large but finite values of A represent prior distributions which we call “weakly informative” because, even in the tail, they have a gentle slope (unlike, for example, a half-normal distribution) and can let the data dominate if the likelihood is strong in that region. "

TODO - see this paper: http://onlinelibrary.wiley.com/doi/10.1046/j.1365-2656.2003.00738.x/full Population-level mechanisms for reddened spectra in ecological time series H. Resit Akçakaya1,*, John M. Halley1,2 andPablo Inchausti1,3