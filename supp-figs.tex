\renewcommand{\thefigure}{S\arabic{figure}}
\setcounter{figure}{0}

\begin{figure}[htbp]
\begin{center}
\includegraphics[width=\textwidth]{log10-assumed.pdf} \caption{Populations in which the abundance was specified as or assumed to be recorded as log10 values. The numbers at the front of each panel label are the GPDD IDs.}
    \label{fig:log10-assumed}
\end{center}
\end{figure}

\begin{figure}[htbp]
\begin{center}
\includegraphics[width=\textwidth]{all-clean-ts-3.pdf} \caption{All filtered times series used in our analysis. The abundances are shown on a log10 vertical axis. Colours indicate taxonomic classes. Open black circles indicate abundance values that were interpolated. Closed black circles indicate single abundance values that were recorded as zero but were set to the next lowest observed abundance.}
    \label{fig:all-ts}
\end{center}
\end{figure}

\clearpage

\begin{figure}[htbp]
\begin{center}
\includegraphics[width=\textwidth]{priors-gomp-base.pdf} \caption{Probability
  density of the Bayesian priors for the base Gompertz model. From left to right: (1) per capita growth
  rate at $\ln$(abundance) = $0$: $\lambda \sim \mathrm{Normal}(0, 10^2)$; (2) scale
  parameter of t-distribution process noise: $\sigma \sim \mathrm{Half\mhyphen
    Cauchy} (2.5)$; (3) t-distribution degrees of freedom parameter: $\nu \sim
  \mathrm{Exponential}(0.01)$. (4) AR1 correlation coefficient of residuals: $\phi \sim \mathrm{Truncated \mhyphen Normal}(0, 1, \mathrm{min.} = -1, \mathrm{max.} = 1)$. Not shown is $b$, the density dependence
  parameter: $b \sim \mathrm{Uniform}(-1, 2)$.}
  \label{fig:priors}
\end{center}
\end{figure}

\clearpage

\begin{figure}[htbp]
\begin{center}
\includegraphics[width=0.8\textwidth]{t-dist-sampling-sim-prior-exp0point01.pdf}
\includegraphics[width=0.8\textwidth]{t-dist-sampling-sim-sigma-prior-exp0point01.pdf}
\caption{
  Testing the ability to estimate $\nu$ (top panels) and the scale parameter of the process error (bottom panels) for a given number of samples (columns) drawn from a distribution with a given true $\nu$ value (rows). The red lines indicate the true population value. When a small number of samples are drawn there may not be samples sufficiently far into the tails to recapture the true $\nu$ value; however, heavy tails are still distinguished from normal tails in most cases, even with only 25 or 50 samples. TODO switch to nu = black swan, 3, 5, and normal.
}
\label{fig:sim-nu}
\end{center}
\end{figure}

\clearpage

\begin{figure}[htbp]
\begin{center}
\includegraphics[width=\textwidth]{sim-gompertz.pdf}
\caption{Simulation testing the Gompertz estimation model when the process error draws are chosen so that $\nu$ can be estimated close to the true value outside the full population model (``effective $\nu$'' within a CV of 0.2 of specified $\nu$).
  The simulation was run across population $\nu$ values (columns) and different scenarios (colours): (1) 100 time steps and no observation error, (2) 50 time steps and no observation error, (3) 50 time steps and observation error drawn from $\mathrm{Normal} (0, 0.2^2)$ but ignored, and (4) 50 time steps with observation error in which the quantity of observation error was assumed known.
  Dashed horizontal lines show the true population values;
  these true values were chosen to represent approximately the median values as estimated from the GPDD.
  Individual dots and lines represent a stochastic draws from the true population distribution and a model fitting (median and 80\% credible intervals).
  %The panels show true values of $\nu$ =  3, 5, normal (very heavy, heavy, and not heavy tailed).
  The panels from top to bottom show $1/\nu$, process noise scale parameter $\sigma$, growth rate parameter $\lambda$, and the density dependence parameter $b$.}
\label{fig:sim-gompertz}
\end{center}
\end{figure}

\begin{figure}[htbp]
\begin{center}
\includegraphics[width=\textwidth]{sim-gompertz-boxplots.pdf}
\caption{The distribution of median estimates from the same simulation as described in Fig.~\ref{fig:sim-gompertz}. The boxes show the interquartile range. The whiskers extend to $1.5$ times the interquartile range and outliers are shown as dots.}
\label{fig:sim-gompertz-boxplots}
\end{center}
\end{figure}


\begin{figure}[htbp]
\begin{center}
\includegraphics[width=\textwidth]{check-sim-box.pdf}
\caption{The probability that $\nu < 10$ (i.e.\ the approximate probability of heavy tails) for the same simulation scenarios as shown in Fig.~\ref{fig:sim-gompertz} and Fig.~\ref{fig:sim-gompertz-boxplots}. Within each scenario the dots represent stochastic draws from the true population distributions combined with model fits.}
\label{fig:sim-prob}
\end{center}
\end{figure}


\begin{figure}[htbp]
\begin{center}
\includegraphics[width=0.9\textwidth]{ts-gpdd-heavy-eg-log10-base.pdf}
\caption{Time series for populations with p($\nu < 10$) $>$ 0.5 using the base Gompertz population model. Panels are ordered by increasing p($\nu < 10$). Vertical axes are on a log10 scale. Colours indicate taxonomic classes. TODO add dots for upswings and downswings as in Table 1. The labels on each panel indicate p($\nu < 10$) $>$ 0.5, the common name for the species, and the GPDD ID number.}
\label{fig:heavy-ts}
\end{center}
\end{figure}

\clearpage

\begin{figure}[htbp]
\begin{center}
\includegraphics[width=\textwidth]{gomp-comparison.pdf}
\caption{Estimates of $\nu$ from alternative models plotted against the base Gompertz model estimates of $\nu$. Shown are medians of the posterior (dots) and 50\% credible intervals (segments). The diagonal line indicates a one-to-one relationship. Different colours indicate various taxonomic classes. The grey-shaded regions indicate regions of disagreement if $\nu = 10$ is taken as a threshold of heavy-tailed dynamics. The Gompertz observation error model assumes a fixed standard deviation of observation error of $0.2$ on the log scale.}
\label{fig:alt}
\end{center}
\end{figure}

\clearpage


\begin{figure}[htbp]
\begin{center}
\includegraphics[width=0.7\textwidth]{admb-coefs}
\includegraphics[width=0.8\textwidth]{length-vs-prob-tails}
\caption{Potential correlates of heavy-tailed dynamics.
  (Top panels) Dots, thick lines, and thin lines represent fixed-effect estimates and their 50\% and 95\% confidence intervals for the covariates listed on the left: $\log \sigma$, $b$, $\lambda$, $\log$(lifespan), $\log$(data set length). All covariates were standardized by subtracting their mean and dividing by their standard deviation.
  The beta distribution model (left panel) uses p$(\nu < 10)$ as the response (probability of heavy tails).
  The binomial distribution model (right panel) uses p$(\nu < 10) > 0.5$ vs.\ p$(\nu < 10) \le 0.5$ as the response (heavy vs.\ not heavy tails).
  The models were fitted as mixed effects models with beta or binomial distributions, logit links, and multilevel intercepts for taxonomic class and order nested within class using the \texttt{glmmADMB} \texttt{R} package \citep{fournier2012,glmmadmb}.
  (Bottom panels) Model predictions for the strongest covariate --- time series length --- with all other predictors set to zero (their mean value when standardized). Dots represent observations for individual populations. Lines and shaded regions represent estimates and 50\% and 95\% confidence intervals.
}
    \label{fig:correlate-coefs}
\end{center}
\end{figure}

% \clearpage
%
% \begin{figure}[htbp]
% \begin{center}
% \includegraphics[width=0.8\textwidth]{length-vs-prob-tails}
% \caption{Model predictions. TODO}
%     \label{fig:length-predictions}
% \end{center}
% \end{figure}

\clearpage

\noindent
Example Stan code for a heavy-tailed Gompertz model with AR1 correlated residuals and a specified level of observation error:

\begin{spacing}{1.15}
\begin{footnotesize}
\begin{verbatim}
data {
  int<lower=3> N;              // number of observations
  vector[N] y;                 // vector to hold ln abundance observations
  real<lower=0> nu_rate;       // rate parameter for nu exponential prior
}
parameters {
  real lambda;                 // Gompertz growth rate parameter
  real<lower=-1, upper=2> b;   // Gompertz density dependence parameter
  real<lower=0> sigma_proc;    // process noise scale parameter
  real<lower=2> nu;            // t-distribution degrees of freedom
  real<lower=-1, upper=1> phi; // AR1 parameter
  vector[N] U;                 // unobserved states
  real<lower=0> sigma_obs;     // specified observation error SD
}
transformed parameters {
  vector[N] epsilon;           // error terms
  epsilon[1] <- 0;
  for (i in 2:N) {
    epsilon[i] <- U[i] - (lambda + b * U[i - 1])
                       - (phi * epsilon[i - 1]);
  }
}
model {
  // priors:
  nu ~ exponential(nu_rate);
  lambda ~ normal(0, 10);
  sigma_proc ~ cauchy(0, 2.5);
  phi ~ normal(0, 1);
  // data model:
  for (i in 2:N) {
    U[i] ~ student_t(nu,
                     lambda + b * U[i - 1]
                     + phi * epsilon[i - 1],
                     sigma_proc);
  }
  y ~ normal(U, sigma_obs);
}
\end{verbatim}
\end{footnotesize}

\clearpage

\noindent
The GPDD IDs used in our analysis:

\begin{footnotesize}
\noindent
{\tt
1 3 4 5 6 7 8 9 10 11 12 13 14 15 16 17 18 44 45 46 47 58 61 64 1149 1150 1153 1157 1159 1160 1162 1163 1165 1166 1168 1169 1170 1173 1174 1177 1179 1184 1185 1188 1189 1190 1195 1196 1197 1199 1200 1201 1202 1203 1204 1205 1206 1207 1217 1227 1228 1229 1233 1234 1235 1237 1238 1239 1240 1243 1244 1247 1342 1377 1522 1523 1524 1525 1534 1602 1613 1618 1633 1660 1663 1664 1667 1669 1670 1671 1674 1682 1683 1792 1826 1829 1830 1831 1865 1866 1868 1869 1870 1875 1876 1880 1881 1883 1885 1886 1887 1888 1893 1894 1927 1964 1965 1966 1968 1970 1971 1973 1974 1976 1981 1982 1983 1986 1987 1991 1992 1993 1994 1998 1999 2003 2004 2005 2006 2007 2012 2013 2015 2016 2017 2018 2019 2020 2024 2025 2026 2027 2028 2031 2032 2033 2034 2066 2721 2722 2726 2732 2735 2736 2757 2758 2759 2770 2771 2772 2774 2775 2777 2778 2781 2829 2844 2857 2867 2869 2887 2903 2915 2974 2976 2991 3001 3003 3017 3051 3056 3059 3068 3214 3216 3218 3233 3249 3251 3253 3260 3265 3283 3356 3358 3360 3378 3442 3466 3468 3470 3477 3482 3508 3521 3625 3627 3639 3664 3673 3676 3678 3680 3706 3708 3716 3774 3776 3784 3795 3799 3811 3827 3829 3838 3840 3853 3866 3882 5019 5020 5032 5034 5035 5039 6057 6144 6527 6528 6529 6530 6531 6532 6533 6534 6535 6536 6537 6539 6541 6542 6547 6548 6549 6550 6553 6554 6555 6556 6558 6560 6561 6562 6564 6565 6566 6567 6568 6569 6570 6581 6582 6583 6633 6673 6674 6675 6676 6677 6678 6681 6683 6684 6685 6686 6687 6688 6770 6865 6867 6868 6869 6870 6876 6882 6885 6889 6890 6902 6904 6917 6920 6921 6922 6939 6940 6973 7048 7052 7053 7054 7060 7061 7067 7088 7089 7091 7092 7093 7094 7098 7099 7101 7102 7115 7116 9191 9192 9194 9195 9196 9200 9211 9215 9216 9217 9218 9219 9220 9221 9222 9223 9224 9225 9232 9308 9309 9330 9331 9381 9382 9393 9436 9437 9438 9439 9440 9441 9442 9443 9444 9445 9446 9468 9469 9470 9472 9477 9486 9488 9489 9490 9491 9492 9500 9501 9502 9503 9506 9515 9517 9518 9519 9586 9587 9606 9611 9612 9639 9641 9642 9644 9646 9647 9648 9650 9652 9654 9655 9656 9657 9658 9659 9661 9662 9663 9665 9667 9668 9669 9672 9673 9674 9675 9676 9677 9678 9679 9680 9681 9682 9688 9689 9690 9691 9767 9768 9769 9770 9771 9772 9773 9774 9775 9776 9777 9778 9793 9794 9795 9796 9797 9799 9800 9801 9802 9803 9804 9805 9806 9807 9808 9809 9810 9811 9812 9813 9814 9815 9816 9817 9818 9819 9820 9821 9822 9823 9824 9825 9826 9827 9828 9829 9830 9831 9835 9836 9893 9894 9895 9896 9897 9898 9899 9900 9901 9902 9903 9904 9905 9907 9919 9921 9932 9933 9934 9936 9938 9948 9949 9950 9951 9953 9990 9991 9993 9994 9995 9997 9998 9999 10000 10001 10002 10005 10006 10007 10008 10009 10010 10011 10012 10013 10029 10030 10031 10036 10039 10040 10041 10042 10044 10045 10046 10047 10048 10049 10050 10051 10053 10054 10055 10060 10061 10063 10065 10070 10071 10085 10088 10089 10090 10092 10093 10094 10096 10097 10098 10099 10100 10101 10110 10111 10112 10113 10114 10117 10118 10120 10121 10122 10123 10124 10125 10127 10128 10131 10134 10136 10137 10139 10140 10141 10142 10143 10144 10145 10149 10153 10156 10158 10159 10160 10161 10162 10163 10164 10165 20527 20530 20532 20534 20535 20536 20537 20539 20540 20541 20542 20543 20544 20546 20547 20548 20549 20550 20551 20552 20553 20555 20577 20578 20579 20580 20581 20582 20583 20587 20626 20628 20634 20635 20636 20639 20649 20650 20651 20652 20653 20654 20655 20656 20657 20658 20659 20660 20662 20663
}
\end{footnotesize}
\end{spacing}
