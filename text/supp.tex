\documentclass[12pt]{article}
\usepackage{geometry}
\geometry{verbose, letterpaper, tmargin = 2.54cm, bmargin = 2.54cm,
  lmargin = 2.54cm, rmargin = 2.54cm}
\geometry{letterpaper}
\usepackage{graphicx}
\usepackage{amssymb}
\usepackage{amsmath}
\usepackage{amsfonts}
\usepackage{setspace}
\usepackage{booktabs}
\usepackage{ragged2e}
\usepackage{verbatim}
\usepackage[sectionbib]{chapterbib}
\usepackage{amsmath}
\usepackage{longtable}
\usepackage{multirow}
\usepackage{array}
\usepackage{tabu} % for spacing between rows in longtable
\usepackage{titlesec}
\usepackage{url}
\usepackage{mathtools}
%\usepackage{cite}
% \usepackage{citesupernumber}
% \usepackage{scicite}

\usepackage{lineno}
\usepackage{xcolor}
\renewcommand\linenumberfont{\normalfont\tiny\sffamily\color{gray}}
\modulolinenumbers[2]

% Linux Libertine:
\usepackage{textcomp}
\usepackage[sb]{libertine}
\usepackage[varqu,varl]{inconsolata}% sans serif typewriter
\usepackage[libertine,bigdelims,vvarbb]{newtxmath} % bb from STIX
\usepackage[cal=boondoxo]{mathalfa} % mathcal
%\useosf % osf for text, not math
\usepackage[supstfm=libertinesups,%
  supscaled=1.2,%
  raised=-.13em]{superiors}

\mathchardef\mhyphen="2D % math hyphen

\textheight 22.0cm

\usepackage[round,sectionbib]{natbib}
\bibpunct{(}{)}{;}{a}{}{;}

\setlength\parskip{0.10in}
\setlength\parindent{0in}

% Fix line numbering and align environment
% http://phaseportrait.blogspot.ca/2007/08/lineno-and-amsmath-compatibility.html
\newcommand*\patchAmsMathEnvironmentForLineno[1]{%
  \expandafter\let\csname old#1\expandafter\endcsname\csname #1\endcsname
  \expandafter\let\csname oldend#1\expandafter\endcsname\csname end#1\endcsname
  \renewenvironment{#1}%
     {\linenomath\csname old#1\endcsname}%
     {\csname oldend#1\endcsname\endlinenomath}}%
\newcommand*\patchBothAmsMathEnvironmentsForLineno[1]{%
  \patchAmsMathEnvironmentForLineno{#1}%
  \patchAmsMathEnvironmentForLineno{#1*}}%
\AtBeginDocument{%
\patchBothAmsMathEnvironmentsForLineno{equation}%
\patchBothAmsMathEnvironmentsForLineno{align}%
\patchBothAmsMathEnvironmentsForLineno{flalign}%
\patchBothAmsMathEnvironmentsForLineno{alignat}%
\patchBothAmsMathEnvironmentsForLineno{gather}%
\patchBothAmsMathEnvironmentsForLineno{multline}%
}

% remove numbers in front of sections:
\makeatletter
\renewcommand\@seccntformat[1]{}
\makeatother


\title{Supporting Information:\\Black-swan events in animal
  populations}

\author{
Sean C. Anderson$^{1*}$ \and
Trevor A. Branch$^2$ \and
Andrew B. Cooper$^3$ \and
Nicholas K. Dulvy$^1$
}
\date{}

\begin{document}

% \nocite{taleb2007}
% \nocite{sornette2009}
% \nocite{may2008}
% \nocite{newman2005}
% \nocite{harnik2012}
% \nocite{fey2015}
% \nocite{mangel1994}
% \nocite{ipcc2012}
% \nocite{keitt1998}
% \nocite{allen2001}
% \nocite{halley2002}
% \nocite{brook2006a}
% \nocite{knape2012}
% \nocite{SOM}
% \nocite{saucy1994}
% \nocite{stafford1971}
% \nocite{potts1980}
% \nocite{schindler2010}
% \nocite{carpenter2006}
% \nocite{nunez2012}
% \nocite{meehl2004}
% \nocite{katz2005}
% \nocite{kirby2009}
% \nocite{denny2009}
% \nocite{nrc2007}
% \nocite{carpenter2012}
% \nocite{caddy1996}
% \nocite{doak2008}
% \nocite{lindenmayer2010}
% \nocite{thompson2013}

\newcommand{\basePriorMean}{102}
\newcommand{\basePriorMedian}{71}
\newcommand{\basePriorProbHeavy}{7.7}
\newcommand{\medianTimeSteps}{26}
\newcommand{\meanTimeSteps}{30.2}
\newcommand{\minTimeSteps}{20}
\newcommand{\maxTimeSteps}{117}
\newcommand{\birdN}{191}
\newcommand{\insectsN}{182}
\newcommand{\mammalsN}{125}
\newcommand{\fishN}{108}
\newcommand{\birdNH}{14}
\newcommand{\insectsNH}{5}
\newcommand{\mammalsNH}{6}
\newcommand{\fishNH}{0}
\newcommand{\birdPH}{7}
\newcommand{\insectsPH}{3}
\newcommand{\mammalsPH}{5}
\newcommand{\fishPH}{0}
\newcommand{\NOrdersHeavy}{16}
\newcommand{\POrdersHeavy}{41}
\newcommand{\baseFiftyObsFiftySwitch}{8}
\newcommand{\baseSeventyFiveObsFiftySwitch}{2}
\newcommand{\totalHeavyFifty}{26}
\newcommand{\totalHeavySeventyFive}{17}
\newcommand{\baseFiftyObsFiftySwitchPerc}{31}
\newcommand{\baseSeventyFiveObsFiftySwitchPerc}{8}
\newcommand{\baseNuTenObsTenSwitch}{8}
\newcommand{\baseNuTen}{26}
\newcommand{\baseNuFiveObsTenSwitch}{2}
\newcommand{\pHeavyNThirty}{0.13}
\newcommand{\pHeavyNSixty}{0.20}
\newcommand{\pIncHeavyNThirtyNSixty}{1.6}
\newcommand{\obsErrorNuFivePerc}{75}
\newcommand{\modelsNoConvergeAROne}{1}
\newcommand{\modelsNoConvergeAROneHeavyBase}{0}
\newcommand{\percImputedPops}{17}
\newcommand{\percImputedPoints}{0.7}
\newcommand{\nuCoefPopN}{606}
\newcommand{\AvesRangePerc}{4--8}
\newcommand{\InsectaRangePerc}{2--3}
\newcommand{\MammaliaRangePerc}{4--6}
\newcommand{\OsteichthyesRangePerc}{0}
\newcommand{\overallMinPerc}{3}
\newcommand{\overallMaxPerc}{5}
\newcommand{\overallBasePerc}{4}
\newcommand{\NPops}{609}
\newcommand{\NOrders}{39}
\newcommand{\NClasses}{7}
\newcommand{\interpPointsPerc}{1}
\newcommand{\nBSUp}{8}
\newcommand{\nBSDown}{51}
\newcommand{\ratioBSDownToUp}{6.4}
\newcommand{\percBSDown}{86}
\newcommand{\crashUnderRange}{1.1--2}
\newcommand{\crashUnderMedian}{1.3}
\newcommand{\probDensSkewedForHeavyPops}{86}
\newcommand{\percNormPopsNotSkewed}{88}
 % R output

\maketitle

\textsuperscript{1}Earth to Ocean Research Group, Department of Biological
Sciences, Simon Fraser University, Burnaby BC, V5A 1S6, Canada

\textsuperscript{2}School of Aquatic and Fishery Sciences, University of
Washington, Box 355020, Seattle, WA 98195, USA

\textsuperscript{3}School of Resource and Environmental Management, Simon
Fraser University, Burnaby, BC, V5A 1S6, Canada

\textsuperscript{*}Corresponding author: Sean C. Anderson; E-mail: sandrsn@uw.edu; Present address: School of Aquatic and Fishery Sciences, Box 355020, University of Washington, Seattle, WA 98195, USA

\linenumbers
\onehalfspacing


%\begin{centering}
%\LARGE
%Supporting Information\\[1.0em]
%\end{centering}


\section{Data selection}

We applied the following data selection and quality-control rules to the
Global Population Dynamics Database (GPDD):

\begin{enumerate}

\item To remove populations with unreliable population indices that could be
  strongly confounded with economics and sampling effort, we removed all
  populations with a sampling protocol listed as \texttt{harvest} as well
  populations with the words \texttt{harvest} or \texttt{fur} in the cited
  reference title.

\item We removed all populations with uneven sampling intervals, i.e.\ we
  removed populations that did not have a constant difference between the
  ``decimal year begin'' and ``decimal year end'' columns.

\item We removed all populations rated as $< 2$ in the GPDD quality assessment
  (on a scale of $1$ to $5$, with $1$ being the lowest quality data)
  \citep{sibly2005, ziebarth2010}.

\item Populations with negative abundance values were removed. Of the
  populations that remained at the end of our other filtering rules, the
  remaining populations with negative abundances listed were all from time
  series that had been standardized by subtracting the mean and dividing by the
  standard deviation. We verified this by locating the original papers the
  datasets were extracted from: \citet{colebrook1978} for zooplankton and
  \citet{lindstrom1995} for grouse. Since the papers did not include the
  original mean time-series values we could not back transform these data
  points.

\item We filled in all missing time steps with \texttt{NA} values and imputed
  single missing values with the geometric mean of the previous and following
  values. We chose a geometric mean to be linear on the log scale that the
  Gompertz and Ricker-logistic models were fit on.

\item We filled in single recorded values of zero with the lowest non-zero
  value in the time series \citep{brook2006a}. This assumes that
  single values of zero result from abundance being low enough that censusing
  overlooked individuals that were actually present. We turned multiple zero
  values in a row into \texttt{NA} values. This implies that multiple zero
  values were either censusing errors or caused by emigration. Regardless, our
  population models were fit on a multiplicative (log) scale and so could not
  account for zero abundance. To avoid distorting the original data too
  strongly, we removed populations in which we filled in more than four zeros.

\item We removed all populations without at least four unique values
  \citep{brook2006a}.

\item We removed all populations with four or more identical values in a row
  since these suggest either recording error or extrapolation between two
  observations.

\item We then wrote an algorithm to find the longest unbroken window of
  abundance (no \texttt{NA}s) with at least $20$ time steps in each population
  time series. If there were any populations with multiple windows of identical
  length, we took the most recent window. This is a longer window than used in
  some previous analyses \citep{brook2006a}, but since our model
  attempts to capture the shape of the distribution tails, our model requires
  more data.

\item We removed GPDD Main IDs \texttt{20531} and \texttt{10139}, which we
  noticed were duplicates of \texttt{20579} (a heron population).
  \texttt{20579} contained additional years of data not present in
  \texttt{10139}. We removed a limited number of populations from class
  Angiospermopsida and Bacillariophyceae to focus the taxonomy in our analysis
  on animals. We also removed any populations with an \texttt{Unknown}
  taxonomic class.

\item Finally, we removed populations with the following GPDD Main IDs, which
  we discovered were data entry errors when verifying the populations with
  suspected black swans: \texttt{1207} because the 1957 data point was entered
  as 2 but should have been 27 \citep{kendeigh1982}, \texttt{6531} because the
  1978 data point was entered as 7 but should have been 47 \citep{minot1986},
  and \texttt{6566} because some of the data did not match the graph
  \citep{heessen1996}.

\end{enumerate}

\noindent
%We provide a supplemental figure of all the time series included in our
%analysis and indicate which values were interpolated (non-zero interpolations)
\percImputedPops\% of populations had at least one point interpolated but
only \percImputedPoints\% of the total observations were interpolated.
We note that interpolation is highly unlikely to lead to
black-swan detections, since black swans involve extreme increases or
decreases.

\section{Prior distributions for Gompertz models}

For the Gompertz model, our weakly informative priors (Fig.~\ref{fig:priors}) were:
\begin{align}
b &\sim \mathrm{Uniform}(-1, 2)\\
\lambda &\sim \mathrm{Normal}(0, 10^2)\\
\sigma &\sim \mathrm{Half\mhyphen Cauchy} (0, 2.5)\\
\nu &\sim \mathrm{Truncated\mhyphen Exponential}(0.01, \mathrm{min.} = 2).
\end{align}
Our prior on \(b\) was uninformative between values of \(-1\) and \(2\). We
would not expect values of \(b\) with levels of density dependence as low as
\(-1\) (very strong inverse density dependence), nor would we generally expect
values above \(1\). We allowed values of \(b\) above \(1\) to allow for
non-stationary time series of growth rates. The estimates of \(b\) were well
within these bounds. Our prior on \(\lambda\) was weakly informative within the
range of expected values for population growth and is similar to the default
priors suggested by \citet{gelman2008d} for intercepts of
regression models. Our half-Cauchy prior on \(\sigma\) follows
\citet{gelman2006c} and \citet{gelman2008d} and the specific scale parameter of
\(2.5\) is based on our expected range of the value in nature from previous
studies \citep{connors2014}. In our testing of a subsample of populations, our
parameter estimates were not qualitatively changed by switching to a uniform
prior on \(\sigma\), but the weakly informative half-Cauchy prior sped up chain
convergence.

To test if the prior on \(\nu\) influenced our estimate of black-swan dynamics,
we refit our models with weaker and stronger priors. Our base model used a
prior on \(\nu\) of Truncated-Exponential(0.01, min.\ = 2). For a weaker prior
we used Truncated-Exponential(0.005, min.\ = 2) and for a stronger prior we
used Truncated-Exponential(0.02, min.\ = 2) (Fig.~\ref{fig:priors}C). Note that
the base and weaker priors are relatively flat within the region of \(\nu <
20\), which is the region we are concerned about when categorizing
populations as heavy tailed. We considered using a Uniform(\(0, 1\))
prior on \(1/\nu\) as used in \citet{gelman2014}, but this
prior puts considerably more weight on lower (heavy-tailed) values of \(\nu\)
than the truncated exponential priors considered here,
and the results are therefore not shown.
Furthermore, we considered the prior suggested in \citet{juarez2010} of
$\nu \sim \mathrm{Gamma}(2, 0.1)$, again limited to $\nu > 2$.
Compared to our main truncated exponential prior,
this prior places a higher probability density at $\nu < 10$ ($0.25$ vs.\ $0.08$)
and has a mode at $\nu = 10$.
This prior raised the estimated value of $\nu$
for some of the populations identified as heavy-tailed
with our default prior,
but overall gave us similar inference
with a Spearman correlation of 0.90 between Pr($\nu < 10$)
for the two priors.

\section{Alternative population models}

\subsection{Autocorrelated residuals} We considered a version of the Gompertz
model in which an autoregressive parameter was fit to the process-noise
residuals:
\begin{align}
x_t &= \lambda + b x_{t-1} + \epsilon_t\\
\epsilon_t &\sim \mathrm{Student}\mhyphen t(\nu, \phi \epsilon_{t-1}, \sigma).
\end{align}
In addition to the parameters in the original Gompertz model, this model
estimates an additional parameter \(\phi\), which represents the correlation of
subsequent residuals. Based on the results of previous analyses with the
GPDD \citep{connors2014} and the chosen prior in previous
analyses \citep{thorson2014a}, we placed a moderately informative prior on
\(\phi\) that assumed the greatest probability density near zero with the
reduced possibility of \(\phi\) being near \(-1\) or \(1\). Specifically, we
chose \(\phi \sim \mathrm{truncated\mhyphen Normal}(0, 0.5, \mathrm{min.} = -1,
\mathrm{max.} = 1)\).

\subsection{Density independence} We fit a simplified version of the
Gompertz model in which the density dependence parameter \(b\) was fixed at
\(1\) (density independent). This is equivalent to fitting a random walk model
(with drift) to the \(\log\) abundances or assuming the growth rates are drawn
from a stationary distribution. The model was as follows:
\begin{align}
x_t &= \lambda + x_{t-1} + \epsilon_t\\
\epsilon &\sim \mathrm{Student}\mhyphen t(\nu, 0, \sigma).
\end{align}
We fit this model for three reasons: (1) it is computationally simpler and so
provides a check that our more complicated full Gompertz model was obtaining
reasonable estimates of \(\nu\), (2) it provides a test of whether density
dependence was affecting our perception of heavy tails, (3) it
matches how some previous authors have modelled heavy tails without accounting
for density dependence \citep{segura2013}.

\subsection{Observation error} Observation error can bias parameter
estimates \citep{knape2012} and is known to affect the ability to detect extreme
events \citep{ward2007}. In our main analysis, we fit a model that ignored
observation error. One way to account for observation error would be to fit a
full state-space model that simultaneously estimates the magnitude of process
noise and observation error. However, simultaneously estimating observation and
process noise is a challenging problem (e.g.\ because the observation and
process noise parameters tend to negatively covary in model fitting) and is
known to sometimes result in identifiability issues with the Gompertz
population model \citep{knape2008}. Furthermore, our model was applied to
hundreds of time series, often of short length (as few as \minTimeSteps\ time
steps) and our model estimates an additional parameter---the shape of the
process deviation tails---potentially making identifiability and computational
issues even greater. Therefore, we considered a version of the base Gompertz
model that allowed for a fixed level of observation error:
\begin{align}
U_t &= \lambda + b U_{t-1} + \epsilon_t\\
x_t &\sim \mathrm{Normal}(U_t, \sigma_\mathrm{obs}^2)\\
\epsilon_t &\sim \mathrm{Student}\mhyphen t(\nu, 0, \sigma_\mathrm{proc}),
\end{align}
where \(U\) represents the served state vector, \(\sigma_\mathrm{obs}\)
represents the standard deviation of observation error (on a log scale), and
\(\sigma_\mathrm{proc}\) represent the process noise scale parameter. We set
\(\sigma_\mathrm{obs}\) to \(0.2\), which represents a typical value used in
simulation analyses \citep{valpine2002, thorson2014b}.

\subsection{Ricker-logistic} We also fit a Ricker-logistic model:
\begin{align}
x_t &= x_{t-1} + r_{\mathrm{max}}\left(1 - \frac{N_{t-1}}{K}\right) + \epsilon_t\\
\epsilon_t &\sim \mathrm{Student}\mhyphen t(\nu, 0, \sigma),
\end{align}
where \(r_\mathrm{max}\) represents the theoretical maximum population growth
rate that is obtained when \(N_t = 0\). The parameter \(K\) represents the
carrying capacity and, as before, \(x_t\) represents the \(\log\) transformed
abundance at time \(t\). The Ricker-logistic model assumes a linear decrease in
population growth rate with increases in abundance. In contrast, the Gompertz
model assumes a linear decrease in population growth rate with increases in
\textit{log} abundance.

To fit the Ricker-logistic models, we chose a prior on \(K\) uniform between
zero and twice the maximum observed abundance (similar and less informative
than in \citet{delean2013}). We set the prior on
\(r_\mathrm{max}\) as uniform between 0 and 20 as in \citet{delean2013}.
We used the same priors on \(\nu\) and \(\sigma\) as in
the Gompertz model.


\section{Heavy-tailed Gompertz model simulations}

We performed two types of simulation testing.
First, we tested how easily the Student-$t$ distribution \(\nu\) parameter could
be recovered given different true values of \(\nu\) and different sample sizes.
Second, we tested the ability of the heavy-tailed Gompertz model to obtain
unbiased parameter estimates of \(\nu\) given that a set of process deviations
was provided in which the effective \(\nu\) value was close to the true \(\nu\)
value.

We separated our simulation into these two components to avoid confounding two
issues. (1) With smaller sample sizes, there may not be a stochastic draw from
the tails of a distribution. In that case, no model, no matter how perfect,
will be able to detect the shape of the tails. (2) Complex models may return
biased parameter estimates if there are conceptual, computational, or coding
errors. Our first simulation tested the first issue and our second simulation
tested the latter. In general, our simulations show that, if anything, our
model under predicts the magnitude and probability of heavy tailed
events---especially given the length of the time series in the GPDD.

First, we tested the ability to estimate $\nu$ given different true values of
$\nu$ and different sample sizes. We took stochastic draws from $t$
distributions with different $\nu$ values ($\nu = 3, 5, 10,$ and $10^6$
[$\approx$ normal]), with central tendency parameters of $0$, and scale
parameters of $1$. We started with $1600$ stochastic draws and then fit the
models again at the first $800, 400, 200, 100, 50,$ and $25$ draws. Each time
we recorded the posterior samples of $\nu$.

We found that we could consistently and precisely recover median posterior
estimates of \(\nu\) near the true value of \(\nu\) with large samples (\(\ge
200\)) (Fig.~\ref{fig:sim-nu} upper panels). At smaller samples we could
usually qualitatively distinguish heavy from not-heavy tails, but the model
tended to underestimate how heavy the tails were. At the same time, at smaller
sample sizes, the model tended to overestimate how large the scale parameter
was (Fig.~\ref{fig:sim-nu} lower panels).

Second, we tested the ability of the heavy-tailed Gompertz model to obtain
unbiased parameter estimates when the process noise was chosen so that
appropriate tail events were present. To generate these process deviations for
the $\nu = 3$ and $\nu = 5$ scenarios, we repeatedly drew proposed candidate
process deviations and estimated the central tendency, scale, and $\nu$ values
each time. We recorded when $\hat{\nu}$ (median of the posterior) was within
$0.2$ CVs (coefficient of variations) of the true $\nu$ value and used this
set of random seed values in our Gompertz simulation.

The following pseudo R code illustrates the procedure to generate process
noise with effective $\nu$ values within a CV of 0.2 of the specified value
(the actual code is available at
\url{https://github.com/seananderson/heavy-tails}):

\clearpage

\begin{footnotesize}
\begin{verbatim}
get_effective_nu_seeds <- function(nu_true = 5, cv = 0.2, N = 50, seed_N = 20) {
  # nu_true: The true nu value
  # cv:      The permitted effective nu coefficient of variation
  # N:       The length of time series
  # seed_N:  The number of seed values to generate
  seeds <- numeric(length = seed_N)
  seed_value <- 0
  for (i in seq_len(seed_N)) {
    nu_close <- FALSE
    while (!nu_close) {
      seed_value <- seed_value + 1
      set.seed(i)
      y <- rt(N, df = nu_true)
      sm <- rstan::stan(... # fit the Stan model here
      med_nu_hat <- median(rstan::extract(sm, pars = "nu")[[1]])
      if (med_nu_hat > (nu_true - cv) & med_nu_hat < (nu_true + cv)) {
        nu_close <- TRUE
        seeds[i] <- seed_value
      }
    }
  }
  seeds
}
nu_3_seeds_N50 <- get_effective_nu_seeds(nu_true = 3)
nu_5_seeds_N50 <- get_effective_nu_seeds(nu_true = 5)
\end{verbatim}
\end{footnotesize}

We then fit our Gompertz models to the simulated datasets with all parameters
(except $\nu$) set near the median values estimated in the GPDD. We repeated
this with $50$ and $100$ samples without observation error, $50$ samples with
observation error ($\sigma_\mathrm{obs} = 0.2$), and $50$ samples with the
same observation error and a Gompertz model that allowed for correctly
specified observation error magnitude. Our results indicate that the Gompertz
model can recapture the true value of $\nu$ when the process noise was chosen
so that appropriate tail events were present (Fig.~\ref{fig:sim-prob} upper panels).
The addition of observation error caused the
model to tend to underestimate the degree of heavy-tailedness. Fitting a model
with correctly specified observation error did not make substantial
improvements to model bias (Fig.~\ref{fig:sim-prob}).

We found that the Gompertz model can recapture
the true value of \(\nu\) when the process noise was chosen so that appropriate
tail events were present (Fig.~\ref{fig:sim-prob} upper panels). The addition
of observation error caused the model to tend to underestimate the degree of
heavy-tailedness. Fitting a model with correctly specified observation error
did not substantially reduce model bias (Fig.~\ref{fig:sim-prob} upper panels).

When converting the posterior distributions of $\nu$ into Pr($\nu < 10$), the
models distinguished heavy and not-heavy tails reasonably well (Fig.~\ref{fig:sim-prob} lower panels).
Without observation error, and using a
probability of $0.5$ as a threshold, the model correctly classified all
simulated systems with normally distributed process noise as not heavy tailed.
The model would have miscategorized only one of $40$ simulations at $\nu = 5$
across simulated populations with $50$ or $100$ time steps
(Fig.~\ref{fig:sim-prob}, scenarios 1 and 2 in lower row, second panel from left).
The model would have correctly categorized all cases where the process noise
was not heavy tailed (Fig.~\ref{fig:sim-prob} bottom-right panel) and all cases
where $\nu = 3$ and there was not observation error. With $0.2$ standard
deviations of observation error, the model still categorized
\obsErrorNuFivePerc\% of cases as heavy tailed when $\nu = 5$ and all but one
case when $\nu = 3$. Allowing for observation error made little improvement to
the detection of heavy tails. Therefore, we chose to focus on the simpler model
without observation error in the main text, particularly given that the true
magnitude of observation error was unknown in the empirical data.

\section{Implementation of the skewed Student-$t$ distribution in Stan}

The probability density for the skewed Student-$t$ distribution
\citep{fernandez1998} equates to the following probability density as
implemented in \citet{king2012}:

\begin{equation}
  \mathrm{Skewed\,Student}\mhyphen t(x) =
\begin{dcases}
\frac{2}{(\gamma + 1/\gamma)}
         \mathrm{Student}\mhyphen t(x \cdot \gamma, \nu, \mu \cdot \gamma, \sigma),& \text{for } x < 0\\
       \frac{2}{(\gamma + 1/\gamma)}
         \mathrm{Student}\mhyphen t(x / \gamma, \nu, \mu / \gamma, \sigma),& \text{for } x \ge 0.
\end{dcases}
\end{equation}

Here $\gamma$ represents the skewness parameter. The Student-$t$($\cdot$) portion
represents in order: the data point of interest, the heavy-tailedness
parameter, the location parameter, and the scale parameter as passed to the
standard Student-$t$ probability density. This translates to the log probability
density:

\begin{equation}
\mathrm{Skewed\,Student}\mhyphen t(x) =
\begin{dcases}
  \log (2 \gamma) - \log(\gamma^2 + 1) +
  \log \mathrm{Student}\mhyphen t(x \cdot \gamma, \nu, \mu \cdot \gamma, \sigma),& \text{for } x < 0\\
  \log (2 \gamma) - \log(\gamma^2 + 1) +
  \log \mathrm{Student}\mhyphen t(x / \gamma, \nu, \mu / \gamma, \sigma),& \text{for } x \ge 0.
\end{dcases}
\end{equation}

Since Stan only needs the probability density up to an additive
constant \citep{stan-manual2014}, we
can replace $\log(2 \gamma)$ in the above for $\log(\gamma)$. The final log
probability density for the skewed Student-$t$ distribution implemented in
Stan\footnote{Credit to Stan developer Bob Carpenter:
\url{https://groups.google.com/d/msg/stan-users/jeZQnQRUsDs/BeqhicWm1GEJ}}
looks like the following:

\begin{verbatim}
real skew_student_t_log(real y, real nu, real mu, real sigma, real skew) {
  real lp;
  lp <- log(skew) - log1p(square(skew));
  if (y < mu)
    return lp + student_t_log(y * skew, nu, mu * skew, sigma);
  else
    return lp + student_t_log(y / skew, nu, mu / skew, sigma);
}
\end{verbatim}

%\clearpage
\section{Silhouette image licenses}

Many of the silhouette images used in Figs 2, and 3 were obtained from
\texttt{phylopic.org} under Creative Commons licenses. We drew the
salmon in Fig.~2. The bird in these figures was obtained
from \texttt{phylopic.org} under a Creative Commons Attribution 3.0 Unported
license with credit to Jean-Raphaël Guillaumin {[}photography{]} and T.
Michael Keesey {[}vectorization{]}). The silhouettes in
Fig.~3 were obtained from the following sources (metadata
obtained with the help of the rphylopic R package,
\url{https://github.com/sckott/rphylopic}):

\LTcapwidth=\textwidth
\singlespacing
\begin{footnotesize}
\begin{longtable}{>{\RaggedRight}m{3.2cm}>{\RaggedRight}p{6.5cm}>{\RaggedRight}p{5.0cm}}
%\caption{Phylopic credits}\\
\toprule
% latex table generated in R 3.2.0 by xtable 1.7-4 package
Taxonomic order & Credit & License URL \\ 
  \midrule
Salmoniformes & Servien (vectorized by T. Michael Keesey) & \url{http://creativecommons.org/licenses/by-sa/3.0/} \\ 
  Gadiformes &  & \url{http://creativecommons.org/publicdomain/mark/1.0/} \\ 
  Perciformes & Ellen Edmonson and Hugh Chrisp (vectorized by T. Michael Keesey) & \url{http://creativecommons.org/publicdomain/mark/1.0/} \\ 
  Pleuronectiformes & Tony Ayling (vectorized by T. Michael Keesey) & \url{http://creativecommons.org/licenses/by-sa/3.0/} \\ 
  Lepidoptera & Curtis (modified by T. Michael Keesey) & \url{http://creativecommons.org/publicdomain/mark/1.0/} \\ 
  Rodentia & Mattia Menchetti & \url{http://creativecommons.org/publicdomain/zero/1.0/} \\ 
  Carnivora & Brian Gratwicke (photo) and T. Michael Keesey (vectorization) & \url{http://creativecommons.org/licenses/by/3.0/} \\ 
  Lagomorpha & Sarah Werning & \url{http://creativecommons.org/licenses/by/3.0/} \\ 
  Coleoptera & Crystal Maier & \url{http://creativecommons.org/licenses/by/3.0/} \\ 
  Odonata & Gareth Monger & \url{http://creativecommons.org/licenses/by/3.0/} \\ 
  Passeriformes & Michael Scroggie & \url{http://creativecommons.org/publicdomain/zero/1.0/} \\ 
  Anseriformes & Sharon Wegner-Larsen & \url{http://creativecommons.org/publicdomain/zero/1.0/} \\ 
  Artiodactyla & Jan A. Venter, Herbert H. T. Prins, David A. Balfour and Rob Slotow (vectorized by T. Michael Keesey) & \url{http://creativecommons.org/licenses/by/3.0/} \\ 
  Diptera & Gareth Monger & \url{http://creativecommons.org/licenses/by/3.0/} \\ 
  Charadriiformes & JJ Harrison (vectorized by T. Michael Keesey) & \url{http://creativecommons.org/licenses/by-sa/3.0/} \\ 
  Hemiptera & T. Michael Keesey & \url{http://creativecommons.org/publicdomain/zero/1.0/} \\ 
  Falconiformes & Liftarn & \url{http://creativecommons.org/licenses/by-sa/3.0/} \\ 
  Galliformes & Steven Traver & \url{http://creativecommons.org/publicdomain/zero/1.0/} \\ 
   \bottomrule

\label{tab:phylopic}
\end{longtable}
\end{footnotesize}
\onehalfspacing

\section{Code availability.} Data and code to reproduce our analysis are
available at \url{https://github.com/seananderson/heavy-tails} (currently
private).

%\bibliography{/Users/seananderson/Dropbox/tex/jshort,supp}

\bibliographystyle{ecologyletters}\bibliography{jshort,newbib}

\clearpage

\section{Supplementary tables and figures}

%\newenvironment{helvetica}{\fontfamily{phv}\selectfont}{\par}

\singlespacing

Table~S1: Summary statistics for the filtered Global Population
Dynamics Database time series arranged by taxonomic class. Columns are number
of populations, number of taxonomic orders, numbers of species, median time
series length, total number of interpolated time steps, total number of
substituted zeros, and total number of time steps.

\onehalfspacing

%\begin{helvetica}
\smallskip
\begin{scriptsize}
\begin{tabular}{lrrrrrrrr}
\toprule
% latex table generated in R 3.3.1 by xtable 1.8-2 package
Taxonomic class & Populations & Orders & Species & Median length & Interpolated pts & Zeros pts & Total pts \\ 
  \midrule
Aves & 191 &  15 & 112 & 27.00 &  68 &  32 & 6160 \\ 
  Insecta & 182 &   7 &  91 & 25.00 &  26 &  55 & 4812 \\ 
  Mammalia & 125 &   8 &  51 & 28.00 &  18 &  21 & 4027 \\ 
  Osteichthyes & 108 &   6 &  35 & 26.50 &  13 &   3 & 3310 \\ 
  Chondrichthyes &   1 &   1 &   1 & 20.00 &   1 &   0 &  20 \\ 
  Crustacea &   1 &   1 &   1 & 33.00 &   0 &   0 &  33 \\ 
  Gastropoda &   1 &   1 &   1 & 21.00 &   0 &   0 &  21 \\ 
   \bottomrule

\label{tab:stats}
\end{tabular}
\end{scriptsize}
%\end{helvetica}

\clearpage

\LTcapwidth=\textwidth

\renewcommand{\thefigure}{S\arabic{figure}}
\renewcommand{\thetable}{S\arabic{table}}


\renewcommand{\tablename}{Table}

\setcounter{table}{1}

\renewcommand{\arraystretch}{0.1}% Tighter

%\begin{helvetica}
\singlespacing
\begin{scriptsize}
\begin{longtable}{>{\RaggedRight}m{1.4cm}>{\RaggedRight}p{6.0cm}>{\RaggedRight}p{0.7cm}>{\RaggedRight}p{1.2cm}>{\RaggedRight}p{4.4cm}>{\RaggedRight}p{1.4cm}}
\caption{Populations with a high probability of heavy-tailed dynamics in the base heavy-tailed Gompertz population dynamics model. Shown are the log abundance time series, population descriptions, Global Population Dynamics Database Main IDs, citation for the data source or separate verification literature, a description of the cause of the black-swan events (if known), the probability of heavy tails as calculated by our model, and median estimate of $\nu$ from our model with 90\% quantile credible intervals indicated in parentheses. Red dots on the time series indicate downward black-swan events and blue values indicate upward black-swan events that have a $10^{-4}$ probability or less of occurring if the population dynamics were explained by a Gompertz model with normally distributed process noise with a standard deviation equal to the scale parameter in the fitted $t$ distribution.}\\

\toprule
% latex table generated in R 3.3.2 by xtable 1.8-2 package
Time series & Population & ID & Ref & Description & Pr($\nu < 10$) \\ 
  \midrule
\includegraphics[width=1.7cm]{../analysis/sparks/6528.pdf} & Shag, \textit{Phalacrocorax aristotelis}, Farne Islands, Northumberland & 6528 & \cite{potts1980} & Red tide event combined with low productivity due to overcrowding & 1.00 \\
  \includegraphics[width=1.7cm]{../analysis/sparks/7115.pdf} & South African fur seal, \textit{Arctocephalus pusillus}, South Africa & 7115 & \cite{shaughnessy1982} & Harvesting and predation changes & 1.00 \\
  \includegraphics[width=1.7cm]{../analysis/sparks/10127.pdf} & Red grouse, \textit{Lagopus lagopus scoticus}, Scotland - un-named area & 10127 & \cite{potts1984} & Environment- and parisite-caused cycles & 0.99 \\
  \includegraphics[width=1.7cm]{../analysis/sparks/9382.pdf} & Pine looper or Bordered white, \textit{Bupalus piniaria}, Kessock & 9382 & \cite{broekhuizen1993} & Unknown, but sampling intensity was decreasing & 1.00 \\
  \includegraphics[width=1.7cm]{../analysis/sparks/10128.pdf} & Red grouse, \textit{Lagopus lagopus scoticus}, Scotland - un-named area & 10128 & \cite{potts1984} & Environment- and parisite-caused cycles & 1.00 \\
  \includegraphics[width=1.7cm]{../analysis/sparks/10007.pdf} & Water vole, \textit{Arvicola terrestris}, Le Pont & 10007 & \cite{saucy1994} & Predator-environment cycle interactions & 1.00 \\
  \includegraphics[width=1.7cm]{../analysis/sparks/20579.pdf} & Grey heron, \textit{Ardea cinerea}, Southern Britain & 20579 & \cite{stafford1971} & Severe winter & 0.98 \\
  \includegraphics[width=1.7cm]{../analysis/sparks/9655.pdf} & Flea beetle, \textit{Chaetocnoma concinna}, Finland & 9655 & \cite{markkula1965} & Cannot locate original source & 0.99 \\
  \includegraphics[width=1.7cm]{../analysis/sparks/1235.pdf} & Wren, \textit{Troglodytes troglodytes}, Eastern Wood, Bookham Common & 1235 & \cite{newton1998} & Severe winter & 0.98 \\
  \includegraphics[width=1.7cm]{../analysis/sparks/9667.pdf} & Gooseberry sawfly, \textit{Nemastus ribesii}, Finland & 9667 & \cite{markkula1965} & Cannot locate original source & 0.97 \\
  \includegraphics[width=1.7cm]{../analysis/sparks/10113.pdf} & Willow grouse, \textit{Lagopus lagopus}, Northern England & 10113 & \cite{dobson1995} & Parasites and predators & 0.98 \\
  \includegraphics[width=1.7cm]{../analysis/sparks/9679.pdf} & Unknown, \textit{Trioza apicalis}, Finland & 9679 & \cite{markkula1965} & Cannot locate original source & 0.86 \\
  \includegraphics[width=1.7cm]{../analysis/sparks/20527.pdf} & Wandering albatross, \textit{Diomedea exulans}, Taiaroa & 20527 & \cite{robertson1998} & Unknown & 0.99 \\
  \includegraphics[width=1.7cm]{../analysis/sparks/10039.pdf} & Red grouse, \textit{Lagopus lagopus scoticus}, Northern Scotland & 10039 & \cite{dobson1995} & Parasites and predators & 0.93 \\
  \includegraphics[width=1.7cm]{../analysis/sparks/10162.pdf} & Red grouse, \textit{Lagopus lagopus scoticus}, Atholl Estate & 10162 & \cite{mackenzie1952} & Bad environmental conditions and overcrowding combined to create crashes & 0.92 \\
  \includegraphics[width=1.7cm]{../analysis/sparks/7099.pdf} & European rabbit, \textit{Oryctolagus cuniculus}, Estate 2, East Anglia & 7099 & \cite{barnes1986} & Disease outbreak followed by years of good weather & 0.78 \\
  \includegraphics[width=1.7cm]{../analysis/sparks/9503.pdf} & Fisher or  Pekan, \textit{Martes pennanti}, Manitoba & 9503 & \cite{keith1963} & Unknown & 0.76 \\
  \includegraphics[width=1.7cm]{../analysis/sparks/2778.pdf} & Wheatear, \textit{Oenanthe oenanthe}, Skokholm Island & 2778 & \cite{lack1969} & Unknown, but decline noted specifically, cold winters caused some crashes & 0.66 \\
  \includegraphics[width=1.7cm]{../analysis/sparks/9659.pdf} & Cabbage root fly or maggot, \textit{Delia radicum}, Finland & 9659 & \cite{markkula1965} & Cannot locate original source & 0.60 \\
   \bottomrule

\label{tab:causes-supp}
\end{longtable}
\end{scriptsize}
\onehalfspacing
%\end{helvetica}

\clearpage


\renewcommand{\figurename}{Figure}

\begin{figure}[htbp]
\begin{center}
\includegraphics[width=0.8\textwidth]{../analysis/gomp-comparison.pdf}

\caption{Estimates of $\nu$ from alternative models plotted against the
    base Gompertz model estimates of $\nu$. Alternative models are \textbf{(A)}
    Ricker-logistic; \textbf{(B)} a Gompertz model with AR1 residual correlation;
    \textbf{(C)} modelling population growth rate with no density dependence; and
    \textbf{(D)} a Gompertz model with fixed magnitude of observation error. Shown
are medians of the posterior (dots) and 50\% credible intervals (segments). The
diagonal line indicates a one-to-one relationship. Different colours indicate
various taxonomic classes. The grey-shaded regions indicate regions of
disagreement if $\nu = 10$ is taken as a threshold of heavy-tailed dynamics.
The Gompertz observation error model assumes a fixed standard deviation of
observation error of $0.2$ on a log scale. Models in \textbf{(B)} and \textbf{(D)}
that had not met convergence thresholds were run for four chains, 256,000 iterations (128,000 warm up), and retaining every 10th sample, after which the MCMC chains for one population in \textbf{(B)} and four populations in \textbf{(D)} had still not met convergence thresholds. These populations are excluded from their respective panels, but all populations with Pr($\nu < 10$) $> 0.5$ in the base model had converged and are shown.}

\label{fig:alt}
\end{center}
\end{figure}

\clearpage

\begin{figure}[htbp]
\begin{center}
\includegraphics[width=0.8\textwidth]{../analysis/gomp-prior-comparison.pdf}

\caption{Estimates of $\nu$ from Gompertz models with alternative priors on $\nu$. \textbf{(A)} A weaker prior on $\nu$; \textbf{(B)} a stronger prior on $\nu$, as illustrated in Fig.~1. Shown are medians of the posterior (dots) and 50\% credible intervals (segments). The diagonal line indicates a one-to-one relationship. Different colours indicate various taxonomic classes. The grey-shaded regions indicate regions of disagreement if $\nu = 10$ is taken as a threshold of heavy-tailed dynamics. In general, the estimates are nearly identical in cases where the data are informative about low values of $\nu$. When the data are less informative about low values of $\nu$, the prior can slightly pull the estimates of $\nu$ towards higher or lower values.}

\label{fig:alt-priors}
\end{center}
\end{figure}

\clearpage


\begin{figure}[htbp]
\begin{center}
\includegraphics[width=0.8\textwidth]{../analysis/t-dist-sampling-sim-prior-exp0point01.pdf}
\includegraphics[width=0.8\textwidth]{../analysis/t-dist-sampling-sim-sigma-prior-exp0point01.pdf}

\caption{Testing the ability to estimate $\nu$ (top panels) and the scale parameter of the process deviations (bottom panels) for a given number of samples (columns) drawn from a distribution with a given true $\nu$ value (rows). The red lines indicate the true population value. When a small number of samples are drawn there may not be samples sufficiently far into the tails to recapture the true $\nu$ value; however, heavy tails are still distinguished from normal tails in most cases, even with only 25 or 50 samples.}

\label{fig:sim-nu}
\end{center}
\end{figure}

\clearpage

\begin{figure}[htbp]
\begin{center}
\includegraphics[width=\textwidth]{../analysis/sim-gompertz-median-dist.pdf}
\includegraphics[width=\textwidth]{../analysis/sim-gompertz-p10.pdf}

\caption{Simulation testing the Gompertz estimation model when the process deviation draws were chosen so that $\nu$ could be estimated close to the true value outside the full population model (``effective $\nu$'' within a CV of 0.2 of specified $\nu$). Upper panels show the distribution of median $\widehat{\nu}$ across 20 simulation runs. Lower panels show the distribution of Pr($\nu < 10$) across 20 simulation runs. We ran the simulations across three population (``true'') $\nu$ values (3, 5, and $1\cdot 10^9$, i.e.\ approximately normal) and four scenarios: (1) 100 time steps and no observation error, (2) 50 time steps and no observation error, (3) 50 time steps and observation error drawn from $\mathrm{Normal} (0, 0.2^2)$ but ignored, and (4) 50 time steps with observation error in which the quantity of observation error was assumed known. Within each scenario the dots represent stochastic draws from the true population distributions combined with model fits. Underlayed boxplots show the median, interquartile range, and $1.5$ times the interquartile range.}

\label{fig:sim-prob}
\end{center}
\end{figure}

\begin{figure}[htbp]
\begin{center}
\includegraphics[width=0.8\textwidth]{../analysis/correlates-p10.pdf}

\caption{Potential covariates of heavy-tailed population dynamics (indicated by a high probability that $\nu < 10$). Shown are (\textbf{A} to \textbf{C}) parameters from the Gompertz heavy-tailed population model ($\sigma$, $\lambda$, $b$), \textbf{(D)} number of time steps, \textbf{(E)} body length, and \textbf{(F)} lifespan. For the Gompertz parameters, $\sigma$ refers to the scale parameter of the Student-$t$ process-noise distribution, $\lambda$ refers to the expected log abundance at the next time step at an abundance of one, $b$ refers to the density dependence parameter. Circles representing a few sharks, crustaceans, and gastropods are filled in white. Median and 90\% credible interval posterior predictions of a beta regression hierarchical model are shown in panels \textbf{(A)} and \textbf{(D)} where there was a high probability the slope coefficient was different from zero (Fig.~3B).}

\label{fig:correlates}
\end{center}
\end{figure}

\clearpage



\begin{figure}[htbp]
\begin{center}
\includegraphics[width=6in]{../analysis/skew-t-illustration}
\includegraphics[width=4in]{../analysis/skewness-vs-nu}

\caption{Skewness and heavy-tailed parameter posteriors. Top panel shows the shape of the skew-$t$ distribution at different combinations of skewness ($\gamma$) and heavy-tailedness ($\nu$). Lower panel shows the posterior medians and 50\% credible intervals for the two parameters. The horizontal red line indicates the median log(skewness) parameter across all populations.}

\label{fig:skew-nu}
\end{center}
\end{figure}

\begin{figure}[htbp]
\begin{center}
\includegraphics[width=1.1\textwidth]{../analysis/heavy-skew-projections-ggplot}
\caption{Projections of population abundance under skew-$t$ (red) and
normally distributed process error models (blue) for populations with
Pr($\nu<10$) $>0.5$. Numbers above panels indicate GPDD population IDs as
shown in Table~S2. Horizontal axes indicates year before last
recorded abundance and a projection five years into the future. For the
projected region, solid lines represents the median projection and
dashed lines represent lower 99\% credible intervals. Vertical axes are
log-distributed. Assuming normal process error tends to underestimate risk of
low abundance if the process error is in fact heavy-tailed.}

\label{fig:skew-projections}
\end{center}
\end{figure}

\begin{figure}[htbp]
\begin{center}
\includegraphics[width=0.6\textwidth]{../analysis/priors-gomp-base.pdf}

\caption{Probability density of the Bayesian priors for the Gompertz models. \textbf{(A)} Per capita growth rate at $\log$(abundance) = $0$: $\lambda \sim \mathrm{Normal}(0, 10^2)$. \textbf{(B)} Scale parameter of t-distribution process noise: $\sigma \sim \mathrm{Half\mhyphen Cauchy} (0, 2.5)$ \textbf{(C)} Student-$t$ distribution degrees of freedom parameter: $\nu \sim \mathrm{Truncated\mhyphen Exponential}(0.01, \mathrm{min.} = 2)$. \textbf{(D)} AR1 correlation coefficient of residuals: $\phi \sim \mathrm{Truncated \mhyphen Normal}(0, 0.5, \mathrm{min.} = -1, \mathrm{max.} = 1)$. Not shown is $b$, the density dependence parameter: $b \sim \mathrm{Uniform}(-1, 2)$. Panel \textbf{(C)} also shows two alternative priors: a weaker prior $\nu \sim \mathrm{Truncated\mhyphen Exponential}(0.005, \mathrm{min.} = 2)$, and a stronger prior $\nu \sim \mathrm{Truncated\mhyphen Exponential}(0.02, \mathrm{min.} = 2)$. The inset panel shows the same data but with a log-transformed x axis. Note that the base and weaker priors are relatively flat within the region of $\nu < 20$ with which we are concerned.}

\label{fig:priors}
\end{center}
\end{figure}





%
% ------------------------------
% Supplemental Tables
% ------------------------------

%\clearpage
%\renewcommand{\thetable}{S\arabic{table}}
%\setcounter{table}{0}

% ------------------------------
% Supplemental Figures
% ------------------------------

%\renewcommand{\thefigure}{S\arabic{figure}}
%\setcounter{figure}{0}

% \begin{centering}
% \clearpage
% \includegraphics[width=\textwidth]{../analysis/all-clean-ts-mammals.pdf}\\
% Figure~\ref{fig:all-ts} (mammals) continued on next page \ldots
%
% \clearpage
% \includegraphics[width=\textwidth]{../analysis/all-clean-ts-birds.pdf}\\
% Figure~\ref{fig:all-ts} (birds) continued on next page \ldots
%
% \clearpage
% \includegraphics[width=\textwidth]{../analysis/all-clean-ts-insects.pdf}\\
% Figure~\ref{fig:all-ts} (insects) continued on next page \ldots
%
% \end{centering}

% \begin{figure}[htbp]
% \begin{center}
% \includegraphics[width=\textwidth]{../analysis/all-clean-ts-fishes-others.pdf}
%
% \caption[All filtered time series used in our analysis.]{(fishes, crustaceans,
%   gastropods, sharks). All filtered time series used in our analysis. The
%   abundances are shown on a log10 vertical axis. Throughout this figure, red
%   dots indicate values that were interpolated and blue dots indicate values
%   that were recorded as zero but were set to the next lowest observed
%   abundance. Numbers before each species name are the GPDD Main ID numbers.}
%
% \label{fig:all-ts}
% \end{center}
% \end{figure}

%\clearpage

\clearpage

\noindent
Example Stan code for a heavy-tailed Gompertz model. The specific code for
the various models in our analysis is available at
\url{https://github.com/seananderson/heavy-tails} (currently private).

%\begin{spacing}{1.15}
%% \begin{footnotesize}
%% \begin{verbatim}
%% data {
%%   int<lower=3> N;              // number of observations
%%   vector[N] y;                 // vector to hold ln abundance observations
%%   real<lower=0> nu_rate;       // rate parameter for nu exponential prior
%% }
%% parameters {
%%   real lambda;                 // Gompertz growth rate parameter
%%   real<lower=-1, upper=2> b;   // Gompertz density dependence parameter
%%   real<lower=0> sigma_proc;    // process noise scale parameter
%%   real<lower=2> nu;            // t-distribution degrees of freedom
%%   real<lower=-1, upper=1> phi; // AR1 parameter
%%   vector[N] U;                 // unobserved states
%%   real<lower=0> sigma_obs;     // specified observation error SD
%% }
%% transformed parameters {
%%   vector[N] epsilon;           // error terms
%%   epsilon[1] <- 0;
%%   for (i in 2:N) {
%%     epsilon[i] <- U[i] - (lambda + b * U[i - 1])
%%                        - (phi * epsilon[i - 1]);
%%   }
%% }
%% model {
%%   // priors:
%%   nu ~ exponential(nu_rate);
%%   lambda ~ normal(0, 10);
%%   sigma_proc ~ cauchy(0, 2.5);
%%   phi ~ normal(0, 1);
%%   // data model:
%%   for (i in 2:N) {
%%     U[i] ~ student_t(nu,
%%                      lambda + b * U[i - 1]
%%                      + phi * epsilon[i - 1],
%%                      sigma_proc);
%%   }
%%   y ~ normal(U, sigma_obs);
%% }
%% \end{verbatim}
%% \end{footnotesize}
%%

\begin{footnotesize}
\begin{verbatim}
data {
  int<lower=1> N;              // number of observations
  vector[N] y;                 // vector to hold log abundance observations
  real<lower=0> nu_rate;       // rate parameter for nu exponential prior
}
parameters {
  real lambda;                 // Gompertz growth rate parameter
  real<lower=-1, upper=2> b;   // Gompertz density dependence parameter
  real<lower=0> sigma_proc;    // process noise scale parameter
  real<lower=2> nu;            // t-distribution degrees of freedom
}
model {
  nu ~ exponential(nu_rate);
  lambda ~ normal(0, 10);
  sigma_proc ~ cauchy(0, 2.5);
  for (i in 2:N) {
    y[i] ~ student_t(nu, lambda + b * y[i-1], sigma_proc);
  }
}
\end{verbatim}
\end{footnotesize}


\clearpage
\noindent
Stan code for the multilevel beta regression:
\begin{footnotesize}
\verbatiminput{../analysis/betareg4.stan}
\end{footnotesize}

\clearpage

\noindent
The GPDD IDs used in our analysis.

\begin{footnotesize}
%\renewcommand{\baselinestretch}{1.11}
\noindent
{\tt
1 3 4 5 6 7 8 9 10 11 12 13 14 15 16 17 18 44 45 46 47 58 61 64 1149 1150 1153 1157 1159 1160 1162 1163 1165 1166 1168 1169 1170 1173 1174 1177 1179 1184 1185 1188 1189 1190 1195 1196 1197 1199 1200 1201 1202 1203 1204 1205 1206 1217 1227 1228 1229 1233 1234 1235 1237 1238 1239 1240 1243 1244 1247 1342 1377 1522 1523 1524 1525 1534 1602 1613 1618 1633 1660 1663 1664 1667 1669 1670 1671 1674 1682 1683 1792 1826 1829 1830 1831 1865 1866 1868 1869 1870 1875 1876 1880 1881 1883 1885 1886 1887 1888 1893 1894 1927 1964 1965 1966 1968 1970 1971 1973 1974 1976 1981 1982 1983 1986 1987 1991 1992 1993 1994 1998 1999 2003 2004 2005 2006 2007 2012 2013 2015 2016 2017 2018 2019 2020 2024 2025 2026 2027 2028 2031 2032 2033 2034 2066 2721 2722 2726 2732 2735 2736 2757 2758 2759 2770 2771 2772 2774 2775 2777 2778 2781 2829 2844 2857 2867 2869 2887 2903 2915 2974 2976 2991 3001 3003 3017 3051 3056 3059 3068 3214 3216 3218 3233 3249 3251 3253 3260 3265 3283 3356 3358 3360 3378 3442 3466 3468 3470 3477 3482 3508 3521 3625 3627 3639 3664 3673 3676 3678 3680 3706 3708 3716 3774 3776 3784 3795 3799 3811 3827 3829 3838 3840 3853 3866 3882 5019 5020 5032 5034 5035 5039 6057 6144 6527 6528 6529 6530 6532 6533 6534 6535 6536 6537 6539 6541 6542 6547 6548 6549 6550 6553 6554 6555 6556 6558 6560 6561 6562 6564 6565 6567 6568 6569 6570 6581 6582 6583 6633 6673 6674 6675 6676 6677 6678 6681 6683 6684 6685 6686 6687 6688 6770 6865 6867 6868 6869 6870 6876 6882 6885 6889 6890 6902 6904 6917 6920 6921 6922 6939 6940 6973 7048 7052 7053 7054 7060 7061 7067 7088 7089 7091 7092 7093 7094 7098 7099 7101 7102 7115 7116 9191 9192 9194 9195 9196 9200 9211 9215 9216 9217 9218 9219 9220 9221 9222 9223 9224 9225 9232 9308 9309 9330 9331 9381 9382 9393 9436 9437 9438 9439 9440 9441 9442 9443 9444 9445 9446 9468 9469 9470 9472 9477 9486 9488 9489 9490 9491 9492 9500 9501 9502 9503 9506 9515 9517 9518 9519 9586 9587 9606 9611 9612 9639 9641 9642 9644 9646 9647 9648 9650 9652 9654 9655 9656 9657 9658 9659 9661 9662 9663 9665 9667 9668 9669 9672 9673 9674 9675 9676 9677 9678 9679 9680 9681 9682 9688 9689 9690 9691 9793 9794 9795 9796 9797 9835 9836 9893 9894 9895 9896 9897 9898 9899 9900 9901 9902 9903 9904 9905 9907 9919 9921 9932 9933 9934 9936 9938 9948 9949 9950 9951 9953 9990 9991 9993 9994 9995 9997 9998 9999 10000 10001 10002 10005 10006 10007 10008 10009 10010 10011 10012 10013 10029 10030 10031 10036 10039 10040 10041 10042 10044 10045 10046 10047 10048 10049 10050 10051 10053 10054 10055 10060 10061 10063 10065 10070 10071 10085 10088 10089 10090 10092 10093 10094 10096 10097 10098 10099 10100 10101 10110 10111 10112 10113 10114 10117 10118 10120 10121 10122 10123 10124 10125 10127 10128 10131 10134 10136 10137 10140 10141 10142 10143 10144 10145 10149 10153 10156 10158 10159 10160 10161 10162 10163 10164 10165 20527 20530 20532 20534 20535 20536 20537 20539 20540 20541 20542 20543 20544 20546 20547 20548 20549 20550 20551 20552 20553 20555 20577 20578 20579 20580 20581 20582 20583 20587 20626 20628 20634 20635 20636 20639 20649 20650 20651 20652 20653 20654 20655 20656 20657 20658 20659 20660 20662 20663
}
%\renewcommand{\baselinestretch}{\textstretch}
\end{footnotesize}
\normalsize
%\end{spacing}

\end{document}
