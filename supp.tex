%!TEX root = anderson-etal-blackswan-timeseries.tex

\begin{centering}
\LARGE
Supporting Material\\[1.0em]
\end{centering}

\section{Data selection}

We applied the following data selection and quality-control rules to the Global
Population Dynamics Database (GPDD):

\begin{enumerate}

\item To remove populations with unreliable population indices that could be
  strongly confounded with economics and sampling effort, we removed all
  populations with a sampling protocol listed as \texttt{harvest} as well
  populations with the words \texttt{harvest} or \texttt{fur} in the cited
  reference title.

\item We removed all populations with uneven sampling intervals. I.e.~we removed
  populations that didn't have a constant difference between the ``decimal year
  begin'' and ``decimal year end'' columns.

\item We removed all populations rated as $< 2$ in the GPDD quality assessment
  (on a scale of $1$ to $5$, with $1$ being the lowest quality data)
  \citep[following][]{sibly2005, ziebarth2010}

\item Populations with negative abundance values were assumed to be log values
  and transformed by taking $10$ to the power of the recorded abundance. In many
  cases this was noted for the population, but not in all cases. We inspected
  each of these \totalAssumedLog~time series to make sure our assumption made
  sense.

\item We filled in all missing time steps with \texttt{NA} values and imputed
  single missing values with the geometric mean of the previous and following
  values. We chose a geometric mean to be linear on the log scale that the
  Gompertz and Ricker-logistic models were fitted on.

\item We filled in single recorded values of zero with the lowest non-zero value
  in the time series \citep[following][]{brook2006a}. This assumes that single
  values of zero result from abundance being low enough that censusing missed
  present individuals. We turned multiple zero values in a row into \texttt{NA}
  values. This implies that multiple zero values were either censusing errors or
  caused by emigration. Regardless, our population models were fitted on
  a multiplicative (log) scale and so could not account for zero abundance.

\item We removed all populations with four or more identical values in a row
  since these suggest either recording error or extrapolation between two
  observations.

\item We removed all populations without at least four unique values
  \citep[following][]{brook2006a}.

\item We then wrote an algorithm to find the longest unbroken window of
  abundance (no \texttt{NA}s) with at least $20$ time steps in each population
  time series. If there were any populations with multiple windows of identical
  length, we took the most recent window. This is a longer window than used in
  some previous analyses \citep[e.g.][]{brook2006a}, but since our model
  attempts to capture the shape of the distribution tails, our model requires
  more data.

\item Finally, we removed GPDD Main ID \texttt{20531}, which we noticed was
  a duplicate of \texttt{10139} (a heron population).

\end{enumerate}

\noindent
We provide a supplemental figure of all the time series included in our analysis
and indicate which values were interpolated (\percImputedPops\% of populations
had at least one point interpolated and only \percImputedPoints\% of the total
observations were interpolated) (Fig.~\ref{fig:all-ts}). Table S1 shows the
final taxonomic breakdown and the number of populations with interpolated
values.

\section{Details on the heavy-tailed Gompertz probability model}

For the Gompertz model, our weakly-informative priors (Fig.~\ref{fig:priors}) were:
\begin{align*}
b &\sim \mathrm{Uniform}(-1, 2)\\
\lambda &\sim \mathrm{Normal}(0, 10^2)\\
\sigma &\sim \mathrm{Half\mhyphen Cauchy} (0, 2.5)\\
\nu &\sim \mathrm{Truncated\mhyphen Exponential}(0.01, \mathrm{min.} = 2). \end{align*}
Our prior on $b$ was uninformative between values of $-1$ and $2$. We would not
expect values of $b$ with levels of inverse density dependence as low as $-1$,
nor would we expect values above $1$. We allowed values of $b$ above $1$ to
allow for somewhat non-stationary time series of growth rates. The estimates of
$b$ were well within these bounds. Our prior on $\lambda$ was very weakly
informative within the range of expected values for population growth and is
similar to priors suggested by \citet{gelman2008d} for intercepts of regression
models. Our prior on $\sigma$ follows \citet{gelman2006c} and
\citet{gelman2008d} and is based on our expected range of the value in nature
from previous studies \citep[e.g.][]{connors2014}. In our testing of a subsample
of populations, our parameter estimates were not qualitatively changed by
switching to an uninformative uniform prior on $\sigma$, but the weakly
informative prior substantially sped up chain convergence.

Our prior on $\nu$ was based on \citet{fernandez1998}; they chose a more
informative exponential rate parameter of $0.1$. We chose a less informative
rate parameter of $0.01$ and truncated the distribution at $2$, since at $\nu
< 2$ the variance of the t distribution is undefined. This prior gives only
a $7.7$\% probability $\nu < 10$ but constrains the sampling sufficiently to
avoid wandering off towards infinity --- above approximately $\nu = 20$ the
t distribution is so similar to the normal distribution
(Fig.~\ref{fig:didactic}) that time series of the length considered here are
unlikely to be sufficiently informative about the precise value of $\nu$. In the
scenario where the data are not heavy tailed (e.g.~Fig.~\ref{fig:didactic}e, h)
the posterior will approximately match the prior (median $= 71$, mean $= 102$)
and not be flagged as likely heavy tailed using the metrics we used in our paper
(e.g.~Pr$(\nu < 10) > 0.5$).

We fitted our models with Stan 2.4.0 \citep{stan-manual2014}, and R 3.1.1
\citep{r2014}. We began with four chains and $2000$ iterations, discarding the
first $1000$ as warm up (i.e.~4000 total samples). If $\hat{R}$ (a measure of
chain convergence) was greater than $1.05$ for any parameter or the minimum
effective sample size, $n_\mathrm{eff}$, (a measure of the effective number of
uncorrelated samples) for any parameter was less than $200$, we doubled both the
total iterations and warm up period and sampled from the model again. These
thresholds are in excess of the minimums recommended by \citet{gelman2006a} of
$\hat{R} < 1.1$ and effective sample size $> 100$ for reliable point estimates
and confidence intervals. In the majority of cases our minimum thresholds were
greatly exceeded. We continued this procedure up to $8000$ iterations ($16000$
total samples) by which all chains were deemed to have sufficiently converged.

\section{Simulation testing the model}

We performed two types of simulation testing. First, we tested how easily the
Student-t distribution $\nu$ parameter could be recovered given different true
values of $\nu$ and sample sizes. Second, we tested the ability of the
heavy-tailed Gompertz model to obtain unbiased parameter estimates of $\nu$
given that a set of process deviations was provided in which the ``effective
$\nu$'' value was close to the true $\nu$ value.

We separated our simulation into these two components to avoid confounding two
issues. (1) With smaller sample sizes, there may not be a stochastic draw from
the tails of a distribution. In that case, no model, no matter how perfect
a model, will be able to detect the shape of the tails. (2) Models (particularly
complex models) may return biased parameter estimates if there are conceptual,
computational, or coding errors in the models. Our first simulation tested the
first issue and our second simulation tested the latter. In general, out
simulations show that, if anything, our model under predicts the magnitude and
frequency of heavy tailed events --- especially given the length of the time
series in the GPDD.

\subsection{Estimating $\nu$ simulations}

First, we tested the ability to estimate $\nu$ given different true values of
$\nu$ and sample sizes. We took stochastic draws from t distributions with
different $\nu$ values ($\nu = 3, 5, 10,$ and $10^6$ [$\approx$ normal]), with
central tendency parameters of $0$, and scale parameters of $1$. We started with
$1600$ stochastic draws and then fitted the models again at the first $800, 400,
200, 100, 50,$ and $25$ draws. Each time we recorded the posterior samples of
$\nu$.

We found that we could consistently and precisely recover median posterior
estimates of $\nu$ near the true value of $\nu$ with large samples ($\ge 200$)
(Fig.~\ref{fig:sim-gompertz} upper panels). At smaller samples the we could
still usually distinguish ``heavy'' from ``not-heavy'' tails, but the model
tended to underestimate how heavy the tails were. At the same time, when the
model underestimated how heavy-tailed the distribution was, it tended to
overestimate how large the scale parameter was (Fig.~\ref{fig:sim-gompertz}
lower panels).

\subsection{Heavy-tailed Gompertz model simulations}

In the second part of our simulation testing, we tested the ability of the
heavy-tailed Gompertz model to obtain unbiased parameter estimates given that
process deviations were provided in which our simpler model (in the previous
section) could accurately estimate the value of $\nu$. To generate these process
errors, we repeatedly generated candidate process deviations and estimated the
central tendency, scale, and $\nu$ values each time. We recorded when
$\hat{\nu}$ was within $0.2$ CVs (coefficient of variations) of the true $\nu$
value and used this set of random seed values in our Gompertz simulation. We
then fitted our Gompertz models to the simulated datasets with all parameters
(except $\nu$) set near the median values estimated in the GPDD. We repeated
this with $50$ and $100$ samples without observation error, $50$ samples with
observation error ($\sigma_\mathrm{obs} = 0.2$), and $50$ samples with the same
observation error and a Gompertz model that allowed for correctly specified
observation error magnitude.

Our results show that the more-complicated Gompertz model can still recapture
the true value of $\nu$ when the process noise was chosen so that appropriate
tail events were present (Figs~\ref{fig:sim-gompertz} and
\ref{fig:sim-gompertz-boxplots}, red and green symbols in the top rows).
Likewise, the other Gompertz parameters were estimated without any systematic
bias (Figs~\ref{fig:sim-gompertz} and \ref{fig:sim-gompertz-boxplots}, red and
green symbols). The addition of observation error caused the model to tend to
underestimate the degree of heavy-tailedness, overestimate the magnitude of
process noise, somewhat overestimate $\lambda$, and overestimate density
dependence (blue symbols in Figs~\ref{fig:sim-gompertz} and
\ref{fig:sim-gompertz-boxplots}). The overestimation of density dependence with
observation error is a known phenomenon \citep{knape2012}. Fitting a model with
correctly specified observation error made marginal improvements to model bias
(purple symbols in Figs~\ref{fig:sim-gompertz} and
\ref{fig:sim-gompertz-boxplots}).

When converting the posterior distributions of $\nu$ into Pr($\nu < 10$) the
models performed well. Without observation error, and using a probability of
$0.5$ as a threshold, the model would have miscategorized only one of $40$
simulations at $\nu = 5$ across a sample size of $50$ or $100$ data points
(Fig.~\ref{fig:sim-prob}, upper panels). The model would have correctly
categorized all cases where the process noise was not heavy tailed (indicated as
``$\nu =$ infinity'' in Fig.~\ref{fig:sim-prob}) and all cases where $\nu = 3$.
With $0.2$ standard deviations of observation error, the model still categorized
XX\% of cases as heavy tailed when $\nu = 5$ and all cases where $\nu = 3$.
Allowing for observation error made little improvement to the detection of heavy
tails (Fig.~\ref{fig:sim-prob}, lower-left vs.\ lower-right panel). Therefore,
we chose to focus on the simpler model without observation error in the main
text.

\section{Alternative population models}

We fit four alternative population models to the time-series data to check how
they would influence our conclusions. Our alternative models allowed for
autocorrelation in the residuals, assumed no density dependence, allowed for
observation error, or assumed a Ricker-logistic functional form. The ranges of
percentages of black swans by taxonomic class cited in the abstract are based on
lower and upper limits across our main Gompertz model and these four alternative
models.

\subsection{Autocorrelated residuals}

We considered a version of the Gompertz model in which an autoregressive
parameter was fitted to the process noise residuals:
\begin{align*}
x_t &= \lambda + b x_{t-1} + \epsilon_t\\
\epsilon_t &\sim \mathrm{Student\mhyphen t}(\nu, \phi \epsilon_{t-1}, \sigma).
\end{align*}
In addition to the parameters in the original Gompertz model, we estimate an
additional parameter $\phi$, which represents the relationship between
subsequent residuals. Based on the results of previous analyses with the GPDD
\citep[e.g.][]{connors2014} and the chosen priors in previous analyses
\citep[e.g.][]{thorson2014a} and to greatly speed up chain convergence when
running our model across all populations, we placed a weakly informative prior
on $\phi$ that assumed the greatest probability density near zero with the
reduced possibility of $\phi$ being near $-1$ or $1$. Specifically, we chose
$\phi \sim \mathrm{Truncated\mhyphen Normal}(0, 1, \mathrm{min.} = -1,
\mathrm{max.} = 1)$. The MCMC chains for a small number of these models (XX) did
not converge according to our criteria ($\widehat{R} < 1.05, n_\mathrm{eff}
> 200$) after 8000 iterations of four chains. We did not include these models in
Fig.~\ref{fig:alt}.

\subsection{Assumed density independence}\label{assumed-density-independence}

We fit a simplified version of the Gompertz model in which the density
dependence parameter $b$ was fixed at $1$ (density independent). This is
equivalent to fitting a random walk model (with drift) to the $\ln$ abundances
or assuming the growth rates are stationary. The model was as follows:

\begin{align*}
x_t &= \lambda + x_{t-1} + \epsilon_t\\
\epsilon &\sim \mathrm{Student\mhyphen t}(\nu, \epsilon, \sigma).
\end{align*}
We fit this model for three reasons: (1) it is computationally simpler and so
provides a check that our more complicated full Gompertz model was obtaining
ballpark reasonable estimates of $\nu$, (2) it provides a test of whether
density dependence was systematically affecting our perception of heavy tails,
(3) it matches how some previous authors have modelled heavy tails without
accounting for density dependence \citep{segura2013}.

\subsection{Assumed observation error}

Observation error can bias parameter estimates \citep[e.g.][]{knape2012} and is
known to affect the ability to detect extreme events \citep{ward2007}. However,
simultaneously estimating observation and process error in state-space is
a challenging problem and is known to result in identifiability issues with the
Gompertz population model \citep{knape2008}. Furthermore, our model estimates an
additional parameter --- the shape of the process error distribution tails ---
potentially making identifiability and computational issues even greater.
Therefore, we considered a version of the base Gompertz model that allowed for
a fixed level of observation error:
\begin{align*}
U_t &= \lambda + b U_{t-1} + \epsilon_t\\
x_t &\sim \mathrm{Normal}(U_t, \sigma_\mathrm{obs})\\
\epsilon_t &\sim \mathrm{Student\mhyphen t}(\nu, 0, \sigma_\mathrm{proc}),
\end{align*}
where $U$ represents the unobserved state vector, and $\sigma_\mathrm{obs}$
represents the standard deviation of observation error (on a log scale). We set
$\sigma_\mathrm{obs}$ to $0.2$, which represents the upper end of values often
used in simulation analyses \citep[e.g.][]{valpine2002, thorson2014b}.

\subsection{Ricker-logistic}

We also fitted a Ricker-logistic model:
\begin{align*}
x_t &= x_{t-1} + r_{\mathrm{max}}\left(1 - \frac{N_{t-1}}{K}\right) + \epsilon_t\\
\epsilon_t &\sim \mathrm{Student\mhyphen t}(\nu, 0, \sigma),
\end{align*}
where $r_\mathrm{max}$ represents the maximum population growth rate that is
obtained when $N$ (abundance) $= 0$. The parameter $K$ represents the carrying
capacity and, as before, $x_t$ represents the $\ln$ transformed abundance at
time $t$. The Ricker-logistic model assumes a linear decrease in population
growth rate ($x_t - x_{t-1}$) with increases in abundance ($N_t$). In contrast,
the Gompertz model assumes a linear decrease in population growth rate with
increases in $\ln$ abundance ($x_t$) (REF).

To fit the Ricker-logistic models, we chose a prior on $K$ uniform between zero
and twice the maximum observed abundance (\citet{clark2010} chose uniform
between zero and maximum observed, which is more informative). We set the prior
on $r_\mathrm{max}$ as uniform between 0 and 20 as in \citet{clark2010}. We used
the same priors on $\nu$ and $\sigma$ as in the Gompertz model.

\section{Modelling covariates of heavy-tailed dynamics}

We fitted a multilevel beta regression model to the predicted probability of
heavy tails, Pr($\nu < 10$), to investigate potential covariates of heavy-tailed
dynamics. The beta distribution is useful when the response data range on
a continuous scale between zero and one. We used a logit link function as is
typically used in logistic regression. The model was as follows:
\begin{align*}
\mathrm{Pr}(\nu_i < 0.5) &\sim \mathrm{Beta}(A_i, B_i)\\
\mu_i &= \mathrm{logit}^{-1}(\alpha
  + \alpha^\mathrm{class}_{j[i]}
  + \alpha^\mathrm{order}_{k[i]}
  + \alpha^\mathrm{species}_{l[i]}
  + X_i \beta),
  \: \text{for } i = 1, \dots, 617\\
A_i &= \phi_\mathrm{disp} \mu_i\\
B_i &= \phi_\mathrm{disp} (1 - \mu_i)\\
\alpha^\mathrm{class}_j &\sim
  \mathrm{Normal}(0, \sigma^2_{\alpha \; \mathrm{class}}),
  \: \text{for } j = 1, \dots, 6\\
\alpha^\mathrm{order}_k &\sim
  \mathrm{Normal}(0, \sigma^2_{\alpha \; \mathrm{order}}),
  \: \text{for } k = 1, \dots, 38\\
\alpha^\mathrm{species}_l &\sim
  \mathrm{Normal}(0, \sigma^2_{\alpha \; \mathrm{species}}),
  \: \text{for } l = 1, \dots, 301,\\
\end{align*}
where $A$ and $B$ represent the beta distribution shape parameters; $\mu_i$
represents the predicted value for population $i$, class $j$, order $k$, and
species $l$; $\phi_\mathrm{disp}$ represents the dispersion parameter; and $X_i$
represents a vector of predictors for population $i$ with associated $\beta$
parameters. The intercepts are allowed to vary from the overall intercept
$\alpha$ by taxonomic classes ($\alpha^\mathrm{class}_j$), taxonomic orders
($\alpha^\mathrm{order}_k$), and species ($\alpha^\mathrm{species}_l$) with
standard deviations $\sigma_{\alpha \; \mathrm{class}}$, $\sigma_{\alpha \;
  \mathrm{order}}$, and $\sigma_{\alpha \; \mathrm{species}}$. Where possible,
we also allowed for error distributions around the predictors by incorporating
the standard deviation of the posterior samples for the Gompertz parameters
$\lambda$, $b$, and $\sigma$ around the mean point value (not shown in the above
equation). We log transformed $\sigma$, time-series length, and lifespan to
match the way they are visually represented in Fig.~\ref{fig:correlates}. All
input variables were standardized by subtracting their mean and dividing by two
standard deviations \citep{gelman2008c} to make their coefficients comparable in
magnitude.

We incorporated weakly informative priors into our model: $\mathrm{Cauchy}(0,
10)$ on the global intercept $\alpha$, $\mathrm{Half\mhyphen Cauchy}(0, 2.5)$ on
all standard deviation parameters, $\mathrm{Half\mhyphen Cauchy}(0, 10)$ on the
dispersion parameter $\phi_\mathrm{disp}$, and $\mathrm{Cauchy}(0, 2.5)$ on all
other parameters \citep{gelman2006c, gelman2008d}. Compared to normal priors,
the Cauchy priors concentrate more probability density at reasonably parameter
values while allowing for higher probability density far into the tails, thereby
allowing the data to dominate the posterior more strongly if it disagrees with
the prior. Our conclusions were not qualitatively changed by using uniform
priors. We fitted our models with 5000 total iterations per chain, 2500 warm-up
iterations, four chains, and discarding every second sample to save memory. We
checked for chain convergence visually and with the same criteria as before
($\widehat{R} < 1.05$ and $n_\mathrm{eft} >200$ for all parameters).

To derive taxonomic-order-level estimates of the probability of heavy tails
accounting for time-series length (Fig \ref{fig:order-estimates}), we fitted
a separate multilevel model with the same structure but with only time-series
length as a predictor. (In this case, we did not want to control for intrinsic
population characteristics such as density dependence.) We obtained order-level
estimates by adding the posteriors for $\alpha$, $\alpha^\mathrm{class}_j$, and
$\alpha^\mathrm{order}_k$.

\bibliographystyle{ecologyletters}
\bibliography{/Users/seananderson/Dropbox/tex/jshort,/Users/seananderson/Dropbox/tex/ref3}
