\documentclass[12pt]{article}
\usepackage{geometry}
\geometry{verbose,letterpaper,tmargin=2.54cm,bmargin=2.54cm,lmargin=2.54cm,rmargin=2.54cm}
\geometry{letterpaper}
\usepackage{graphicx}
\usepackage{amssymb}
\usepackage{amsmath}
\usepackage{amsfonts}
\usepackage{setspace}
\usepackage{booktabs}
\usepackage{lineno}
\usepackage{ragged2e}

\usepackage{longtable}
\usepackage{multirow}
\usepackage{array}
\usepackage{tabu} % for spacing between rows in longtable
\setlength{\tabulinesep}{9pt}

 % Linux Libertine:
  \usepackage{textcomp}
     \usepackage[sb]{libertine}
     \usepackage[varqu,varl]{inconsolata}% sans serif typewriter
     \usepackage[libertine,bigdelims,vvarbb]{newtxmath} % bb from STIX
     \usepackage[cal=boondoxo]{mathalfa} % mathcal
     \useosf % osf for text, not math
     \usepackage[supstfm=libertinesups,%
       supscaled=1.2,%
       raised=-.13em]{superiors}


\mathchardef\mhyphen="2D

\textheight 22.0cm

\usepackage[round]{natbib}
\bibliographystyle{apalike}
\bibpunct{(}{)}{;}{a}{}{;}

\title{Black swans in ecological time series\\Black swans in population dynamics\\Heavy-tailed ecological time series\\Black swans in ecological dynamics}
\author{
Sean C. Anderson$^{1\ast}$ \and
Trevor A. Branch$^2$ \and
Andrew B. Cooper$^3$ \and
Nicholas K. Dulvy$^1$
}
\date{}

% remove numbers in front of sections:
\makeatletter
\renewcommand\@seccntformat[1]{}
\makeatother

\begin{document}
\newcommand{\basePriorMean}{102}
\newcommand{\basePriorMedian}{71}
\newcommand{\basePriorProbHeavy}{7.7}
\newcommand{\medianTimeSteps}{26}
\newcommand{\meanTimeSteps}{30.2}
\newcommand{\minTimeSteps}{20}
\newcommand{\maxTimeSteps}{117}
\newcommand{\birdN}{191}
\newcommand{\insectsN}{182}
\newcommand{\mammalsN}{125}
\newcommand{\fishN}{108}
\newcommand{\birdNH}{14}
\newcommand{\insectsNH}{5}
\newcommand{\mammalsNH}{6}
\newcommand{\fishNH}{0}
\newcommand{\birdPH}{7}
\newcommand{\insectsPH}{3}
\newcommand{\mammalsPH}{5}
\newcommand{\fishPH}{0}
\newcommand{\NOrdersHeavy}{16}
\newcommand{\POrdersHeavy}{41}
\newcommand{\baseFiftyObsFiftySwitch}{8}
\newcommand{\baseSeventyFiveObsFiftySwitch}{2}
\newcommand{\totalHeavyFifty}{26}
\newcommand{\totalHeavySeventyFive}{17}
\newcommand{\baseFiftyObsFiftySwitchPerc}{31}
\newcommand{\baseSeventyFiveObsFiftySwitchPerc}{8}
\newcommand{\baseNuTenObsTenSwitch}{8}
\newcommand{\baseNuTen}{26}
\newcommand{\baseNuFiveObsTenSwitch}{2}
\newcommand{\pHeavyNThirty}{0.13}
\newcommand{\pHeavyNSixty}{0.20}
\newcommand{\pIncHeavyNThirtyNSixty}{1.6}
\newcommand{\obsErrorNuFivePerc}{75}
\newcommand{\modelsNoConvergeAROne}{1}
\newcommand{\modelsNoConvergeAROneHeavyBase}{0}
\newcommand{\percImputedPops}{17}
\newcommand{\percImputedPoints}{0.7}
\newcommand{\nuCoefPopN}{606}
\newcommand{\AvesRangePerc}{4--8}
\newcommand{\InsectaRangePerc}{2--3}
\newcommand{\MammaliaRangePerc}{4--6}
\newcommand{\OsteichthyesRangePerc}{0}
\newcommand{\overallMinPerc}{3}
\newcommand{\overallMaxPerc}{5}
\newcommand{\overallBasePerc}{4}
\newcommand{\NPops}{609}
\newcommand{\NOrders}{39}
\newcommand{\NClasses}{7}
\newcommand{\interpPointsPerc}{1}
\newcommand{\nBSUp}{8}
\newcommand{\nBSDown}{51}
\newcommand{\ratioBSDownToUp}{6.4}
\newcommand{\percBSDown}{86}
\newcommand{\crashUnderRange}{1.1--2}
\newcommand{\crashUnderMedian}{1.3}
\newcommand{\probDensSkewedForHeavyPops}{86}
\newcommand{\percNormPopsNotSkewed}{88}
 % R output

%\begin{spacing}{1.4}
%\doublespacing
\onehalfspacing

\maketitle
\thispagestyle{empty}

\noindent
\textit{Authors are listed alphabetically (ABCD!) and open to reordering.}

\noindent
\textsuperscript{1}Earth to ocean research group, department of biological sciences, Simon Fraser University, Burnaby BC, V5A 1S6, Canada
\noindent
\textsuperscript{2}School of Aquatic and Fishery Sciences, University of Washington, Box 355020, Seattle, WA 98195, USA

\noindent
\textsuperscript{3}School of Resource and Environmental Management, Simon Fraser University, Burnaby, BC, V5A 1S6, Canada

\noindent
\textsuperscript{*}Corresponding author: Sean C. Anderson; Earth to Ocean Research Group, Department of Biological Sciences, Simon Fraser University, Burnaby BC, V5A 1S6; Phone: 1-778-782-3989; E-mail: sean\_anderson@sfu.ca

\clearpage

\setcounter{page}{1}

\noindent
\textit{Main message, to be deleted (25 words)}: Black swans are present \& taxonomically widespread but rare in population dynamics.
Extreme climate, predation, parasites, and their interactions are common causes; intrinsic drivers are unclear.

%When they do occur they tend to be driven by ... little evidence....

\section{Abstract}

Black swans are statistically improbable events that have profound implications when they occur. Such events are known to occur in financial, social, and environmental systems; the magnitude of environmental extremes is projected to increase with climate change. But, the prevalence and magnitude of black swans in ecological population dynamics is unknown. Here, we develop a probability model that allows us to estimate the degree of heavy-tailedness in ecological process noise. We apply our model to \NPops\ time series from around the world across \NOrders\ taxonomic orders and seven classes. We find strong evidence of black swans, but they are rare, occurring in \overallMinPerc--\overallMaxPerc\% of populations and most frequently for birds (\AvesRangePerc\%) followed by mammals (\MammaliaRangePerc\%), insects (\InsectaRangePerc\%), and fishes (\OsteichthyesRangePerc\%). When they occur, they tend to be driven by climate and severe winters, cycles of natural enemies (parasites and predators) and interactions between these elements. We find little evidence of intrinsic life-history correlates.
Our results suggest [FOR ECOLOGICAL MODELLING]
and the importance of establishing management strategies that are robust to extreme events \ldots.

%lso mention the advance of our method: probabilistic in quantifying heavy tailedness, incorporates population dynamics, can allow for autocorrelation and observation error, and incorporate prior information

%- conclude with ecological, modelling, policy, and management implications

\section{Introduction}

\begin{enumerate}
\item Black swans are statistically improbable events whose occurrence has major ramifications. One of the most profound black swans in ecology was the asteroid marking the mass extinctions at the K-T boundary. Today, it is the extremeness of climate that is expected to cause the greatest societal damage. While extremes in the physical environment are present and widely understood, it remains unclear the extent to which ecological systems buffer or suffer from black swans.

\item What is the existing evidence for heavy tails in ecological time series?

\item There are two main possible reasons for why we may find little evidence of ecological black swans. (1) don't exist, (2) a paucity of appropriate methods or length/quality of time series.

\item Here we develop \ldots our research questions and main conclusions.

\end{enumerate}

%Increasing realization about the importance of extreme events in the environment (REFs)
%and of the importance of ecological surprises and black swans (REFs)

\bigskip
Papers to work in:

\citep{inchausti2002,halley2002,inchausti2001}

\citep{jentsch2007}

\citep{ward2007}

\citep{garcia-carreras2011}
\citep{sornette2009}

\citep{nunez2012}

\citep{thompson2013}
\citep{beaugrand2012}
\citep{pine-iii2009}

\citep{doak2008}

\citep{smale2013}

\citep{easterling2000}
\citep{scheffer2003}
\citep{katz2005}

\citep{taleb2007}

\citep{vasseur2014}

\citep{vert-pre2013}
\citep{lindenmayer2010}

\citep{valpine2002}
\citep{gregory2010}
\citet{garcia-carreras2011}
\citet{brook2006}
\citep{herrandoprez2014}

\citep{sibly2005, ziebarth2010}

\ldots

\section{Methods}

\begin{enumerate}
  \item Short overview paragraph
  \item The data: global population dynamics database; briefly introduce the GPDD and how we filtered it. Expand in the supplement. Mention the breakdown of taxonomy and interpolation (Table~S1).
  \item The heavy-tailed Gompertz population model. Mention the simulation testing but don't go into details here.
  \item Mention the alternative population models we fit.
  \item Modelling the possible correlates (mixed effects models with binomial or beta distributions and logit links; nested random effects for taxonomic class and taxonomic order). Life-history data from \citet{brook2006a}.
  \end{enumerate}


The Student-t Gompertz model is:
\begin{align*}
x_t &= \ln N_t\\
x_t &= \lambda + b x_{t-1} + \epsilon_t\\
\epsilon_t &\sim \mathrm{Student\mhyphen t}(\nu, 0, \sigma),
\end{align*}

\noindent where $x_t$ is the $\ln$ abundance at time $t$. The model is density independent if $b = 1$, maximally density dependent if $b = 0$, and inversely density dependence if $b < 0$. The parameter $\lambda$ represents the expected population growth rate at $x_t = 0$. The process noise, $\epsilon_t$, is modelled as a Student-t distribution that is centered at $0$, has a scale parameter of $\sigma_\mathrm{proc}$, and has a degrees of freedom parameter of $\nu$. If $\nu$ is small ($\lesssim 10$) the distribution has much heavier tails than a normal distribution.  For example, at $\nu = 2$, the probability of drawing a value less than -5 is 1.8\%, whereas the probability of drawing such a value from a normal distribution is nearly zero ($2.9\cdot10^{-5}$\%). As $\nu$ approaches infinity the distribution approaches the normal distribution (Fig~\ref{fig:didactic}). By estimating the value of $\nu$ we can quantity how heavy-tailed the process noise deviations are.

We fit alternative models that allowed for autocorrelation of the residuals, allowed for observation error, and allowed the functional form of the population dynamics to be represented as Ricker-logistic (Supporting Material).

%Mention that we simulation tested the model.

%Mention the priors (Fig.~\ref{fig:priors}) and expand on their justification in the supplement.

%$\nu$ prior: \citep{fernandez1998}
%$\sigma$ prior: \citep{gelman2006c}
%consider alternative inverse prior on $\nu$ as in \citep{gelman2014}



%justify why we only present these as hypotheses but do not formally model them

\bigskip
Supporting material to be noted:

\begin{enumerate}

  \item Table showing the taxonomic breakdown, number of interpolated points, etc (Table~S1).

  \item Plot showing time series we assumed were recorded as log transformed  (Fig.~\ref{fig:log10-assumed}).

  \item Plot showing all the time series we used with interpolated and zero imputed values (Fig.~\ref{fig:all-ts}).

  \item Plot showing the priors (Fig.~\ref{fig:priors}).
\end{enumerate}

\section{Results}

Main figures and tables:

\begin{enumerate}
  \item Fig.~\ref{fig:didactic} illustrates the method: it shows the t-distribution tails, example simulated time series, and model fits to those time series.

  \item Fig.~\ref{fig:nu-coefs} shows the posterior distributions of the estimated $\nu$ values. These are split by taxonomic class and order.

  \item Fig.~\ref{fig:correlates} shows possible biological and time-series-property correlates of heavy-tailed behaviour.

  \item Table.~\ref{tab:sparks} selects populations that were categorized as heavy-tailed and digs into the causes. These are a non-random sample that I was able to verify in the literature.

\end{enumerate}


%- quantify how frequent and strong heavy tails were (Fig.~\ref{fig:nu-coefs})

Supporting material to be noted quickly:

\begin{enumerate}

  \item Simulation testing: the ability to recover $\nu$ when randomly sampling from various distributions (Fig.~\ref{fig:sim-nu}); heavy-tailed Gompertz model performance and confidence interval coverage given the process noise has deviations that are known to have effective $\nu$ equal to true $\nu$ (Fig.~\ref{fig:sim-gompertz}); boxplots of the same output (Fig.~\ref{fig:sim-gompertz-boxplots}); probability that $\nu < 10$ for the Gompertz simulation testing (Fig.~\ref{fig:sim-prob}).

  \item Time series of all heavy tailed populations (Fig.~\ref{fig:heavy-ts}).

  \item The effect of alternative population dynamics models on $\nu$ estimates (Fig.~\ref{fig:alt}).

  \item Coefficients from modelling covariates of the probability of heavy tails (Fig.~\ref{fig:correlate-coefs}).

  \item Example Stan code for heavy-tailed Gompertz model with AR1 residuals and observation error (a model including everything).

  \item GPDD IDs used.

\end{enumerate}

Modelling result: an increase of 1 time step of data (given that you are ballpark around the mean --- 30 time steps --- to start with) results in an approximately 1\% increase in the expected probability of observing heavy tails. (Using Gelman's `divide by 4' rule for interpreting logistic regression coefficients.)

\section{Discussion}

\begin{enumerate}
  \item Summary of our findings
  \item How do our results mesh with previous related analyses?
  \item Why might we expect to see heavy tails in ecological time-series? (mixture of normals or extreme drivers)
  \item Are the observed frequencies by taxonomic class real or an observational phenomena?
  \item Ways forward
  \item Policy implications
\end{enumerate}

\section{Acknowledgements}

Funding: SCA: SFU Graduate Fellowship, NKD: NSERC SFU \ldots, ABC: \ldots, TAB: \ldots

Earth to Ocean's research group for helpful discussions

Global Population Dynamics Database

Silhouettes: \texttt{phylopic.org}: rabbit by Sarah Werning, grey heron by Ardea cinerea, hoverfly by Gareth Monger. All under Creative Commons Attribution 3.0 Unported license.

Compute Canada's WestGrid high-performance computing resources

\bibliography{/Users/seananderson/Dropbox/tex/jshort,/Users/seananderson/Dropbox/tex/ref3}

\clearpage

\section{Tables}

\LTcapwidth=\textwidth
\bibpunct{}{}{;}{a}{}{;}

\begin{small}
\begin{longtable}{>{\RaggedRight}m{2.0cm}>{\RaggedRight}p{3.0cm}>{\RaggedRight}p{7.0cm}>{\RaggedRight}p{2.0cm}}

\caption{Example population dynamic black swans from the Global Population Dynamics Database and descriptions of their causes. Red and blue dots indicate downward and upward events that have a probability of occurring of 0.001\% or less if the population dynamics were explained by a Gompertz model with normally distributed process noise. These populations are a non-random sample that we were able to verify in the primary literature.}\\

\toprule
Time series (log scale) & Population & Black swan description & Reference \\
\midrule

\includegraphics[width=2cm]{sparks/6528} &
Shag,
\textit{Phalacrocorax aristotelis},
UK &
Shortage of nest sites reduced productivity; red-tide event in 1968 caused extreme mortality; no longer a nest shortage; population rapidly increased &
\citep{potts1980}\\

\includegraphics[width=2cm]{sparks/10007} &
Water vole,
\textit{Arvicola terrestris},
UK &
Short-term population cycles from predator interactions combined with longterm environmental cycle caused sharp downswing  &
\citep{saucy1994}\\

\includegraphics[width=2cm]{sparks/7115} &
Fur seal,
\textit{Arctocephalus pusillus},
South Africa &
Strong decreases in harvesting, loss of predators, and diamond mining regulations reducing human traffic caused sharp upswings  &
\citep{shaughnessy1982}\\

\includegraphics[width=2cm]{sparks/10113} &
Willow grouse,
\textit{Lagopus lagopus},
UK &
Parasite and predation effects interacted to cause low years  &
\citep{dobson1995}\\

\includegraphics[width=2cm]{sparks/10162} &
Red grouse,
\textit{Lagopus lagopus scoticus},
UK &
Good environmental conditions produced high numbers and vulnerable populations; bad conditions and overcrowding combined to create crashes  &
\citep{mackenzie1952}\\

\includegraphics[width=2cm]{sparks/1235} &
Wren,
\textit{Troglodytes troglodytes},
UK &
Severe winters where food was submerged in snow caused population crash &
\citep{newton1998} \\

\includegraphics[width=2cm]{sparks/20579} &
Grey heron,
\textit{Ardea cinerea},
UK &
Severe winters in 1929, 1940--1942, and 1962--1963; 1963 event so severe that recovery took three times as long as expected &
\citep{stafford1971} \\

%\includegraphics[width=2cm]{sparks/20580} &
%Chamois, \textit{Rupicapra rupicapra}, Switzerland &
% &
%\citep{brook2006a}\\

%\includegraphics[width=2cm]{sparks/5019} &
%Barbary macaque,  &
% &
%REF\\

%\includegraphics[width=2cm]{sparks/9675} &
%Carrot fly (\textit{Psila rosae}, Finland &
% &
%\citep{markkula1965}\\

%\includegraphics[width=2cm]{sparks/10139} &
%Grey heron (\textit{Ardea cinerea}), UK &
% &
%\citep{stafford1971}\\

\bottomrule
\label{tab:sparks}
\end{longtable}
\end{small}

% reset citation style:
\bibpunct{(}{)}{;}{a}{}{;}

%\clearpage

\section{Figures}

\begin{figure}[htbp]
\begin{center}
\includegraphics[width=\textwidth]{t-nu-eg.pdf}
\caption{An illustration of fitting population dynamics models that allow for heavy tails, represented by the Student-t degrees of freedom parameter $\nu$.
  (a, b) The probability density for t distributions with a scale parameter of 1 and different values of $\nu$.
  Small values of $\nu$ create heavy tails.
  As $\nu$ approaches infinity the distribution approaches the normal distribution.
  For example, at $\nu = 2$, the probability of drawing a value less than -5 is 1.8\%, whereas the probability of drawing such a value from a normal distribution is nearly zero ($2.9\cdot10^{-5}$\%).
  (c--e) Simulated population dynamics from a Gompertz model with process noise drawn from t distributions with different values of $\nu$.
  Coloured dots in panels c and d represent jumpts with less than a 1 in 1000 chance of occurring in a normal distribution.
  (f--h) Estimates of $\nu$ from models fit to the times series in panels c--e.
  Shown are the posterior samples (histograms), median and interquartile range of the posterior (IQR) (dots and line segments), and the exponential prior on $\nu$ (dashed lines).
  Colour shading behind panels f--h illustrates the region of heavy tails.}
\label{fig:didactic}
\end{center}
\end{figure}

\clearpage

\begin{figure}[htbp]
\begin{center}
\includegraphics[width=0.42\textwidth]{nu-coefs-2.pdf}
\caption{
Estimates of how heavy-tailed population dynamics are for \nuCoefPopN\ populations of birds, mammals, insects, and fishes. Small values of $\nu$ ($\lesssim 10$) suggest heavy-tailed dynamics; larger values of $\nu$ suggest approximately normal-tailed dynamics. Vertical points and line segments represent posterior medians and 50\% / 90\% credible intervals for individual populations. Inset plots show probability that $\nu < 10$ (probability of heavy tails) for populations arranged by taxonomic order. Taxonomic orders are sorted by decreasing mean p($\nu < 10$).
Taxonomic orders with three or fewer populations in panel a are omitted for space.}\label{fig:nu-coefs}
\end{center}
\end{figure}

\clearpage

\begin{figure}[htbp]
\begin{center}
\includegraphics[width=0.9\textwidth]{correlates.pdf}
\caption{
Potential correlates of heavy-tailed behaviour (indicated by low $\nu$ values). Shown are (a--c) parameters from the Gompertz heavy-tailed population model, (b) number of time steps, (c) body length, and (d) lifespan.
For the Gompertz parameters, $\sigma$ refers to the scale parameter of the Student-t process-noise distribution, $\lambda$ refers to the population growth rate at an abundance of one, $b$ refers to the density dependence parameter ($1$ is maximally density independent, $0$ is maximally density dependent, and $<0$ is inversely density dependent).
For parameter values, points and lines represent medians and 50\% credible intervals of the posterior.
Circles representing four populations are filled with white and represent a shark, crustaceans, and one gastropod.
}
\label{fig:correlates}
\end{center}
\end{figure}

\clearpage

\begin{centering}
\LARGE
Supporting material\\[1.5em]
\end{centering}

\section{Data selection}

To TODO \ldots

\begin{enumerate}

\item To remove populations with unreliable population indices that could be strongly confounded with economics and sampling effort, we removed all populations with a sampling protocol listed as \texttt{"harvest"} as well populations with the words \texttt{"harvest"} or \texttt{"fur"} in the cited reference title.
\item We removed all populations with uneven sampling intervals. I.e\ we removed populations that didn't have a constant difference between the ``decimal year begin'' and ``decimal year end'' columns.
\item We removed all populations rated as $< 2$ in the GPDD quality assessment (on a scale of $1$ to $5$, with $1$ being the lowest quality data) \citep[following][]{sibly2005, ziebarth2010}
\item Populations with negative abundance values were assumed to be log values and transformed by taking $10$ to the power of the recorded abundance. In many cases this was noted for the population, but not in all cases. We inspected each of these  \totalAssumedLog\ time series to make sure our assumption made sense (Fig.~\ref{fig:log10-assumed}).
\item We filled in all missing time steps with \texttt{NA} values and imputed single missing values with the geometric mean of the previous and following values. We chose a geometric mean to be linear on the log scale that the Gompertz and Ricker-logistic models were fit on.
\item We filled in single recorded values of zero with the lowest non-zero value in the time series \citep[following][]{brook2006a}. This assumes that single values of zero result from abundance being low enough that censusing missed present individuals. We turned multiple zero values in a row into \texttt{NA} values. This implies that multiple zero values were either censusing errors or caused by emigration. Regardless, our population  models were fit on a multiplicative (log) scale and so could not account for zero abundance.
\item We removed all populations with four or more identical values in a row since these suggest either recording error or extrapolation between two observations.
\item We removed all populations without at least four unique values \citep[following][]{brook2006a}.
\item We then wrote an algorithm to find the longest unbroken window of abundance (no \texttt{NA}s) with at least 20 time steps in each population time series.  If there were any populations with multiple windows of identical length, we took the most recent window. This is a longer window than used in some previous analyses \citep[e.g.][]{brook2006a}, but since our model attempts to capture the shape of the distribution tails, our model requires more data.
\item Finally, we removed GPDD Main ID \texttt{20531}, which we noticed was a duplicate of \texttt{10139} (a heron population).
\end{enumerate}

We provide a supplemental figure of all the time series included in our analysis and indicate which values were interpolated (\percImputedPops\% of populations had at least one point interpolated and only \percImputedPoints\% of the total observations were interpolated) (Fig.\ SX). Table~S1 shows the final taxonomic breakdown and the number of populations with interpolated values.

\section{Details on the heavy-tailed Gompertz probability model}

For the Gompertz, the mostly weakly-informative priors I'm using are:
\begin{align*}
b &\sim \mathrm{Uniform}(-1, 2)\\
\lambda &\sim \mathrm{Normal}(0, 10^2)\\
\nu &\sim \mathrm{Exponential}(0.01)\\
\sigma &\sim \mathrm{Half\mhyphen Cauchy} (5).
\end{align*}

See Figure~\ref{fig:priors}.

Priors justification: (TODO this is old)

\begin{itemize}
\item $b$ is bounded just past stationary so we can detect if they are non-stationary will keeping the sampler from wandering off too far

\item $\lambda$ prior (variance = 100) is basically uninformative within the range of expected values for population growth... it allows even a X probability at a value of X

\item $\nu$ is based on \citet{fernandez1998}; they chose a more informative 0.1 value, we chose a less informative 0.01 and justify it based on performance in supplemental figure X (gives X probability of value less than 10) but constrains the sampling somewhat since above 40 or 50 the shape of the t distribution is almost identical and data are not usually informative at these data quantities... sampler would head off to infinity otherwise

\item $\phi$ given a standard deviation of 1 given prior analyses which suggest autocorrelation in these datasets is minimal; prior is weak enough to allow high or low values, but given little information will stay near our expectation; similar approach in \citep{thorson2014a}

\item $\sigma_\mathrm{proc}$ can be justified based on \citet{gelman2006c} and the expected range of this variable in nature from previous studies \citep[e.g.][]{connors2014}
\end{itemize}

We fit our models with Stan 2.4.0 \citep{stan-manual2014}, and R 3.1.1 \citep{r2014}. I'm starting with 4 chains and 2000 iterations with the first 1000 as warmup (i.e.\ 4000 total samples). If rhat is greater than 1.05 for any parameter or the minimum effective sample size is less than 200 for any parameter then I double both the total iterations and warmup and run again. This continued up to 8000 iterations (16000 total samples) \ldots

\section{Simulation testing the model}

Throughout all of this --- show that the model, if anything, under-predicts heavy tails but with sufficient data is unbiased.

2 parts: how many samples from the true population t distribution do you need to detect low nu? And, given that you have a set of deviations in which nu is detectable (effective nu is within 0.5 CV of true nu), can the more complex Gompertz still capture this?

First part: We drew from t distributions with different nu values and mean of 0, scale of 1. We started with 1600 samples and then fitted again at the first 800, 400, 200,  100, 50, 25. Each time we recorded the nu posterior.

Second part: To generate a series of process deviations with an effective nu approximately equal to the true nu, we generated process deviation sets repeatedly and estimated the mean, scale, and nu values each time. We recorded when the estimated nu was within 0.5 CVs of the true nu and used this set of random seed values in our Gompertz simulation. We then fit AR1 Gompertz models to the simulated datasets with all parameters (except nu) set near the median values estimated in the GPDD.

\section{Alternative population models}

\subsection{Ricker-logistic}

We also fitted a Ricker-logistic model:
\begin{align*}
x_t &= x_{t-1} + r_{\mathrm{max}}\left(1 - \frac{N_{t-1}}{K}\right) + b x_{t-1} + \epsilon_t\\
\epsilon_t &\sim \mathrm{Student\mhyphen t}_\nu(0, \sigma),
\end{align*}
\noindent
where  $r_\mathrm{max}$ represents the maximum population growth rate that is obtained when $N$ (abundance) $= 0$. The parameter $K$ represents the carrying capacity and, as before, $x_t$ represents the $\ln$ transformed abundance at time $t$. The Ricker-logistic model assumes a linear decrease in population growth rate ($x_t - x_{t-1}$) with increases in abundance ($N_t$). In contrast, the Gompertz model assumes a linear decrease in population growth rate with increases in $\ln$ abundance ($x_t$) (REF).

To fit the Ricker-logistic models, we chose a prior on $K$ uniform between zero and twice the maximum observed abundance (\citet{clark2010} chose uniform between zero and maximum observed, which is less conservative). We set the prior on $r_\mathrm{max}$ as uniform between 0 and 20 as in \citet{clark2010}. We used the same priors on $\nu$ and $\sigma$ as in the Gompertz model.

\subsection{Autocorrelated residuals}

We considered a version of the Gompertz model in which an autoregressive parameter was fit to the process noise residuals:
\begin{align*}
x_t &= \lambda + b x_{t-1} + \epsilon_t\\
\epsilon_t &\sim \mathrm{Student\mhyphen t}_\nu(\phi \epsilon_{t-1}, \sigma).
\end{align*}
\noindent
In addition to the parameters in the original Gompertz model, an additional parameter, $\phi$, is estimated that represents the relationship between of subsequent process noise residuals. Based on the results of previous analyses with the GPDD \citep[e.g.][]{connors2014} and the chosen priors in previous analyses \citep[e.g.][]{thorson2014a} and to greatly speed up chain convergence when running our model across all populations, we placed a weakly informative prior on $\phi$ that assumed the greatest probability density near zero with the reduced possibility of $\phi$ being near $-1$ or $1$. Specifically, we chose $\phi \sim \mathrm{Truncated\mhyphen Normal}(0, 1, \mathrm{min.} = 1, \mathrm{max.} = 1)$.

\subsection{Assumed observation error}

We considered a version of the base Gompertz model that allowed for a specified level of observation error:
\begin{align*}
U_t &= \lambda + b U_{t-1} + \epsilon_t\\
x_t &\sim \mathrm{Normal}(U_t, \sigma_\mathrm{obs})\\
\epsilon_t &\sim \mathrm{Student\mhyphen t}_\nu(0, \sigma_\mathrm{proc}),
\end{align*}

\noindent
where $U$ represents the unobserved state vector, and $\sigma_\mathrm{obs}$ represents the standard deviation of observation error (on a log scale), which was set at $0.2$. TODO justify $0.2$.

%\end{spacing}

\renewcommand{\thetable}{S\arabic{table}}
\setcounter{table}{0}

% latex table generated in R 3.1.1 by xtable 1.7-3 package
% Thu Sep 25 13:22:15 2014
\begin{table}[ht]
\centering
\caption{Summary statistics for the filtered Global Population Dynamics Database time series arranged by taxonomic class. Columns are: number of populations, number of taxonomic orders, numbers of species, median time series length, total number of interpolated time steps, and total number of substituted zeros.} 
\begin{tabular}{lrrrrrr}
  \toprule
Taxonomic class & Populations & Orders & Species & Median length & Interpolated pts & Zeros pts \\ 
  \midrule
Aves & 227 &  15 & 113 &  26 &  68 &  32 \\ 
  Insecta & 182 &   7 &  91 &  25 &  26 &  55 \\ 
  Mammalia & 125 &   8 &  51 &  28 &  18 &  21 \\ 
  Osteichthyes & 109 &   6 &  36 &  26 &  14 &   4 \\ 
  Crustacea &  13 &   2 &   2 &  28 &   1 &   0 \\ 
  Chondrichtyhes &   1 &   1 &   1 &  20 &   1 &   0 \\ 
  Gastropoda &   1 &   1 &   1 &  21 &   0 &   0 \\ 
   \bottomrule
\end{tabular}
\end{table}


\renewcommand{\thefigure}{S\arabic{figure}}
\setcounter{figure}{0}

\begin{figure}[htbp]
\begin{center}
\includegraphics[width=\textwidth]{log10-assumed.pdf} \caption{Populations in which the abundance was specified as or assumed to be recorded as log10 values. The numbers at the front of each panel label are the GPDD IDs.}
    \label{fig:log10-assumed}
\end{center}
\end{figure}

\begin{figure}[htbp]
\begin{center}
\includegraphics[width=\textwidth]{all-clean-ts-3.pdf} \caption{All filtered times series used in our analysis. The abundances are shown on a log10 vertical axis. Colours indicate taxonomic classes. Open black circles indicate abundance values that were interpolated. Closed black circles indicate single abundance values that were recorded as zero but were set to the next lowest observed abundance.}
    \label{fig:all-ts}
\end{center}
\end{figure}

\clearpage

\begin{figure}[htbp]
\begin{center}
\includegraphics[width=\textwidth]{priors-gomp-base.pdf} \caption{Probability
  density of the Bayesian priors for the base Gompertz model. From left to right: (1) per capita growth
  rate at $\ln$(abundance) = $0$: $\lambda \sim \mathrm{Normal}(0, 10^2)$; (2) scale
  parameter of t-distribution process noise: $\sigma \sim \mathrm{Half\mhyphen
    Cauchy} (2.5)$; (3) t-distribution degrees of freedom parameter: $\nu \sim
  \mathrm{Exponential}(0.01)$. (4) AR1 correlation coefficient of residuals: $\phi \sim \mathrm{Normal}(0, 1)[-1, 1]$. Not shown is $b$, the density dependence
  parameter: $b \sim \mathrm{Uniform}(-1, 2)$.}
  \label{fig:priors}
\end{center}
\end{figure}

\clearpage

\begin{figure}[htbp]
\begin{center}
\includegraphics[width=0.8\textwidth]{t-dist-sampling-sim-prior-exp0point01.pdf}
\includegraphics[width=0.8\textwidth]{t-dist-sampling-sim-sigma-prior-exp0point01.pdf}
\caption{
  Testing the ability to estimate $\nu$ (top panels) and the scale parameter of the process error (bottom panels) for a given number of samples (columns) drawn from a distribution with a given true $\nu$ value (rows). The red lines indicate the true population value. When a small number of samples are drawn there may not be samples sufficiently far into the tails to recapture the true $\nu$ value; however, heavy tails are still distinguished from normal tails in most cases, even with only 25 or 50 samples. TODO switch to nu = black swan, 3, 5, and normal.
}
\label{fig:sim-nu}
\end{center}
\end{figure}

\clearpage

\begin{figure}[htbp]
\begin{center}
\includegraphics[width=\textwidth]{sim-gompertz.pdf}
\caption{Simulation testing the Gompertz estimation model when the process error draws are chosen so that $\nu$ can be estimated close to the true value outside the full population model (``effective $\nu$'' within a CV of 0.2 of specified $\nu$).
  The simulation was run across population $\nu$ values (columns) and different scenarios (colours): (1) 100 time steps and no observation error, (2) 50 time steps and no observation error, (3) 50 time steps and observation error drawn from $\mathrm{Normal} (0, 0.2^2)$ but ignored, and (4) 50 time steps with observation error in which the quantity of observation error was assumed known.
  Dashed horizontal lines show the true population values;
  these true values were chosen to represent approximately the median values as estimated from the GPDD.
  Individual dots and lines represent a stochastic draws from the true population distribution and a model fitting (median and 80\% credible intervals).
  %The panels show true values of $\nu$ =  3, 5, normal (very heavy, heavy, and not heavy tailed).
  The panels from top to bottom show $1/\nu$, process noise scale parameter $\sigma$, growth rate parameter $\lambda$, and the density dependence parameter $b$.}
\label{fig:sim-gompertz}
\end{center}
\end{figure}

\begin{figure}[htbp]
\begin{center}
\includegraphics[width=\textwidth]{sim-gompertz-boxplots.pdf}
\caption{The distribution of median estimates from the same simulation as described in Fig.~\ref{fig:sim-gompertz}. The boxes show the interquartile range. The whiskers extend to $1.5$ times the interquartile range and outliers are shown as dots.}
\label{fig:sim-gompertz-boxplots}
\end{center}
\end{figure}


\begin{figure}[htbp]
\begin{center}
\includegraphics[width=\textwidth]{check-sim-box.pdf}
\caption{The probability that $\nu < 10$ (i.e.\ the approximate probability of heavy tails) for the same simulation scenarios as shown in Fig.~\ref{fig:sim-gompertz} and Fig.~\ref{fig:sim-gompertz-boxplots}. Within each scenario the dots represent stochastic draws from the true population distributions combined with model fits.}
\label{fig:sim-prob}
\end{center}
\end{figure}


\begin{figure}[htbp]
\begin{center}
\includegraphics[width=0.9\textwidth]{ts-gpdd-heavy-eg-log10-base.pdf}
\caption{Time series for populations with p($\nu < 10$) $>$ 0.5 using the base Gompertz population model. Panels are ordered by increasing p($\nu < 10$). Vertical axes are on a log10 scale. Colours indicate taxonomic classes. TODO add dots for upswings and downswings as in Table 1. The labels on each panel indicate p($\nu < 10$) $>$ 0.5, the common name for the species, and the GPDD ID number.}
\label{fig:heavy-ts}
\end{center}
\end{figure}

\clearpage

\begin{figure}[htbp]
\begin{center}
\includegraphics[width=\textwidth]{gomp-comparison.pdf}
\caption{Estimates of $\nu$ from alternative models plotted against the base Gompertz model estimates of $\nu$. Shown are medians of the posterior (dots) and 50\% credible intervals (segments). The diagonal line indicates a one-to-one relationship. Different colours indicate various taxonomic classes. The grey-shaded regions indicate regions of disagreement if $\nu = 10$ is taken as a threshold of heavy-tailed dynamics. The Gompertz observation error model assumes a fixed standard deviation of observation error of $0.2$ on the log scale.}
\label{fig:alt}
\end{center}
\end{figure}

\clearpage


\begin{figure}[htbp]
\begin{center}
\includegraphics[width=0.7\textwidth]{admb-coefs}
\includegraphics[width=0.8\textwidth]{length-vs-prob-tails}
\caption{Potential correlates of heavy-tailed dynamics.
  (Top panels) Dots, thick lines, and thin lines represent fixed-effect estimates and their 50\% and 95\% confidence intervals for the covariates listed on the left: $\log \sigma$, $b$, $\lambda$, $\log$(lifespan), $\log$(data set length). All covariates were standardized by subtracting their mean and dividing by their standard deviation.
  The beta distribution model (left panel) uses p$(\nu < 10)$ as the response (probability of heavy tails).
  The binomial distribution model (right panel) uses p$(\nu < 10) > 0.5$ vs.\ p$(\nu < 10) \le 0.5$ as the response (heavy vs.\ not heavy tails).
  The models were fitted as mixed effects models with beta or binomial distributions, logit links, and multilevel intercepts for taxonomic class and order nested within class using the \texttt{glmmADMB} \texttt{R} package \citep{fournier2012,glmmadmb}.
  (Bottom panels) Model predictions for the strongest covariate --- time series length --- with all other predictors set to zero (their mean value when standardized). Dots represent observations for individual populations. Lines and shaded regions represent estimates and 50\% and 95\% confidence intervals.
}
    \label{fig:correlate-coefs}
\end{center}
\end{figure}

% \clearpage
%
% \begin{figure}[htbp]
% \begin{center}
% \includegraphics[width=0.8\textwidth]{length-vs-prob-tails}
% \caption{Model predictions. TODO}
%     \label{fig:length-predictions}
% \end{center}
% \end{figure}

\clearpage

\noindent
Example Stan code for a heavy-tailed Gompertz model with AR1 correlated residuals and a specified level of observation error:

\begin{spacing}{1.15}
\begin{footnotesize}
\begin{verbatim}
data {
  int<lower=3> N;              // number of observations
  vector[N] y;                 // vector to hold ln abundance observations
  real<lower=0> nu_rate;       // rate parameter for nu exponential prior
}
parameters {
  real lambda;                 // Gompertz growth rate parameter
  real<lower=-1, upper=2> b;   // Gompertz density dependence parameter
  real<lower=0> sigma_proc;    // process noise scale parameter
  real<lower=2> nu;            // t-distribution degrees of freedom
  real<lower=-1, upper=1> phi; // AR1 parameter
  vector[N] U;                 // unobserved states
  real<lower=0> sigma_obs;     // specified observation error SD
}
transformed parameters {
  vector[N] epsilon;           // error terms
  epsilon[1] <- 0;
  for (i in 2:N) {
    epsilon[i] <- U[i] - (lambda + b * U[i - 1])
                       - (phi * epsilon[i - 1]);
  }
}
model {
  // priors:
  nu ~ exponential(nu_rate);
  lambda ~ normal(0, 10);
  sigma_proc ~ cauchy(0, 2.5);
  phi ~ normal(0, 1);
  // data model:
  for (i in 2:N) {
    U[i] ~ student_t(nu,
                     lambda + b * U[i - 1]
                     + phi * epsilon[i - 1],
                     sigma_proc);
  }
  y ~ normal(U, sigma_obs);
}
\end{verbatim}
\end{footnotesize}

\clearpage

\noindent
The GPDD IDs used in our analysis:

\begin{footnotesize}
\noindent
{\tt
1 3 4 5 6 7 8 9 10 11 12 13 14 15 16 17 18 44 45 46 47 58 61 64 1149 1150 1153 1157 1159 1160 1162 1163 1165 1166 1168 1169 1170 1173 1174 1177 1179 1184 1185 1188 1189 1190 1195 1196 1197 1199 1200 1201 1202 1203 1204 1205 1206 1207 1217 1227 1228 1229 1233 1234 1235 1237 1238 1239 1240 1243 1244 1247 1342 1377 1522 1523 1524 1525 1534 1602 1613 1618 1633 1660 1663 1664 1667 1669 1670 1671 1674 1682 1683 1792 1826 1829 1830 1831 1865 1866 1868 1869 1870 1875 1876 1880 1881 1883 1885 1886 1887 1888 1893 1894 1927 1964 1965 1966 1968 1970 1971 1973 1974 1976 1981 1982 1983 1986 1987 1991 1992 1993 1994 1998 1999 2003 2004 2005 2006 2007 2012 2013 2015 2016 2017 2018 2019 2020 2024 2025 2026 2027 2028 2031 2032 2033 2034 2066 2721 2722 2726 2732 2735 2736 2757 2758 2759 2770 2771 2772 2774 2775 2777 2778 2781 2829 2844 2857 2867 2869 2887 2903 2915 2974 2976 2991 3001 3003 3017 3051 3056 3059 3068 3214 3216 3218 3233 3249 3251 3253 3260 3265 3283 3356 3358 3360 3378 3442 3466 3468 3470 3477 3482 3508 3521 3625 3627 3639 3664 3673 3676 3678 3680 3706 3708 3716 3774 3776 3784 3795 3799 3811 3827 3829 3838 3840 3853 3866 3882 5019 5020 5032 5034 5035 5039 6057 6144 6527 6528 6529 6530 6531 6532 6533 6534 6535 6536 6537 6539 6541 6542 6547 6548 6549 6550 6553 6554 6555 6556 6558 6560 6561 6562 6564 6565 6566 6567 6568 6569 6570 6581 6582 6583 6633 6673 6674 6675 6676 6677 6678 6681 6683 6684 6685 6686 6687 6688 6770 6865 6867 6868 6869 6870 6876 6882 6885 6889 6890 6902 6904 6917 6920 6921 6922 6939 6940 6973 7048 7052 7053 7054 7060 7061 7067 7088 7089 7091 7092 7093 7094 7098 7099 7101 7102 7115 7116 9191 9192 9194 9195 9196 9200 9211 9215 9216 9217 9218 9219 9220 9221 9222 9223 9224 9225 9232 9308 9309 9330 9331 9381 9382 9393 9436 9437 9438 9439 9440 9441 9442 9443 9444 9445 9446 9468 9469 9470 9472 9477 9486 9488 9489 9490 9491 9492 9500 9501 9502 9503 9506 9515 9517 9518 9519 9586 9587 9606 9611 9612 9639 9641 9642 9644 9646 9647 9648 9650 9652 9654 9655 9656 9657 9658 9659 9661 9662 9663 9665 9667 9668 9669 9672 9673 9674 9675 9676 9677 9678 9679 9680 9681 9682 9688 9689 9690 9691 9767 9768 9769 9770 9771 9772 9773 9774 9775 9776 9777 9778 9793 9794 9795 9796 9797 9799 9800 9801 9802 9803 9804 9805 9806 9807 9808 9809 9810 9811 9812 9813 9814 9815 9816 9817 9818 9819 9820 9821 9822 9823 9824 9825 9826 9827 9828 9829 9830 9831 9835 9836 9893 9894 9895 9896 9897 9898 9899 9900 9901 9902 9903 9904 9905 9907 9919 9921 9932 9933 9934 9936 9938 9948 9949 9950 9951 9953 9990 9991 9993 9994 9995 9997 9998 9999 10000 10001 10002 10005 10006 10007 10008 10009 10010 10011 10012 10013 10029 10030 10031 10036 10039 10040 10041 10042 10044 10045 10046 10047 10048 10049 10050 10051 10053 10054 10055 10060 10061 10063 10065 10070 10071 10085 10088 10089 10090 10092 10093 10094 10096 10097 10098 10099 10100 10101 10110 10111 10112 10113 10114 10117 10118 10120 10121 10122 10123 10124 10125 10127 10128 10131 10134 10136 10137 10139 10140 10141 10142 10143 10144 10145 10149 10153 10156 10158 10159 10160 10161 10162 10163 10164 10165 20527 20530 20532 20534 20535 20536 20537 20539 20540 20541 20542 20543 20544 20546 20547 20548 20549 20550 20551 20552 20553 20555 20577 20578 20579 20580 20581 20582 20583 20587 20626 20628 20634 20635 20636 20639 20649 20650 20651 20652 20653 20654 20655 20656 20657 20658 20659 20660 20662 20663
}
\end{footnotesize}
\end{spacing}
\end{document}
