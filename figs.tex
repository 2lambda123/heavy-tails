\begin{figure}[htbp]
\begin{center}
\includegraphics[width=\textwidth]{t-nu-eg.pdf}
\caption{An illustration of fitting population dynamics models that allow for heavy tails, represented by the Student-t degrees of freedom parameter $\nu$.
  (a, b) The probability density for t distributions with a scale parameter of 1 and different values of $\nu$.
  Small values of $\nu$ create heavy tails.
  As $\nu$ approaches infinity the distribution approaches the normal distribution.
  For example, at $\nu = 2$, the probability of drawing a value less than -5 is 1.8\%, whereas the probability of drawing such a value from a normal distribution is nearly zero ($2.9\cdot10^{-5}$\%).
  (c--e) Simulated population dynamics from a Gompertz model with process noise drawn from t distributions with different values of $\nu$.
  Coloured dots in panels c and d represent jumpts with less than a 1 in 1000 chance of occurring in a normal distribution.
  (f--h) Estimates of $\nu$ from models fit to the times series in panels c--e.
  Shown are the posterior samples (histograms), median and interquartile range of the posterior (IQR) (dots and line segments), and the exponential prior on $\nu$ (dashed lines).
  Colour shading behind panels f--h illustrates the region of heavy tails.}
\label{fig:didactic}
\end{center}
\end{figure}

\clearpage

\begin{figure}[htbp]
\begin{center}
\includegraphics[width=0.44\textwidth]{nu-coefs-2.pdf}
\caption{
Estimates of how heavy-tailed population dynamics are for \nuCoefPopN\ populations of birds, mammals, insects, and fishes.
Small values of $\nu$ ($\lesssim 10$) suggest heavy-tailed dynamics; larger values of $\nu$ suggest approximately normal-tailed dynamics.
Vertical points and line segments represent posterior medians and 50\% / 90\% credible intervals for individual populations.
Inset plots show probability that $\nu < 10$ (probability of heavy tails) for populations arranged by taxonomic order.
Taxonomic orders are sorted by decreasing mean p($\nu < 10$).
Taxonomic orders with three or fewer populations in panel a are omitted for space.
Red and orange colours highlight populations with a high probability of heavy-tailed dynamics.
}
\label{fig:nu-coefs}
\end{center}
\end{figure}

\clearpage

\begin{figure}[htbp]
\begin{center}
%\includegraphics[width=0.9\textwidth]{correlates.pdf}
\includegraphics[width=0.9\textwidth]{correlates-p10.pdf}
\caption{
Potential correlates of heavy-tailed behaviour (indicated by a high probability that $\nu < 10$).
Shown are (a--c) parameters from the Gompertz heavy-tailed population model, (b) number of time steps, (c) body length, and (d) lifespan.
For the Gompertz parameters, $\sigma$ refers to the scale parameter of the Student-t process-noise distribution, $\lambda$ refers to the population growth rate at an abundance of one, $b$ refers to the density dependence parameter ($1$ is maximally density independent, $0$ is maximally density dependent, and $<0$ is inversely density dependent).
%For parameter values, points and lines represent medians and 50\% credible intervals of the posterior.
Circles representing a limited number of sharks, crustaceans, and gastropods are filled in white.
Median and 90\% credible interval posterior predictions of a beta regression multilevel model are shown in panels a and d where there was a high probability the slope coefficient was different from zero (Supporting Material).
}
\label{fig:correlates}
\end{center}
\end{figure}

\clearpage
