%!TEX root = anderson-etal-blackswan-timeseries.tex

\noindent
\emph{Main message, to be deleted (25 words)}: Black swans are present \&
taxonomically widespread but rare in population dynamics. Extreme climate,
predation, parasites, and their interactions are common causes; intrinsic
drivers are unclear.

\section{Abstract}

Black swans are statistically improbable events that nonetheless occur ---
often with profound implications. While extremes in the physical
environment, such as monsoons and heat waves, are widely studied and increasing
in magnitude and frequency, it remains unclear the extent to which ecological
populations buffer or suffer from extremes. Here, we develop a probability
model to estimate the degree of heavy-tailedness (presence of black swans) in
ecological process noise. We apply our model to \NPops~time series from around
the world across \NOrders~taxonomic orders and seven classes. We find strong
evidence of black swans, but they are rare, occurring in
\overallMinPerc--\overallMaxPerc\%of populations; most frequently for birds
(\AvesRangePerc\%) followed by mammals (\MammaliaRangePerc\%), insects
(\InsectaRangePerc\%), and fishes(\OsteichthyesRangePerc\%). When they occur,
they tend to be driven by climate and severe winters, cycles of parasites and
predators, and interactions between these elements. They are more frequently
detected for populations with longer time series and lower levels of process
noise. We find little evidence of intrinsic life-history covariates. TODO close
with modelling and policy implications

\section{Introduction}

Black swans are unexpected extreme events with potentially dramatic
consequences \citep{taleb2007,sornette2009}. One of the most striking black
swans in ecology was the asteroid marking the mass extinction of the K-T
boundary \citep{harnik2012}. Today, it is the extremeness of climate, in
concert with shifts in mean temperature, that is expected to cause the greatest
societal damage \citep{ipcc2012}. But, while extremes in the physical
environment such as wave height, storm severity, and temperature are present and
widely accepted \citep{gaines1993,katz2005}, it remains unclear the extent to
which ecological systems buffer or suffer from black swans \citep{nunez2012}.

%Existing evidence for black swans or heavy tails in population dynamics is
%mostly anecdotal or for particular systems.
There is compelling anecdotal evidence for ecological black swans, but
systematic evidence has been elusive. A survey of ecologists indicated that
surprising results in field experiments may be far more common than we assume
\citep{doak2008}, and a number of major environmental events on geologic and
contemporary time scales could be considered to have caused ecological black
swans \citep{nunez2012}. Other work has detected heavy-tail dynamics for
particular systems. For example, \citet{segura2013} identified heavy tails in
time series of marine microbe populations and \citet{drake2014} identified
heavy tails in time to extinction for experimental waterflea populations.
However, systematic evidence has been limited. What appeared to be heavy tails
in breeding bird abundance \citep{keitt1998}, could also be explained by 
mixing of species in the original dataset \citep{allen2001}. A subsequent
large scale analysis of the Global Population Dynamics Database (GPDD) found
little evidence for heavy-tail distributions in longer abundance time series
\citet{halley2002}.

There are two key reasons why we may find little evidence of ecological black
swans. First, they might not systematically exist. Indeed, the majority of
model fitting and risk forecasting assumes that population dynamics are not
heavy tailed \citep[e.g.][]{brook2006a,dennis2006,knape2012}. Alternatively,
dynamics might be heavy tailed, but a paucity of statistical methods
appropriate for commonly available series has limited our ability to detect
them. For example: (1) The generalized extreme value (GEV) distribution has
been applied to environmental data \citep[e.g.][]{katz2005}. The GEV describes
the distribution of the most extreme event per time interval (e.g.~heaviest
rainfall per year), but this approach requires long time series that can be
condensed into time intervals. (2) Thin- and heavy-tailed distributions have
been competed with each other in an information theoretic framework
\citep{halley2002}, but this analysis did not quantify the probability of
black swans on a continuous scale or allow for population dynamics. (3) Fitting
a catastrophic mixture distribution state space model allows one to quantify
the probability that population events are extreme on a continuous scale
but is data intensive \citep{ward2007}.

Here we develop a framework for identifying ecological black swans in
population dynamic process noise --- the stochastic jumps from time step to
time step. Our framework quantifies both the magnitude and probability of
heavy-tailed dynamics, allows for a range of population dynamic models, can
incorporate observation uncertainty, and is applicable to commonly available
lengths of abundance time series. We apply our model to hundreds of populations
from around the world across four taxonomic classes to address three questions:
(1) how frequent and strong are black swans in population dynamics across
taxonomy, (2) what are some verified causes of black swans in ecological time
series, and (3) are there characteristics of time series or intrinsic
life-history characteristics associated with heavy-tailed dynamics. We find
a high probability of ecological black swans across a wide range of taxa but in
a limited number of populations. When they occur, extreme climate, predation,
parasites, and their interactions are common documented causes. We find little
evidence that intrinsic life-history characteristics are related to the
probability of black swans.

\section{Methods}

To obtain estimates of the probability and magnitude of black swans, we fitted
population dynamic models to hundreds of abundance time series from around the
world. The key addition of our models, compared to typical population models,
was estimating the shape of the process noise tails. We then compared how these
estimates related to taxonomy, time series properties, and life-history
characteristics, and for a subset of populations, we documented the established
causes of ecological black swans.

\subsection{Time-series data}

We obtained abundance time series from the Global Population Dynamics Database
(GPDD) \citep{gpdd2010}. The GPDD contains nearly 5000 time series of abundance
across $\sim$1000 species and $\sim$100 taxonomic orders. To derive
a high-quality subset of populations that were suitable for our analyses, we
filtered the data (Supporting Material). Our filtering removed populations from
less reliable data sources, removed those without sufficient data for our
models, and interpolated some missing values. Our final dataset contained
\NPops~populations across \NOrders~taxonomic orders and seven taxonomic classes
with a median of \medianTimeSteps~time steps (range of
\minTimeSteps--\maxTimeSteps) (Table S1, Fig.~\ref{fig:all-ts}).

\subsection{Population models}

Our main analysis focuses on the commonly applied Gompertz population dynamics
model \citep[e.g.][]{knape2012,dennis2014,connors2014}. The Gompertz model
represents population growth as a linear function in $\log$ space. If we let
$x_t$ represent the $\log$ abundance ($N$) at time $t$, we can represent the Gompertz
model as:
\begin{align*}
x_t &= \lambda + b x_{t-1} + \epsilon_t\\
\epsilon_t &\sim \mathrm{Normal}(0, \sigma^2).
\end{align*}
The parameter $\lambda$ represents the expected $\log$ number of new
individuals at time $t + 1$ if $N_t = 1$. The model is density independent if
$b = 1$, maximally density dependent if $b = 0$, and inversely density
dependent if $b < 0$. The process noise $\epsilon_t$ is modelled as normally
distributed with standard deviation $\sigma$. In previous analyses of the GPDD,
the Gompertz has proved to be the most frequently parsimonious population model
\citep{brook2006}.

We extended the standard Gompertz model by allowing the process noise to be
drawn from a Student-t distribution with degrees of freedom parameter $\nu$,
and scale parameter $\sigma$:
\begin{align*}
\epsilon_t &\sim \mathrm{Student\mhyphen t}(\nu, 0, \sigma).
\end{align*}
If $\nu$ is small ($\lesssim 10$), the t distribution has much heavier tails
than a normal distribution. For example, at $\nu = 2$, the probability of
drawing a value more than five standard deviations below the mean is $1.8$\%,
whereas the probability of drawing such a value from a normal distribution is
nearly zero ($2.9\cdot10^{-5}$\%). As $\nu$ approaches infinity the
distribution approaches the normal distribution. By estimating the value of
$\nu$ we can quantity how heavy-tailed the process noise deviations are
(Fig.~\ref{fig:didactic}).

We chose weakly informative priors to incorporate our understanding of
plausible population dynamics \citep[Supporting Material]{gelman2014}. For
$\nu$, we chose an exponential prior with rate parameter of $0.01$ truncated at
values above two (Fig.~\ref{fig:priors}) --- a slightly less informative prior
than suggested by \citet{fernandez1998}. This prior gives only a 7.7\%
probability that $\nu < 10$ but constrains the sampling sufficiently to avoid
wandering off towards infinity --- above approximately $\nu = 20$ the
t distribution is so similar to the normal distribution that time series of the
length considered here are unlikely to be informative about the precise value
of $\nu$ (Fig.~\ref{fig:didactic}).

We estimated all models in a Bayesian framework using Stan \citep[][version
2.4.0]{stan-manual2014} via the R computing environment \citep{r2014}. Stan
samples from the posterior distribution with an adaptive version of Hamiltonian
Markov chain Monte Carlo called the No-U-Turn Sampler and generally obtains
less correlated samples than algorithms such as the Gibbs sampler
\citep{hoffman2014}. We assured that chains had sufficiently converged and the
sampler had obtained sufficient independent samples from the posterior
($\widehat{R} < 1.05$, $n_\mathrm{eff} > 200$; Supporting Material). We used
simulated data to test how easily we could detect $\nu$ given different sample
sizes and to ensure we could recover unbiased parameter estimates from the
Gompertz model (Supporting Material, Fig.~\ref{fig:sim-nu},
\ref{fig:sim-gompertz}, \ref{fig:sim-gompertz-boxplots}).

We fit four alternative population models to test if they systematically
changed our conclusions. (1) Autocorrelation has been suggested as a reason for
increased observed variability of abundance time series through time, which
could create apparent heavy tails \citep{inchausti2002}; therefore, we fit
a model that modelled the serial correlation in the residuals. (2) Previous
work has modelled abundance or growth rates without accounting for density
dependence \citep{halley2002,segura2013}; therefore, we fit a simpler model in
which we assumed density independence. (3) Observation error could bias
parameter estimates \citep{knape2012} or mask our ability to detect heavy tails
\citep{ward2007}; therefore, we fit a model where we allowed for a fixed
quantity of observation error ($0.2$ standard deviations on a log scale). (4)
The Gompertz model assumes that population growth rate declines linearly with
log abundance. We also fit an alternative model, the Ricker-logistic model,
which assumes that population growth rate declines linearly with abundance
itself (Supporting Material).

\subsection{Covariates of population dynamic black swans}

We investigated possible covariates of heavy-tailed population dynamics
visually and through multilevel modelling. We plotted characteristics of the
time series ($\sigma$, $\lambda$, $b$, and time-series length) along with two
life-history characteristics (body length and maximum lifespan obtained from
\citet{brook2006a}) against our estimates of $\nu$. We formally investigated
these relationships by fitting beta regression multilevel models to the
probability that $\nu$ was less than $10$ (i.e.~approximately the probability
of heavy tails). For covariates that were derived from Gompertz model parameter
estimates, we incorporated standard deviations around the means. To account for
broad patterns of phylogenetic relatedness, we allowed for varying intercepts
at the taxonomic class, order, and species level (Supporting Material).

Finally, we investigated a sample of populations that our method categorized as
having a high probability of heavy tails (Pr$(\nu < 10) > 0.5$). Where
possible, we found the documented causes of ecological black swans in the
primary data source cited in the GPDD or in other literature describing the
population. These populations were chosen haphazardly with the purpose of
generating hypotheses about the causes of population dynamic black swans.

\section{Results}

We found strong but rare evidence for black-swan dynamics. Defining black-swan
dynamics as a greater than $0.5$ probability that $\nu < 10$, we observed black
swans most frequently for birds (\birdPH\%) followed by mammals (\mammalsPH\%),
insects (\insectsPH\%), and fishes (\fishPH\%) (Fig.~\ref{fig:nu-coefs},
\ref{fig:heavy-ts}). Black swans were taxonomically widespread, occurring for
at least one population in \POrdersHeavy\% of the taxonomic orders recorded.
Accounting for time series length and partially pooling inference across
taxonomic class and order with a multilevel model, there was stronger evidence
for black swans in insect populations than is apparent in
Fig.~\ref{fig:nu-coefs} --- four of 10 orders with the highest median
probability of heavy tails were insect orders --- however, there was
considerable uncertainty in these estimates (Fig.~\ref{fig:order-estimates}).

The majority of our heavy-tailed estimates were robust to alternative
population models and observation error (Fig.~\ref{fig:alt}). Including an
autocorrelation structure in the residuals, modelling population growth rates
without density dependence, or modelling the population dynamics as
Ricker-logistic did not systematically alter our conclusions. Allowing for
a fixed quantity of observation error decreased our estimates of black-swan
dynamics, increasing the median estimate of $\nu$ from $<10$ to $>10$ in
$\baseNuTenObsTenSwitch$ of $\baseNuTen$ populations although the majority of
$\nu$ estimates remained qualitatively similar (Fig.~\ref{fig:alt})

Across populations, the probability of observing black swan dynamics was
positively related to time-series length and negatively related to magnitude of
process noise ($\sigma$) but not clearly related to population growth rate
($\lambda$), density dependence ($b$), or maximum lifespan
(Fig.~\ref{fig:correlates}, \ref{fig:correlate-coefs}). Longer time-series
length was the strongest covariate of observing black swan dynamics (as was
similarly observed in simulation testing, Fig.~\ref{fig:sim-nu}). For example,
we were about \pIncHeavyNThirtyNSixty~times more likely to observe heavy tails
in a population with 60 time steps than in one with 30 time steps
(Fig.~\ref{fig:perc-inc-p}). However, the absolute change in probability with
increased time series length was small ($0.\pHeavyNSixty$ vs.\
$0.\pHeavyNThirty$ in the previous example, Fig.~\ref{fig:correlates}).

We assembled the documented causes of black swan population dynamics in seven
populations with high probabilities of heavy tails (Table 1). The majority of
documented events were downward black swans and the majority involved
a confluence of events. For example, a shortage of nest sites for the shag
long-necked bird in the UK was thought to reduce population productivity and
make the population vulnerable when a red tide hit in 1968 \citep{potts1980}.
The population experienced a black swan downswing, but the new availability of
nest sites resulted in a rapid population upswing \citep{potts1980}. As
another example, an interaction between environmentally and predation-mediated
population cycles are thought to have caused a downward black swan for a water
vole population \citep{saucy1994}. Other black swans were the result of
a sequence of extreme climate events on their own. For example, severe winters
in 1929, 1940--1942, and 1962--1963 were associated with black swan downswings
in grey heron abundance in the UK \citep{stafford1971}. Our analysis finds
that the last event was a combination of two black swan events in a row and it
took the population three times longer to recover than predicted
\citep{stafford1971}.

\section{Discussion}

We found strong evidence that the process noise in ecological time series was heavy-tailed under the Gompertz or Ricker-logistic models in \overallMinPerc--\overallMaxPerc\%
of populations assessed. We detected black swans more frequently for longer
time series and for populations with a smaller magnitude of process error, as
we might expect. However, we failed to find strong evidence that black swans
were associated with density dependence, population growth rate, or lifespan.
In cases that we verified, black swans were often a result of the interaction
between elements of extreme climate, predators and parasite cycles, and strong
changes in human pressures. Our empirical results, sensitivity analyses, and
simulation tests suggest that estimating the tail shape of process noise is
a viable method of detecting black swan population dynamics and if anything
will underestimate the probability of black swans. The presence of black swans
highlights the importance of developing management strategies that detect
quickly, respond to, and are robust to extremes in population dynamics ---
particularly as the frequency and magnitude of climatic extremes increase over
the next century \citep{easterling2000,ipcc2012}.

Our results clarify previous related analyses. Previous work with an older
version of the GPDD assessed the distribution of abundance time series but
focused on identifying if the lognormal distribution was the most frequently
parsimonious model \citep{halley2002}. For heavy-tailed distributions,
\citet{halley2002} fit the extremely heavy-tailed Cauchy distribution and the
four-parameter Levy stable distribution and found little information criteria
support for these distributions in longer time series. However, black swan
events are by definition rare, and the majority of time series in the GPDD are
short for these purposes. Therefore, we would not expect to observe black swans
in a large proportion of populations. Furthermore, by probabilistically
quantifying process noise tails on a continuous spectrum, and allowing
for non-stationary time series and density dependence, our analysis allows for
a more nuanced description of the evidence of ecological black swans. In an
earlier study, \citet{keitt1998} described heavy (power law) tails in breeding
bird population abundance. However, this finding was challenged by
\citet{allen2001}, who showed that mixing data across species could create
falsely generate heavy tails. This rebuttal supports our use of the
t distribution to represent heavy tails, since a t distribution can be
represented by a mixture of normal distributions \citep[with
inverse-gamma-distributed variances,][]{gelman2014}.

We might expect to observe black swan dynamics in ecological time series
because of either unmodelled intrinsic properties of populations or extrinsic
forces acting on populations. Since a t-distribution can be formed by a mixture
of normal distributions \citep{gelman2014}, we could observe heavy tails if we
miss some underlying mixture of intrinsic processes \citep{allen2001}. That
process might be an aggregation of populations across space, or population
diversity, or some intrinsic change in population variability through time.
Extrinsic forces could also cause black swan dynamics
\citep[e.g.][]{nunez2012}. These forces could be extreme themselves. For
example, extreme climate, predation from or competition with other species
experiencing black swans, or sharp changes in human pressure such hunting,
fishing, or habitat destruction might cause black swans. Alternatively, the
interaction of multiple ``normal'' extrinsic forces could give rise to black
swan ecological dynamics. This could occur if the interaction nature is
synergistic \citep[e.g.][]{kirby2009} or even if non-synergistic interactions
experience a rare alignment \citep{denny2009}.

There are a number of caveats when considering the generality of our results.
First, the GPDD data represent a taxonomically and geographically biased sample
of populations --- the longer time series we focus on are dominated by
commercially and recreationally important species and a disproportionate number
of populations are located in the United Kingdom. Although we expect that we
would find strong but rare evidence for black swans in other large taxonomic or
geographic samples, the common forces driving those black swans (e.g.~severe
winters in Table~1) likely differ. Second, in a large database with disparate
data sources, we cannot ensure the quality of every data point;
it is likely that some of the black swans in our dataset are 
recording mistakes mistakes.
Conversely, recording mistakes could mask heavy tails if initial observations are discarded or altered as assumed errors. Third, the temporal scales
of observation and population dynamics vary considerably across populations in
the GPDD and this likely influences the detection of heavy tails. For example,
if we make frequent observations relative to generation time (e.g.~for many
large-bodied mammals) we will average across generations and perhaps miss
black swans. Furthermore, if we census populations infrequently relative to
generation time (e.g.~many insects in the GPDD) the recorded data may average
across extreme and less-extreme events and dampen black swan dynamics.

Our results suggest important future issues to address about ecological black
swans. Given that black swans do occur, can we forecast the probability of
black swans in space and time? Furthermore, what management policies allow us
to detect them quickly after they happen? Can we isolate the
components of ecological dynamics that experience black swans events by moving
from phenomenological models such as the Gompertz to mechanistic models, that
take into account, for example, recruitment? We expect that greater insight
into the mechanisms and covariates of ecological black swans may be best
obtained through specific geographic and taxonomic subsets of data where longer
time series are available \citep[e.g.][]{segura2013}. Finally, what is the
impact of allowing for black swans in forecasts of ecological risk? Financial
analysts are concerned with the shape of financial-return-distribution tails
because these impact estimates of risk --- the probability of a specific
magnitude of undesired event occurring \citep{rachev2008}. It is possible that
a comparable focus in ecology would alter our perception of ecological
forecasts such as extinction risk.

Recognizing the prevalence of heavy-tailed dynamics suggests a number of policy
directions. First, ecological resource management can draw from other
disciplines focussed on heavy tails. For example, earthquake preparedness and
response is focussed on black swan events. Similarly to ecological black swans,
we can rarely predict the specific location and timing of large earthquakes.
But, earthquake preparedness involves spatial planning based on forecast
probabilities to focus early detection efforts and develop disaster response
plans (e.g.~REF).
A similar focus could benefit resource management once our ability to
predict the spatial probability and covariates of ecological black swans
improves. Second, we can choose management policy that is robust to heavy
tails. For example, setting target population abundances that are appropriately
far from critical limits may buffer black swan events
\citep[e.g.][]{caddy1996}, and maintaining genetic, phenotypic, and behavioural
diversity may allow some components of populations persist when others are
affected by disease or extreme environmental forces
\citep[e.g.][]{schindler2010, anderson2014}. Finally, extreme and unexpected,
surprising, or counterintuitive ecological dynamics offer a tremendous
opportunity to learn about ecological systems, evaluate when our models break
down, and tune future management policy \citep{doak2008, pine-iii2009,
  lindenmayer2010}. In recent decades ecology has moved toward focussing on
aspects of variance in addition to mean responses \citep[e.g.][]{loreau2010a,
  thompson2013}. Our results suggests that an added focus on extremes
represents the next frontier, particularly in the face expected increases in
climate extremes \citep{meehl2004,ipcc2012}.

\section{Acknowledgements}

We thank Justin D. Yeakel, TODO, and other members of the Earth to Ocean
research group for helpful discussions and comments. We are grateful to the
contributors and maintainers of the Global Population Dynamics Database and to
Compute Canada's WestGrid high-performance computing resources. Three
silhouette images were obtained from \texttt{phylopic.org} under a Creative
Commons Attribution 3.0 Unported license: rabbit (Sarah Werning), hoverfly
(Gareth Monger), bird (Jean-Raphaël Guillaumin {[}photography{]} and T.
Michael Keesey {[}vectorization{]}). Funding was provided by a Simon Fraser
University Graduate Fellowship (SCA), the Natural Sciences and Engineering
Research Council of Canada (NKD, ABC), the Canada Research Chairs Program
(NKD), and TODO (TAB).

\section{Supplementary Material}

The following supplementary material is available online for this article:\\
Figures S1--S11\\
Table S1\\
Example Stan code for a heavy-tailed Gompertz model and multilevel beta
regression\\
The GPDD IDs used in our analysis\\
All code and data to recreate our analysis are available at:\\
\url{https://github.com/seananderson/heavy-tails}

\bibliographystyle{ecologyletters}
\bibliography{/Users/seananderson/Dropbox/tex/jshort,/Users/seananderson/Dropbox/tex/ref3}

\clearpage

\section{Tables}

\LTcapwidth=\textwidth
\bibpunct{}{}{;}{a}{}{;}

\begin{small}
\begin{longtable}{>{\RaggedRight}m{2.0cm}>{\RaggedRight}p{3.0cm}>{\RaggedRight}p{7.0cm}>{\RaggedRight}p{2.0cm}}

\caption{Example population dynamic black swans from the Global Population
  Dynamics Database and descriptions of their causes. Red and blue dots
  indicate downward and upward events that have a probability of occurring of
  0.001\% or less if the population dynamics were explained by a Gompertz model
  with normally distributed process noise. These populations are a non-random
  sample that we were able to verify in the primary literature.}\\

\toprule
Time series (log scale) & Population & Black swan description & Reference \\
\midrule

\includegraphics[width=2cm]{../analysis/sparks/6528} &
Shag,
\textit{Phalacrocorax aristotelis},
UK &
Shortage of nest sites reduced productivity; red-tide event in 1968 caused
extreme mortality; no longer a nest shortage; population rapidly increased &
\citep{potts1980}\\

\includegraphics[width=2cm]{../analysis/sparks/10007} &
Water vole,
\textit{Arvicola terrestris},
UK &
Short-term population cycles from predator interactions combined with longterm
environmental cycle caused sharp downswing  &
\citep{saucy1994}\\

\includegraphics[width=2cm]{../analysis/sparks/7115} &
Fur seal,
\textit{Arctocephalus pusillus},
South Africa &
Strong decreases in harvesting, loss of predators, and diamond mining
regulations reducing human traffic caused sharp upswings  &
\citep{shaughnessy1982}\\

\includegraphics[width=2cm]{../analysis/sparks/10113} &
Willow grouse,
\textit{Lagopus lagopus},
UK &
Parasite and predation effects interacted to cause low years  &
\citep{dobson1995}\\

\includegraphics[width=2cm]{../analysis/sparks/10162} &
Red grouse,
\textit{Lagopus lagopus scoticus},
UK &
Good environmental conditions produced high numbers and vulnerable populations;
bad conditions and overcrowding combined to create crashes  &
\citep{mackenzie1952}\\

\includegraphics[width=2cm]{../analysis/sparks/1235} &
Wren,
\textit{Troglodytes troglodytes},
UK &
Severe winters where food was submerged in snow caused population crash &
\citep{newton1998} \\

\includegraphics[width=2cm]{../analysis/sparks/20579} &
Grey heron,
\textit{Ardea cinerea},
UK &
Severe winters in 1929, 1940--1942, and 1962--1963; 1963 event so severe that
recovery took three times as long as expected &
\citep{stafford1971} \\

%\includegraphics[width=2cm]{../analysis/sparks/20580} &
%Chamois, \textit{Rupicapra rupicapra}, Switzerland &
% &
%\citep{brook2006a}\\

%\includegraphics[width=2cm]{../analysis/sparks/5019} &
%Barbary macaque,  &
% &
%REF\\

%\includegraphics[width=2cm]{../analysis/sparks/9675} &
%Carrot fly (\textit{Psila rosae}, Finland &
% &
%\citep{markkula1965}\\

%\includegraphics[width=2cm]{../analysis/sparks/10139} &
%Grey heron (\textit{Ardea cinerea}), UK &
% &
%\citep{stafford1971}\\

\bottomrule
\label{tab:sparks}
\end{longtable}
\end{small}

% reset citation style:
\bibpunct{(}{)}{;}{a}{}{;}


\begin{figure}[htbp]
\begin{center}
\includegraphics[width=\textwidth]{../analysis/t-nu-eg.pdf}
\caption{
An illustration of fitting population dynamic models that allow for heavy
tails, represented by the Student-t degrees of freedom parameter $\nu$. (a, b)
The probability density for t distributions with a scale parameter of 1 and
different values of $\nu$. Small values of $\nu$ create heavy tails. As $\nu$
approaches infinity the distribution approaches the normal distribution. For
example, at $\nu = 2$, the probability of drawing a value less than -5 is
1.8\%, whereas the probability of drawing such a value from a normal
distribution is nearly zero ($2.9\cdot10^{-5}$\%). (c--e) Simulated population
dynamics from a Gompertz model with process noise drawn from t distributions
with different values of $\nu$. Coloured dots in panels c and d represent jumps
with less than a 1 in 1000 chance of occurring in a normal distribution. (f--h)
Estimates of $\nu$ from models fit to the times series in panels c--e. Shown
are the posterior samples (histograms), median and interquartile range of the
posterior (IQR) (dots and line segments), and the exponential prior on $\nu$
(dashed lines). Colour shading behind panels f--h illustrates the region of
heavy tails.}
\label{fig:didactic}
\end{center}
\end{figure}

\clearpage

\begin{figure}[htbp]
\begin{center}
\includegraphics[width=0.44\textwidth]{../analysis/nu-coefs-2.pdf}
\caption{
Estimates of how heavy-tailed population dynamics are for \nuCoefPopN\
populations of birds, mammals, insects, and fishes. Small values of $\nu$
($\lesssim 10$) suggest heavy-tailed dynamics; larger values of $\nu$ suggest
approximately normal-tailed dynamics. Vertical points and line segments
represent posterior medians and 50\% / 90\% credible intervals for individual
populations. Inset plots show probability that $\nu < 10$ (probability of
heavy tails) for populations arranged by taxonomic order and sorted by
decreasing mean Pr($\nu < 10$). Taxonomic orders with three or fewer
populations in panel a are omitted for space. Red to yellow points highlight
populations with a high probability of heavy-tailed dynamics.
} \label{fig:nu-coefs} \end{center} \end{figure}

\clearpage

\begin{figure}[htbp]
\begin{center}
\includegraphics[width=0.9\textwidth]{../analysis/correlates-p10.pdf}
\caption{
Potential covariates of heavy-tailed population dynamics (indicated by a high
probability that $\nu < 10$). Shown are (a--c) parameters from the Gompertz
heavy-tailed population model, (b) number of time steps, (c) body length, and
(d) lifespan. For the Gompertz parameters, $\sigma$ refers to the scale
parameter of the Student-t process-noise distribution, $\lambda$ refers to the
expected log abundance at the next time step at an abundance of one, $b$
refers to the density dependence parameter ($1$ is maximally density
independent, $0$ is maximally density dependent, and $<0$ is inversely density
dependent).
%For parameter values, points and lines represent medians and 50\% credible
%intervals of the posterior.
Circles representing a limited number of sharks, crustaceans, and gastropods
are filled in white. Median and 90\% credible interval posterior predictions of
a beta regression multilevel model are shown in panels a and d where there was
a high probability the slope coefficient was different from zero (Supporting
Material).
}
\label{fig:correlates}
\end{center}
\end{figure}
